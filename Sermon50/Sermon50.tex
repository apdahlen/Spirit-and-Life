

\section{Fourth Sunday after Epiphany: In Faith and in Doubt}


\begin{quote}

Matthew 14:22–33: And immediately Jesus made his disciples get into the boat and go ahead of him to the other side, while he dismissed the crowds. And after he had dismissed the crowds, he went up on the mountain by himself to pray. When evening came, he was there alone. But the boat was already far out on the sea, distressed by the waves, for the wind was against them. And in the fourth watch of the night Jesus came to them, walking on the sea. And when the disciples saw him walking on the sea, they were terrified and said, “It is a ghost,” and they cried out in fear. But immediately Jesus spoke to them, saying, “Take courage; I am here. Do not be afraid.”

And Peter answered him, “Lord, if it is you, command me to come to you on the water.” He said, “Come.” And Peter got out of the boat and walked on the water toward Jesus. But when he saw the strong wind, he was afraid, and beginning to sink he cried out, “Lord, save me!” Jesus immediately reached out his hand and took hold of him, saying to him, “You of little faith, why did you doubt?” And when they got into the boat, the wind ceased. And those in the boat fell before him and said, “Truly you are the Son of God.”

\end{quote}

\bigskip




This narrative sets before us the apostles of Jesus in the hard and yet blessed school of the cross, as a pattern for the disciples of Jesus and for his followers and servants in all times.

Jesus had revealed his divine power by feeding five thousand men, besides women and children, with five loaves and two fish. The twelve apostles had been his willing, obedient, and believing instruments in that great miracle. And the people had become glad and uplifted when they received food in the wilderness in a wondrous manner, just as ancient Israel in the days of Moses.

Then enthusiasm arose among the people; a jubilant murmur ran through the assembly: He is the prophet who is to come; him we will have as king. And did not the twelve also feel a blessed rapture at their Master’s glorious self-revelation, and at the same time a secret joy in observing the powerful impression that Jesus’ work had made upon the people? So do we often rejoice when we see something similar of God’s work, and when it seems to us that now it goes well, now many are being won for Jesus, now all the people are gathering around him.

There was faith among the twelve at this moment; therefore Jesus wanted them to depart from the people, lest the people’s carnal enthusiasm should become a temptation and a fall for them. Therefore he also willed that they should be separated from him outwardly, so that they might learn to believe without seeing. Therefore “he immediately compelled them to get into the boat and go over to the other side, while he dismissed the crowds.” He himself wished to be alone to pray for the people and for the twelve, who each in his own way were placed in a dangerous and difficult situation, that it might yet be granted them to understand the Lord’s meaning with the great miracle.

The twelve apostles set out upon the sea according to the Lord’s command. Alone in the darkness of the night they were to row over to the other side. It was not far, and the men were accustomed to the sea; there seemed to be no danger at all. Nevertheless it became for them a peculiar trial. For the twelve found themselves in that distinctive state of mind which the miracle and the people’s enthusiasm had produced, a half-spiritual, half-carnal uplift, in which they felt as though everything must yield, and as though all obstacles must give way, since now the kingdom of God was coming with power, and they were to be its bearers under the leadership of Jesus.

Jesus saw that they needed to be tested. Therefore he sent them out alone against the storm, which he himself had caused, and they labored and toiled the whole night against wind and waves, and yet made almost no progress. It was a hard night after a long day, when the work had not felt heavy because they had been borne by spirit and faith, but which had nevertheless wearied them with that kind of exhaustion that is felt doubly when the tension is over and the miracle’s excitement has faded. And when they were in the middle of the sea, the boat was in danger, and the hearts of the disciples were discouraged and fearful. Darkness and storm, labor and the raging waves had taken away the festive mood, and there was not much left of their courage and faith.

But Jesus, who was upon the mountain in prayer for them, saw them in their distress and came to their aid. Yet he used his own way and manner. They were to reach the place to which he had sent them, but they were not to reach the goal without him. If they had trusted at all in their own strength when they began the journey across the sea without having Jesus in the boat, they had now learned their own weakness. And now Jesus came to them in a new miracle: he came walking on the sea. Then icy fear seized them, and the Savior who came to help them became for them a ghost, which foretold death rather than their salvation. But Jesus spoke kindly to them and said, “Take courage; I am here. Do not be afraid.” Thus it was he himself who came; there was the Lord, there was the Savior, there was the friendly helper who came, he who had watched over them with fatherly love and who was ready to help with divine power. In the hour of weakness he himself was near to them with miraculous power.

Then Peter’s faith awakened with overflowing strength: “Lord, if it is you, command me to come to you on the water.” He thought that he too would take part in this miracle, as he had taken part in the miracle of the feeding. And the Lord, who rejoiced in his faith and wished that he should gain more experience, said to him, “Come.”

And indeed, the incredible happens: Peter steps out of the boat; he who just before had been afraid, though he had the boat between himself and the water, now walks out upon the raging sea to come to Jesus. At first it goes well. The sea bears him—no, faith bears him upon the sea; but then Peter looks at the sea and the waves, and as his gaze leaves Jesus, fear again slips into his heart, and with fear, doubt; faith no longer bears him, and he sinks into the angry waves.

But Peter is not to go under. He is given time to utter the cry of distress: “Lord, save me!” And immediately Jesus takes his hand and leads him safely into the boat and says, “You of little faith, why did you doubt?” Then the storm subsided outwardly and inwardly, upon the sea and in the hearts, and with holy trembling all in the boat fell down before the mighty Savior and said, “Truly you are the Son of God.”

\textbf{As the little boat upon the Sea of Galilee with the precious band of disciples on board, so is the Church of God upon earth since Jesus has ascended and is no longer visibly present among his believers.} The world into which Jesus has sent them, while he himself has gone to the Father, is a hostile world. Dark and threatening as the storm cloud in the west at evening stood the black, thick paganism against which the Lord sent the disciples when he commanded them to go out into all the world and preach the gospel to all creation. And it was not long before apostles and evangelists came to experience that “the wind was against them,” and that there was toil and struggle with the spiritual hosts of wickedness under heaven, which enlisted princes and kings on earth into their service in order to crush and annihilate the little Church of God.

But in this distress of the Church there is continually this blessed consolation for all God’s true and genuine children and for all sincere workers, that Jesus is in prayer for them with the Father; he watches the little struggling host; and when distress rises to its highest point, and faith is on the verge of yielding to doubt, then he himself comes near, walking over the surging and roaring sea of the peoples of the world; and when human wickedness lifts itself high and seems as though it would crush the Church, then at that very moment the Lord reveals himself and brings help to the terrified in their need. Thus the death of the martyrs is both the world’s threat and at the same time a blessed testimony of the Lord’s nearness; and through the firmness of stones and the flames of the pyres sounds his own loving voice: “Take courage; I am here; do not be afraid.”

And when again and again the servants of the Lord have ventured forth against the violent passions of human hearts, both within the Church and among the pagans, then they have indeed had to repeat Peter’s experience, that they sank whenever they looked away from the Lord and toward the many difficulties; but no one who has tried to raise the cry of distress like Peter, “Lord, save me!” has cried in vain. The Lord has heard also the cry of the one of little faith—yes, of the doubting one—and immediately helped and spoken the reproving and yet blessedly consoling word: “You of little faith, why did you doubt?”

Why? In truth you have no reason for doubt if your situation is like Peter’s. If it is at the Lord’s command that you walk upon the sea, then pay heed only to the Lord’s command and the Lord’s eye, which so kindly follows you the whole time while you are placed in such great danger. And above all, if you feel that you are beginning to sink, then do not neglect, even in the very moment of doubt, to cry out to your Savior, and behold, he will immediately stretch out his hand to you and hold you up.

There are so many fearful souls among us who walk so timidly and therefore so close to sinking. They see resistance all around them, and they feel doubt and discouragement and unworthiness within themselves. Hear the voice of Jesus: “Do not be afraid; only believe!” “You of little faith, why do you doubt?” Jesus, who has commanded us to walk in the world with his testimony against the world, nevertheless sees all our path and all our danger, and he sees no reason why we should doubt. If we see something that frightens us, it is because we see too low and too short; we do not look high enough up to the Lord, nor far enough forward to the glory with God.

“I live, and you shall live,” says Jesus to his believers. Amid the many dangers and great hardships of the wilderness journey, let us hold fast to the invisible one as though we saw him. And even if we are led in strange ways through storm and calm, let us hold fast to the faithful Savior’s hand, which is stretched out to us through the dark storm clouds that seem to close heaven entirely over us. And if we should even let go of his supporting hand for a moment, so that we come to know that we sink unless he holds us up, then let us remember that such experience is beneficial for us, so that we may once again learn the ancient cry of distress: Lord, save us! Amen.


