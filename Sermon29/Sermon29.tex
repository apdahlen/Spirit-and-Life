

\section{Sunday after New Year: A Treasure in Heaven}


\begin{quote}

Luke 12:32–34. “Do not fear, little flock; for it has pleased your Father to give you the Kingdom. Sell what you have and give alms. Provide yourselves with purses that do not grow old, with a treasure that does not fail in heaven, where no thief approaches and no moth destroys. For where your treasure is, there your heart will be also.”

\end{quote}

\bigskip


The words ‘Do not fear, little flock’ appear here in a striking context. A man had come to Jesus in order to take advantage of his selflessness and great influence among the people. He demanded that his brother divide the inheritance with him and wanted Jesus to compel the brother to do so. “Take heed and beware of covetousness; for even if someone has abundance, his life does not consist in his possessions,” was Jesus’ answer. Then he told the parable of the rich farmer and, on that basis, exhorted especially his disciples not only against all anxiety for earthly things, but also against all pursuit of them.

“Seek first the Kingdom of God and his righteousness, and all these things shall be added to you.”

This was one of the sharp blows by which Jesus often struck down the disciples’ earthly expectations of the Messiah’s kingdom, and by which they were even “greatly astonished,” as when, on the occasion of the rich young man, he said: “It is easier for a camel to go through the eye of a needle than for a rich man to enter the Kingdom of God” (Mark 10:25).

“Who then can be saved?” the disciples asked in their terror and despondency.

Therefore the Lord applies his heavenly balm to the wound when he here says:

“Do not fear, little flock; for your Father has been pleased to give you the Kingdom.”

But immediately thereafter he adds again: Sell what you have! Give alms! Provide yourselves a treasure in heaven, so that your hearts may be there; and do not be “unfruitful or barren” (2 Peter 1:8), but let your loins be girded and your lamps burning, and—above all—watch! watch!

So it is with the Kingdom of grace: renunciation, labor, struggle, and vigilance go hand in hand with grace. Not that they can produce grace; for then it would no longer be grace (Romans 4:4). But where these are absent, it is because grace has not been received; where works are lacking, it is because faith is lacking (James 2).

Oh, how the self-righteous and the sharp-witted scribes have striven to separate these two things from one another—some by building on works and despising grace, others by resting in grace and mocking works.

But here the Lord has united the two and forced the disciples’ fleshly mindset to face itself, just as Paul also says: “We are glorified with Christ, if indeed we suffer with him” (Romans 8:17).

“For the Lord desires truth in the inward being,” and “blessed is the one in whose spirit there is no deceit.” If you say that you are willing to receive by grace the forgiveness of sins, life, and salvation—yes, heaven itself—from God, and yet in sincerity are not willing to use the power of that grace to do the little that the Lord asks of you: to suffer and labor with him through a short earthly life, a small breath of eternity, or to renounce and relinquish, for the glory of heaven, the small, petty things that are called earthly riches, covetousness, and honor—then you have neither learned to know nor to value the treasure of grace in heaven, nor have you sincerely received it. You deceive yourself and others, and the truth is not in you.

For if God has forgiven you ten thousand talents, would it be something great to forgive a brother, or even an enemy, a few dollars? And if you neither can nor will do this, can you then sincerely pray, “Forgive me my debt,” when the Lord nevertheless says: “If you do not forgive others their trespasses, neither will your Father forgive your trespasses” (Matthew 6:15)?

And what sincerity is there in saying that you have your life with Christ in God, and yet being ashamed or shrinking back from letting it be revealed among people, when Jesus says: “Whoever does not confess me before people, I will not confess before my Father,” and Paul adds: “There is therefore now no condemnation for those who are in Christ Jesus, who walk according to the Spirit and not the flesh.”

\emph{But for the one} who is willing and ready\\ 
to do everything the Savior asks of him\\
though in weakness,\\ 
yet in sincerity of heart—

\emph{for the one} who with joy is willing\\ 
to convert all earthly possessions\\ 
into spiritual values and in all he does asks: 

\textbf{How can this serve my Savior’s purpose?}

\emph{for the one} who strives to be merciful\\
as our heavenly Father is merciful,\\ 
to love as Jesus loved,\\ 
and not to close his eyes to a brother’s need,\\ 
but to help and relieve and heal where God gives opportunity,\\ 
and who in all earthly things sees only the perishable and fragile,\\ 
which thieves steal and moths consume,\\ 
and therefore has his gaze and his heart\\ 
continually turned toward the invisible and eternal,\\ 
having received the Kingdom of grace in humble faith—

\emph{for the one} who thus feels his foreignness\\
and abandonment in the world,\\ 
his smallness and weakness, \\
the unceasing danger\\ 
and pursuit by the world and its prince—\\
oh, how blessed and precious is this word: 

\textbf{“Do not fear, little flock!”}

Do not fear, you worm Jacob—you little flock of Israel, I am your Redeemer—do not fear, Mary; what is impossible for people is possible for God—do not fear, you of little faith, for I still the storm—in all distress and danger, spiritual and temporal, in tribulation and persecution, in despondency and abandonment, he stands before you, who has all authority in heaven and on earth, and says: 

\textbf{“Do not fear!”}

You are little, that is true; but you are a flock, that is just as true—you congregation of God. You have a shepherd who went into death for you, and who rendered powerless the one who has the power of death; a shepherd whom God has subjected all things under and set as head over the congregation, his body, his fullness, who fills all in all (Ephesians 1:22–23).

He says to you, O congregation—to you, you poor, little, anxious soul who flees to him as the chick under the hen’s wings: 

“Do not fear!”

Why? “Because your Father has been pleased to give you the Kingdom.”

Oh, what blessed words! What balm for wounded hearts; what strength in the weak; what victory for the downtrodden!

Your Father—that is what the holy, righteous, and almighty God is for the little ones who have believed in the one whom he sent. A Father who says: “Though a woman forget her nursing child, yet I will not forget you.” A Father who, without any merit of yours, but solely out of his fatherly goodness and mercy, has been pleased to have compassion on you in your lostness and misery of sin, and despite your enmity and resistance has had compassion on you, so that he has given this his only-begotten Son into death, so that everyone who believes in him shall not perish.

Since this unspeakably loving Father, according to an eternal decision, has been pleased in the Son to have mercy on poor sinners, he now stands here in the Word and by his Holy Spirit with hands full of grace and bestows upon a world-harried, anxious soul “everything that pertains to life and godliness through the knowledge of him who called us by his glory and power.” He gives freely and abundantly without money and without payment, without merit and without works, to everyone who in his distress and misery sinks down before him and childlike and sincerely asks for it.

And what does he give? The Kingdom. Nothing less. Precisely that which the disciples at this moment feared to lose because they understood it carnally; precisely that which a hungry and thirsty human soul in its inmost being both desires and needs—not goods or gold, not the glory and honor of the world, nor even their opposite, but God in us: God with all that he is and has, with the Son and the Holy Spirit dwelling in the heart of a poor and helpless sinner; with cleansing from all sin; with the right of children in God’s house; with authority over everything God has, according to the word: “Whatever you ask in my name, you shall receive”; with the same power to overcome Satan and his entire kingdom with which God raised the Son from the dead; with an incorruptible, undefiled, and unfading inheritance kept in heaven; and a hope that does not put to shame, because the love of God has been poured into our hearts through the Holy Spirit—this is the Kingdom, this is the “treasure in heaven,” this is the unspeakable gift which it has pleased the Father to give to Jesus’ disciples, precisely because they are small, a little flock, and precisely because they are anxious and concerned about their own weakness and helplessness in the midst of an evil world, so that at all times they need to hear this blessed word of his Holy Spirit: 

“Do not fear!”

What then shall we say, friends? “If God is for us, who can be against us? He who did not spare his own Son but gave him up for us all—how shall he not also with him graciously give us all things?” Oh yes; “for I am convinced that neither death nor life, neither angels nor rulers nor powers, neither the present nor the future, neither height nor depth, nor any other created thing will be able to separate us from the love of God in Christ Jesus our Lord.”

Is this “treasure in heaven” worth some small renunciation, to deny ourselves and all that is dear to us in the world? Or is there anything in the world that, either for the individual soul or for the whole congregation of God, can outweigh the blessedness of being able to appropriate this word: 

“Do not fear, little flock; 

for your Father has been pleased to give you the Kingdom”?



