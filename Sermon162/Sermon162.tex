

\section{Third Sunday after Trinity: The Publican Matthew}


\begin{quote}

Matthew 9:9–13. And as Jesus went on from there, he saw a man sitting at the toll booth, called Matthew; and he said to him: Follow me! And he rose and followed him. While he was at table in the house, many publicans and sinners came and sat at table with Jesus and his disciples. And when the Pharisees saw it, they said to his disciples: Why does your Master eat with publicans and sinners? But when Jesus heard it, he said to them: The healthy have no need of a physician, but those who are sick. Go and learn what this means: I desire mercy and not sacrifice. For I have not come to call the righteous, but sinners to repentance.

\end{quote}

\bigskip

Matthew was a publican; and from the toll booth Jesus called him with a gracious yet authoritative word: Follow me! From the toll booth Matthew rose and followed Jesus.

It is not with many or grand words that Matthew relates this serious moment in his life. He is not like those preachers whose pleasure it seems to be to proclaim their own history of conversion. He passes over it so swiftly, as though he were afraid to dwell upon himself, in order to hasten to tell what Jesus further did and said on this occasion. For it is the person and word and work of Jesus which so seized Matthew that it was impossible for him to remain sitting in the toll booth when the call sounded: Follow me! With a ready and willing heart, the poor sinner followed him who had just said to the afflicted sufferer upon the bed of pain: Son, take courage! Your sins are forgiven you. Rise, take your bed, and go to your house! The word of Jesus swept through Matthew’s heart — swift as thought, powerful as love. The toll booth and the accounts saw him no more. He rose and left it all. From that moment he was Jesus’ disciple and would become his apostle. He belonged to Jesus, and Jesus to him. Therefore his speech and writing are about Jesus, not about himself.

This is one of the small beginnings of that which is great in the Kingdom of God. For indeed we do not know much about Matthew; but this we know, that the forgiven publican has written the Gospel according to Matthew; and when we know this, then we know also that the Lord has made the publican Matthew one of the chief instruments in the service of his Kingdom.

Or perhaps you are one of the world’s careless skeptics who disdainfully says or thinks: The Gospel of Matthew—is that anything great? Friend, have you read what you despise? Begin to read; continue to read; read until you come to the end; read until every section and every narrative stands living before your thought; read until the whole Gospel becomes one mighty testimony for your heart concerning the Kingdom of God and its power to overcome sin and death and to give the life of love and the hope of the resurrection; then you shall learn to think otherwise both of the Gospel of Matthew and of the power of God which is therein revealed.

If the Gospel of Matthew is a poem, as the wise of the world now say, then Matthew is a great poet, behind whom the other poets of the world may creep into hiding and conceal themselves, shrinking into their own smallness; but if the Gospel of Matthew is God’s living truth, as the Spirit clearly bears witness, then Matthew is nothing other than a humble instrument in the service of Jesus’ love and in the surpassing greatness of his Spirit’s power.

Matthew himself surely felt that it was a great day for him when Jesus looked upon him and called him; for from Luke we hear that Matthew, who is also called Levi, made a great feast for Jesus in his house, and that thus Jesus came to sit at table with many publicans and sinners.

Then the Pharisees murmured. They felt anger — the sharp sense that their honor was being diminished, that here matters were unfolding which deprived them of their honor and their praise. Should publicans and sinners perhaps receive a share in the Kingdom of God? Would the new Prophet truly not keep company with the Pharisees and bow to their party?

But to the Pharisees’ murmuring question to the disciples: Why does your Master eat with publicans and sinners? Jesus himself answers with words which to this day have become a flowing fountain of consolation for all poor sinners; but also a stumbling stone and a rock of offense for all proud and self-righteous hearts:

The healthy have no need of a physician; but those who are sick. Go and learn what this means: I desire mercy and not sacrifice! For I have not come to call the righteous, but sinners to repentance.

% FIXME: offensive and scandalous. Explore other options.
 
Jesus is a physician; therefore it is not strange to see him with the sick and among the sick. There he is in his proper place. It is strange that anyone can take offense at so self-evident a matter. But where love has departed from the hearts, compassionate love becomes offensive and intolerable. The “pure” cannot associate with the “impure”; the “righteous” cannot draw near to sinners; pride and self-righteousness are sinners which thrive best in their own company, and which are even capable of imagining that this company is also that of God and of Jesus Christ.

But Jesus is a physician; therefore do not fear, you who are sick! Remember that it is precisely for the sake of your sickness that Jesus has come; were you whole and sound, then you would have nothing to do with him, and he nothing with you. Therefore fear the sickness; for truly it is a sickness unto death, and hasten to send prayerful word for the physician; he will quickly come and heal you.

For this is his will, as it is written: I desire mercy and not sacrifice.

In these words the whole blessedness of Christianity is contained. Take these words to heart, and you will grow as a Christian the more they fill your mind and your whole soul. God will give, not take. That is the point. Our own understanding and all pagan religions and all Jewish Phariseeism and all human moralism would persuade us that God requires sacrifice from us, requires that we should give him something, that we should show him worship, that we should serve him, that we should do something for him, then God will be satisfied with us, then he will reward us, then he will do something for us again, when we have first done something for him.

But the Gospel of Jesus Christ is this, that God desires mercy and not sacrifice. Go and learn what this means, says Jesus. 

This is a blessed, saving doctrine which strikes down all Pharisaic pride. 

God desires mercy. 

Do not come with sacrifice and gift to him. 

He is the one who will give; he is the one who will help. 

Therefore come with a broken heart and a contrite spirit and hear what the Spirit says: In this is love, not that we have loved God, but that he has loved us and has sent his Son to be an atonement for our sins. That is mercy. All is gift, all is grace; the only thing with which you can come before God is an emptied heart, the poverty of spirit. Come poor and hungry and thirsty as you are. The Kingdom of Heaven will be yours; righteousness, peace, and joy in the Holy Spirit will fill your soul.

Then you turn from your sins. For Jesus’ grace and mercy, the forgiveness of sins and the gift of the Holy Spirit, do not serve that you should remain in sin, but that you should be freed from sin. Christ’s call to sinners is a call to repentance. He seeks the company of sinners, not in order to partake in their sinful joy, but in order to make sinners penitent, that they might partake in his heavenly and holy joy.

May the Lord grant that we know our sickness, so that we may know our physician and his healing.

