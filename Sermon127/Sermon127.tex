
\section{Fourth Sunday after Easter: The Living Church}

\begin{quote}
John 7:37–39: Now on the last, the great day of the feast, Jesus stood and cried out, saying: If anyone thirsts, let him come to me and drink. He who believes in me, out of his inmost being shall flow, as the Scripture says, rivers of living water. But this he said concerning the Spirit, whom those who believed in him were to receive; for the Holy Spirit was not yet, because Jesus was not yet glorified.
\end{quote}

\bigskip

The words of Jesus in this text point both backward to the strange experiences of old Israel during the wilderness journey with its precious promises, and forward to the birth of the new Israel, the Church, and the fulfillment of all promises on the day of Pentecost:

If anyone thirsts, let him come to me and drink!

The whole Feast of Tabernacles was a remembrance of the wandering through the wilderness and of the many works of love which the Lord there performed toward his people; it was Israel’s great festival of joy. Throughout the feast there was such gladness and jubilation that the rabbis said: he who has not been present at a Feast of Tabernacles does not know what joy is. In particular there was on each of the seven feast days a ceremony which is intimately bound up with the words here spoken by the Lord, and which was also each day received with indescribable jubilation. One of the priests went at the head of the whole people down to the Pool of Siloam, filled a golden vessel with water from it, brought it up into the forecourt amid the shouts of the people and the sound of drums and trumpets, and offered it there together with wine as a drink offering, while the people, accompanied by drums and trumpets, sang the well-known words from the twelfth chapter of Isaiah:

You shall draw water with joy from the wells of salvation.

After these seven days of noise and jubilation there followed a day of quiet Sabbath rest; this is the day which is called the last, the great. Under excessive and clamorous joy seriousness can so easily be forgotten; therefore the Lord uses the stillness of this last day to gain a hearing and to direct the attention of the people to the true meaning of those words of Isaiah which they had sung with such great enthusiasm, and therefore says:

If anyone thirsts, let him come to me and drink.

In Jesus — Mary’s and Joseph’s Son from Nazareth — is the true well of salvation; in him is the fulfillment of all prophecy.

From the fleshpots of Egypt and the sweet waters of the Nile the people of Israel had been led out into the wilderness. In a moment of holy desire and longing they had let God lead them to give up all the abundance of Egypt in order to be delivered from yoke and bondage, to serve the living God in freedom, and one day to reach the promised Canaan. But many of them had not considered that there lay a long, desolate, perilous stretch between Egypt and Canaan. Hunger came, and thirst tormented them and their beasts; then the longing of the heart began to steal back toward the abundance of Egypt, and they murmured against Moses and against God.

Give us water to drink! they cried in their distress; in their need they tempted God and said: Is the Lord among us or not?—although they had seen his works in Egypt and at the Red Sea.

But the Lord had not forsaken them; he had compassion upon them and let Moses strike the Rock with the staff, and water flowed out, as Paul also says: They all drank of the spiritual Rock that followed them, and the Rock was Christ.

For all God’s mighty works are spiritual, because they have regard to the salvation of our souls; and when the Lord caused water to run from the Rock for the fainting people, it was not only that they should quench their bodily thirst, but that they should turn and look toward the Rock who was to come.

Most of them did not do so; and even when they had entered the promised land, they often forgot the Lord, the Rock with the living water, in order to turn to their own broken cisterns; and God’s severe judgments came upon them as in the wilderness, until they again were driven out of the land which the Lord had promised them.

Then there was again hunger and thirst among the poor people. The Lord has forsaken me, the Lord has forgotten me, cries Zion in her distress.

Can a woman forget her child? Even these may forget, yet I will not forget you, says the Lord to Zion.

And when the people were near to perishing, he caused it to be proclaimed by the prophet Isaiah:

Ho, everyone who thirsts, come to the waters; and you who have no money, come, buy and eat; come, buy wine and milk without money and without price.

Oh, what a prelude to the Gospel of Jesus. All who thirst, no distinction; it is only the question whether you thirst. Then all is ready. The fountain is there, living, springing. It is he who is to come, the Messiah; only look to him as the dying in the wilderness looked to the bronze serpent, and your life shall be saved, your thirst slaked. For the prophet says further:

I will pour water upon the thirsty land and streams upon the dry ground; I will pour my Spirit upon your offspring and my blessing upon your descendants.

Thus he pointed for the fainting people beyond the Rock in the wilderness to the spiritual Rock, which is Christ; and thus Jesus stands here on the last, the great and quiet day of the Feast of Tabernacles, pointing both to the Rock in the wilderness and to the prophecies of the prophets, and says:

If anyone thirsts, let him come to me and drink.

Yes, much more:

The rivers of living water of which the Scripture, as cited above, speaks, shall flow out of the life of those who come to him; for he who drinks of the water that I shall give him shall never thirst again, but it shall become in him a fountain of water springing up unto eternal life.

And yet even this, with all its blessed fulfillment, was still only a promise of yet more glorious things. For, as John says, he spoke concerning the Spirit, whom those who believed in him were to receive. The Savior was not yet glorified; Gethsemane and Golgotha still lay between; he must first pass through death and the grave and hell; he must first return to his Father in heaven and sit at his right hand before he could perform the greatest of all wonders, that which no eye has seen and no ear heard and which has not arisen in the heart of man: to pour the love of God into a poor, lost soul by the Holy Spirit—himself personally present. This same Spirit sets himself at the door of a human heart, knocks and enters in, makes his dwelling there and communes with it, gives peace for strife, healing for all wounds, the right of sonship and inheritance, and with his blessed Savior’s hand he strikes upon the rock-wall of the heart, so that there wells forth within an eternal fountain of life. Therefore Jesus says in the Gospel for the fourth Sunday after Easter: It is to your advantage that I go away; for if I do not go away, the Comforter shall not come to you; but if I go, I will send him to you.

To receive both the Father and the Son in one’s heart through the Holy Spirit, that is what happens to everyone who believes; that is what kings and prophets desired to see, and angels longed to look into; that is what in the Word and in the Church is offered to every poor lost sinner.

And thus the Savior stands today in the midst of us and cries:

If anyone thirsts, let him come to me and drink.

Do you thirst? Will you come? As the children of Israel fainted in the wilderness and cried for water, so the soul lies wasting and dying in the wilderness of the world. In vain it seeks life and relief now in a righteous life, now in the pleasure and sin of the world; they are broken cisterns which hold no water, and the soul is tormented until the anguish of death and the fear of judgment press it into despair or numb it with the clammy cold of spiritual death, the certain foretaste of perdition.

Why will you die, O house of Israel? The Lord has no pleasure in the death of the wicked, but that he turn and live.

Why will you faint and die in the wilderness of the world, poor soul? There is a Savior in Israel; you also may draw water with joy from the wells of salvation.

Shall he cry in vain, the blood-stained Savior who has trodden the winepress alone?

It is sinners he calls. Whether you are righteous before men and a hypocrite before God, or despised and cast out also by men on account of your sins—it matters not—if only you thirst; he stretches out his hand and says: Come! you shall receive the fountain of life in your own heart; relief, healing, new life, the right of sonship, if only you come—without money and without price. There is no obstacle here; Baptism has free access, for the Holy Spirit shines forth.

All is ready; all God’s promises are fulfilled. They are Yes and Amen.

Come! The Lord calls.

The Spirit and the Bride say: Come! And let him who hears say: Come! And let him who thirsts come! And whoever will, let him take the water of life freely.

