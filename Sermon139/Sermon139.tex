

\section{Sixth Sunday after Easter: Fear! Fear not!}


\begin{quote}
% FIXME by removing the KJV tone
Luke 12:4–12: I tell you, my friends: do not fear those who kill the body and after that can do nothing more. But I will show you whom to fear: fear him who, after he has killed, has authority to cast into Hell. Yes, I tell you: Fear him! 

Are not five sparrows sold for two pennies? Yet not one of them is forgotten before God. Even the very hairs of your head are all numbered. So do not fear; you are of more worth than many sparrows. 

I tell you: everyone who acknowledges me before others, the Son of Man will also acknowledge before the angels of God. But he who denies me before men will be denied before the angels of God. And everyone who speaks a word against the Son of Man, it will be forgiven; but he who blasphemes against the Holy Spirit, will not be forgiven. 

And when they bring you before synagogues and rulers and authorities, do not be anxious how or what you will answer in your defense, or what you will say; for the Holy Spirit will teach you in that very hour what you must say.

\end{quote}

\bigskip

In this passage, ‘fear’ is not used in the sense of reverence, of rendering worship, as when we say to fear God, but simply in the sense of being afraid—of real anxiety and dread.

Fear, as well as shame, has since the Fall become part of human nature. The sigh that passes through creation also pierces the human soul, filling it with bodily fear and spiritual anguish—as with Cain. The purpose of this stirring of conscience, which like a trembling moves the innermost being of man, was to drive him, ashamed and humbled, to God, to cry for grace. You can see this impulse even among the wildest pagans.

But as men in their corruption through sin and under the continued influence of the Devilish powers have perverted all things, so we find also that the pagans exchanged the incorruptible God for an image made like corruptible man and even birds and four-footed beasts and creeping things, and in their anguish bent the knee and brought atoning sacrifices to these self-made gods.

So it is today. What men ought not to fear, that they fear; and they make light of the things for which they ought to bear the most serious anxiety. See, here is a young man who has risen into high and wealthy society, where his vanity is flattered, his abilities and riches praised. His affairs, however, are declining without anyone yet knowing it; it will not be possible for him to maintain his brilliant position without becoming a deceiver. What shall he do? If he openly confesses his condition, he knows that laughter and contempt will be his portion, and that his friends will turn their backs upon him. He cannot endure this thought. He fears honesty more than death—and chooses crime. For a short time he still enjoys, as in a frenzy, the brief glitter of self-regard, until all collapses, and he does not truly come to himself until he wakes up in prison.

It is but one of a thousand examples which everyone has occasion to observe in all the fields of life, in small as in great, and not least in one’s own deceitful heart. It is a devilish distortion of the voice of conscience, which daily intrudes even into the Christian's life. And in truth, it is there above all other places that Satan gladly would do harm.

Therefore the Lord speaks these warning words to his disciples. A disciple of Christ must confess. No one knows the Father except through the Son, and no one can call Jesus Lord except by the Holy Spirit. But it is the work of the Holy Spirit to convict the world of sin, righteousness, and judgment, in order that men should not die in their sin but turn and live; therefore Jesus says that when the Advocate comes, he shall testify about me; and you also shall testify.

So it is. The Holy Spirit gives his testimony through men; therefore he dwells in a human heart, that this heart, by the same word through which it was given new life, should testify of the Father and the Son. To testify together with the Holy Spirit concerning the truth of the Gospel, that is to confess Jesus before men. To keep silent about such testimony—or to distort it—is to act against the Spirit’s conviction in the heart. That is the first step toward sinning against the Spirit. For the Christian, it is the first step back toward eternal perdition. That is why it is so dreadful, and why the Lord warns his disciples so earnestly; for it is precisely what Satan desires, what he works at day and night, to cause the Christian to keep silent about or distort or even deny the testimony of God’s Spirit before men; then he gains complete dominion and produces what Jesus in the first verse of this chapter calls the leaven of the Pharisees, which is hypocrisy.

And to this the disciples are exposed: to fear men more than Satan and his deceit.

Jesus knew that the disciples would have to step forward before men with that same testimony for which he himself must suffer the death of reproach upon the cross. He knew that they should stand before synagogues and rulers and those who have authority, that shame and persecution, indeed death, would be their portion; and he knew that they were weak and might come to fear those who have power to kill them, as they themselves fled in Gethsemane. Therefore he opens their eyes to what they should fear and what they should not fear; he knew their deceitful heart.

Fear only one thing, he says; fear the one who can destroy your soul, infect you with the leaven of the Pharisees, and drive you to sin against the Spirit, for which there is no forgiveness either in this world or in the one to come.

Fear him!

But don’t fear men; at most they can kill the body, which in any case one day must die; more they cannot do. So do not let fear of men drive you away from what God requires of you: that you let the Spirit bear witness through you of sin and righteousness and judgment, of grace and love and eternal salvation, that soul after soul may be saved; confess me before men!

Yet if the Lord’s word concerning Satan’s seduction and concerning eternal condemnation should be warning enough against the dreadful danger of acting, in small as in great, against the conviction of God’s Spirit and failing to confess Jesus before men, there is in the Lord’s word to the disciples also a glorious and heavenly encouragement to lay aside all false fear and step forward with the testimony in word and life before men.

First: how should he for whom Jesus was crucified, who has heard his cry of anguish upon the cross, and through the Holy Spirit and a living faith knows that Jesus has paid for all his sins, how should such a soul, so richly graced be able to deny the Lord, his Savior, this one request: Confess me before men! and fear not?

Second: is not the Lord strong enough to defend and protect his own both against men and against the devil? Fear not, you worm Jacob, you little flock of Israel! Fear not, little flock; for it is your Father’s good pleasure to give you the kingdom! He who numbers your hairs and cares for the little sparrows, should he forget you, whom he has engraved on his hands? 

\textbf{Therefore: Confess Jesus before men and fear not!}

Third: if there is an eternal danger of damnation in keeping silent about and denying Jesus before men, then there is—strange to say—a reward, an eternal, heavenly reward, for those who fulfill his loving request to us: to confess him before men. For he says that he will then confess them before the angels of God, before the Father in his glory. O what a reward for bearing the testimony of salvation through the blood of Jesus among men, risking everything—in word and in life—to be glorified with him, to live and reign with him in his kingdom, as he himself rose from the dead, lives, and reigns for all eternity!

This then is his warning and promise to the disciples; it is a warning and promise to all disciples: Fear the devil and the leaven of the Pharisees!—fear not men, but confess my name before them!

Friend, have you had courage to confess Jesus before men? Has it burned in your heart when you have heard his name reviled, as when your dearest or you yourself were mocked or mistreated? Have you wished to be accursed for Christ, if thereby sinners might be drawn to Jesus and saved through his blood? Or do you pass so quietly through the world, so unnoticed, that the world itself loves you and praises you, so that you could not bear in your heart to offend it by testifying to it of sin and righteousness and judgment, or by going to the sinner fallen among robbers, bearing him away and bringing him to shelter with Jesus, the crucified?

O friend, fear not men, but confess Jesus and fear the devil!

For what does it profit a man if he gains the whole world and forfeits his soul? Or what shall a man give in exchange for his soul? For whoever is ashamed of me and of my words in this adulterous and sinful generation, the Son of Man will also be ashamed of them when he comes in the glory of his Father with the holy angels.

Fear therefore him who can destroy both body and soul in Hell! Fear not men, but confess Jesus before the world!

