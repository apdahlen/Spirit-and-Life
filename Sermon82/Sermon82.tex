

\section{Third Sunday in Lent: A Man with an Unclean Spirit in the Synagogue}


\begin{quote}

Luke 4:31–37. And he went down to Capernaum, a city of Galilee, and taught them Sabbath after Sabbath. And they were greatly astonished at his teaching, for he spoke with authority. And in the synagogue there was a man who had an unclean spirit and cried out with a loud voice, “Ah! What business have you with us, Jesus of Nazareth? Have you come to destroy us? I know who you really are—the Holy One of God.” And Jesus rebuked him, saying, “Be silent. Come out of him!” And the devil threw him among them and came out of him without doing him any harm. And terror came upon all, and they spoke with one another, saying, “What is this? For with authority and power he commands the unclean spirits, and they come out.” And the report about him went out throughout the surrounding countryside.

\end{quote}

\bigskip


To a “word with authority” there belongs first a direct and uncompromising word about sin, and then an equally direct word about grace.

Not infrequently are preachers deficient in both; indeed, one may safely say that where there is a lack of power and seriousness in a word that names sin for what it is, there is an equal lack in a word that proclaims grace, and vice versa. At times one speaks to the assembled congregation as though they were all Christians already, about how good it is to be God’s children, how beautiful it is to gather around God’s word, and so forth, until through such proclamation a great many are allowed to lay themselves sweetly down upon their pillow of sin and sleep securely.

At other times, to be sure, sin is spoken of, but in such a manner that everyone feels it concerns those who are outside, and the hearers go away each with a secret pleasure over how powerfully “the others were chastised.”

But at times—yes, often, perhaps always—it happens as in the days of Jesus, that a man with an unclean spirit comes into the synagogue and sits quite calmly and listens. It is precisely this person the preacher must seek out and address, so to speak, in private from the pulpit.

A man with an unclean spirit is one in whom Satan has gained a dwelling, just as the Spirit of God comes to dwell in the one who repents and believes. Such a person has been so ensnared by the devil’s cunning that the sting of conscience is nearly gone. He not only lives in sin, but takes his joy and satisfaction in it. 

Such a person is under an unclean devil’s spirit.

Such a sinful condition in one or several respects can very well coexist with considerable respectability and even with apparent godliness in other matters. Only this distinction is commonly made: if the sin is one of the so-called “gross” sins—that is, such as even secular society condemns or punishes, such as dishonesty, sexual immorality, drunkenness, and the like—then such persons are often expelled both from ecclesiastical and civil associations and regarded by most with aversion, almost with disgust.

But if, on the other hand, the sins are of an inner, more “refined” sort, such as hatred, anger, selfishness, greed, and the like, which make no difference whatsoever before God, then they are not only tolerated; such people are even sometimes set in places of honor within God’s congregation itself, and their sin is excused and glossed over with expressions such as “refined sins,” “sins of weakness,” “besetting sins,” and the like—to irreparable harm both to those concerned and to Christ’s congregation on earth.

While therefore scarcely any congregation would tolerate a murderer or perjurer or thief in its midst, all congregations harbor those who are wrathful, slanderers, and unmerciful, and one excuses them and says of the one, “Yes, poor fellow, he is somewhat hot-tempered,” of the other, “she is a bit loose with her tongue,” and of the third, “he is surely a little tight-fisted.”

That is why every congregation has more than a few who live under an unclean spirit, and unfortunately it is all too often precisely with regard to these that there is lacking that word which accuses the conscience and, like a two-edged sword, pierces down into the depths of the heart, to divide soul and spirit and to judge the thoughts and intentions of the heart.

All too often we are inclined to think of the former—the criminals, the deeply fallen sinners—“It helps nothing”; they do not come to church anyway; and with regard to the others to content ourselves that they are “interested in churchly matters”; “we are, after all, not knowers of hearts.”

But if we are not knowers of hearts, then God’s word, from our heart and in our mouth, is nevertheless well suited to search the innermost being of man and to tear down Satan’s strongholds, and the Lord himself has spoken so weighty and comforting a word, that “tax collectors and prostitutes go into the kingdom of God before the sons of the kingdom.”

Therefore let us not forget the “man with an unclean spirit who has come into the synagogue”; perhaps there sits such a one on every bench; it is precisely the lost sheep, the prodigal son or daughter, whom the Lord wills that you should seek out and lead back to the Father’s house.

Over there sits a drunkard who has become a complete slave to his vice; today he is sober and has by chance come into God’s house. Here is a woman “caught in adultery”; she has almost slipped into the assembly, and had anyone known it, she might perhaps have been denied entrance. Further forward sits a man who has sworn a false oath over a piece of property and is shunned by all. Near by you see a rich man who drove a poor, hungry boy from his door just as he was on his way to church, and at his side a young man who nourishes and feeds his soul day and night with crude and unchaste thoughts.

Alas, how many there are today who have come into the synagogue with an unclean spirit! Leave all the others and speak to these alone; they are well worth it; they have immortal souls purchased with the blood of Jesus. Let the Holy Spirit speak through you—naming sin, righteousness, and judgment plainly, so that it begins to flash and thunder like the voice of the Last Day in these wretched, devil-bound souls, and their eyes are opened so that they see the shame of their nakedness, and their hearts tremble with unrest and fear, and the devil begins to fear losing his prey.

\textbf{Then speak to them of grace and righteousness and blessedness; paint Christ crucified before them; tell how he has loved them, of his bloody sweat, of his hellish agony on the cross and in death, and that he did it all to save them from the devil and from eternal perdition, and that he still stands there with outstretched arms and says, “Come to me, you who are burdened, and I will give you rest.”}

And speak so earnestly and so insistently of this love of Christ that the poor sinners begin to tremble still more, with cries for salvation and with hope, so that the unclean devil who has bound them leaps up in terror before the power of the cross and the fire of the word, and tears and rends them, bound as they are, and out of sheer fear must give testimony to the word: What business have you with us, Jesus of Nazareth? Have you come to destroy us? I know who you really are—the Holy One of God.” 

And what happened in Capernaum shall happen anew among us: the Lord shall again say to such a poor, enslaved sinner, “Be silent, come out of him!” The devil will indeed still tear and rend, but he must come out when the mighty word of the Lord sounds. The Spirit of God comes in instead, speaks the word lovingly to the anxious, trembling heart: “Do not fear!”—shines into all the innermost corners of the heart, sweeps out all the devil’s uncleanness, and says: “Take courage, son; take courage, daughter; your sins are forgiven you!”

Then there is joy among the angels of heaven because a sinner has repented, and fear and stirring among human beings because authority and power have also been given to human beings to cast out unclean spirits; and one and another shall be troubled and begin to test himself and say, “Do I have an unclean spirit? Do I have an unclean spirit?” until throughout the whole congregation there comes a stirring and a storm-wind of the Spirit that cleanses the air and drives the loose and shallow roots to seek deeper and firmer ground in order to be able to stand when the Lord comes.

Therefore let us never forget that there may also be in our synagogue “a man with an unclean spirit,” and that it is the will and command of our precious Savior that we should seek him out in the firm, simple faith that Jesus by his word both can and will save him.

