
\section{Fifth Sunday after Trinity: Peter’s Confession}

\begin{quote}

Matthew 16:13–19: When Jesus came into the region of Caesarea Philippi, he asked his disciples, “Who do people say the Son of Man is?” And they said, “Some say John the Baptist, others Elijah, and others Jeremiah or one of the prophets.” He said to them, “But you—who do you say I am?” Simon Peter answered, “You are the Christ, the Son of the living God.” And Jesus answered him, “Blessed are you, Simon son of Jonah. Flesh and blood did not reveal this to you, but my Father who is in heaven. And I tell you, you are Peter, and upon this rock I will build my church, and the gates of hell will not overcome it. I will give you the keys of the kingdom of heaven, and whatever you bind on earth will be bound in heaven, and whatever you loose on earth will be loosed in heaven.”

\end{quote}

\bigskip

In this text we are told how Jesus asks what fruit his words and works have borne. Yet he divides his hearers into two groups. He first asks what the people say, and then what the disciples say. He expects a better answer from the latter, and he receives it. But in both cases the question is the same: Who do they say that I am? Or more briefly: Who am I? This, therefore, is his true question to all. It is also the question of life and eternity for each one of us. Here every soul on earth must pause and consider: Who is Jesus? What do I know of him, not according to memorized books and sermons, but according to my own personal experience?

Have you ever attempted to give an honest answer to this question? If you have not, then test yourself now and answer before God what your personal knowledge of Jesus of Nazareth really is.

It is striking what the people say about the Son of Man. When Jesus says: What do the people say of me, who am the Son of Man, then it is surely his meaning that people ought to know the Son of Man. He, the true Man who is born into the world, should not be unknown or a stranger among men. People ought to recognize him again, just as a mother knows her son, even if he has been away a long time; just as childhood memories return upon the soil of home when we tread it once more after long, long absence.

But the people did not know the Son of Man. And yet, when they call him John the Baptist or Elijah or Jeremiah, it is clear enough that this wondrous Son of Man awakens the human conscience. A consciousness arose in them: they stood face to face with a man who made them think of sin, the Fall, and lost glory. For what are John the Baptist and Elijah and Jeremiah but the mightiest voices of repentance that have sounded in the human race? Of them people must think when they see and hear Jesus of Nazareth. His presence compels them to feel, whether they wish to or not, that their sin separates them from him and from God, that their sin is their corruption and perdition. Where does that come from, that Jesus’ quiet and blessed activity awakens such thoughts? Why do people hear in his speech a voice of judgment that brings them to remember their sin? 

It is because he is what we ought to be, and are not. It is because the Son of Man is true Man, perfect Man, a Man who still bears the crown that belonged to man from the beginning; he still has the image of God which we have lost. Therefore consciences awaken wherever Jesus draws near; therefore his most gracious words and works are felt as strong testimonies of sin, as powerful reminders of a glory which once was ours, but is no more. Every feature of his life becomes a sharp sword in the heart of the one who loves sin and lives in it.

But when Jesus then turns to the disciples and asks them: But you, who do you say that I am? then there is one of them who can answer rightly. It is he who once in his fishing boat had fallen down at Jesus’ feet and cried: Lord, depart from me; for I am a sinful man. Simon, son of Jonah, had experienced the same as those people who had come into contact with Jesus; and with Simon it had become a complete and full brokenness and surrender, so that Jesus had also raised him up and made him a new man. Therefore he knew Jesus better and altogether differently than the people; he could answer with rock‑firm faith: You are the Christ, the Son of the living God. That was the sum of Peter’s spiritual experience. He could speak words no one had yet spoken. It is a word of overwhelming force; therefore it also receives from Jesus an even mightier answer: Blessed are you, Simon, son of Jonah; for flesh and blood has not revealed this to you, but my Father who is in heaven. And I also say to you, that you are Peter, and upon this rock I will build my church, and the gates of hell will not prevail against it. And I will give you the keys of the kingdom of heaven, and whatever you bind on earth will be bound in heaven, and whatever you loose on earth will be loosed in heaven.

Thus has no human being ever spoken. And indeed one might think and write about each of these words without ever reaching the depths of them. Also, in the history of the Church they have, through misuse and misinterpretation, exercised a distinct influence. That would deserve closer examination. But we must here confine ourselves to what is most essential. And then above all there are three things to observe in these words of Jesus to Simon, which are as serious and as edifying for every other true confessor as for him who first heard them from Jesus.

The first is this, that the right and true confession of Jesus is not produced by flesh and blood, but is a revelation from the Father who is in heaven. Flesh and blood signifies all human ability and human power of whatever name it may be called, whether learning or acuteness, thinking or feeling. All of it falls short. The confession unto salvation of which Jesus here speaks, and of which Paul testifies in Romans, must be grounded in revelation from the Father in heaven. When we learn from the Father to go to the Son and from the Son to go to the Father, and thus driven to the cross and drawn to the mercy seat, find peace with God, and become his children, then only do we rightly know by personal experience and by God’s revelation that Jesus, the crucified and risen one, is the Son of the living God. Then we can confess with the mouth unto salvation.

Friend, has it been so with you? Have you come to the Son drawn and driven by the Father? Have you come to the Father lovingly led by the Son? Have you received the witness of the Spirit in your heart that you are a child of God?  Then you can confess with joy like Simon, son of Jonah: You are the Christ, the Son of the living God; you live and I live in you.

The second is this, that the genuine and true confession of Jesus also makes every confessor a rock and pillar in God’s Church. As Simon, son of Jonah, became Peter through his confession and his testimony, so also will everyone who confesses Jesus to be the Savior, the Christ, the Son of the living God, become a pillar of the truth and a foundation upon earth; and though the storms of hell rage about this rock, they will not overthrow it. Out of the mouth of infants and nursing children the Lord prepares praise and power upon earth, when they with stammering tongue confess Jesus to be the Savior. And this heavenly wisdom, which is hidden from the wise and understanding of the earth, is the Church’s firm and invincible defense against hell’s wickedness and falsehood. This truth prevails at last; blessed is the one who has built upon it in life, for it will not fail him in death.

The third is this, that the true confessor of Jesus Christ receives the keys of the kingdom of heaven to bind and to loose upon earth, so that it also is bound and loosed in heaven. It is a solemn word and a great power. It is the glorious power of God’s liberated children; who can fully describe it? They themselves are Christ’s bound ones and yet Christ’s freed ones, thus they know both bonds and freedom. They belong to Christ, that is their only bond; they are delivered out of the bondage of sin and of the world and of Satan, that is their freedom. Therefore they also can bind and loose with the imperishable keys of the kingdom of heaven. They can testify to the one who without God and without Christ flounders in the life of the world and sin, that there is no salvation in any other, and that there is no other name given under heaven by which we must be saved, except the name of Jesus alone. That is the bond which binds heaven and earth together, and with which every heart must be bound that will be saved. It is a key of the kingdom of heaven which opens and no one shuts, which shuts and no one opens. 

And the Heaven of Heavens will answer Yes and Amen to this bond. They can also testify to all those who seek salvation in this name, that there is freedom and blessed release from all the bonds of sin and of judgment and of hell in this name of Jesus; and that again is a key of the kingdom of heaven, to which Father, Son, and Holy Spirit will bear witness in heaven that it is faithful and true. And through the simple testimony of the true confessors human hearts will be bound and human hearts loosed, and only the Lord’s own Day will fully reveal what has been bound and loosed through the simple confession of the name of Jesus from believing human lips.

Blessed are you, Simon, son of Jonah, and blessed is everyone who confesses that Jesus is the Christ, the Son of the living God. May the Lord himself awaken in our hearts the true and living faith and place the testimony and the confession upon our lips, that we might become his instruments to bind and loose upon earth in Jesus’ name, and that it may be bound and loosed in heaven.

