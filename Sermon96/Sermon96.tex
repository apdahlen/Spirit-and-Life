\section{Maundy Thursday: The Example of Humility}

\begin{quote}
John 13:1–15: Now before the feast of the Passover, when Jesus knew that His hour had come that He should depart out of this world to the Father, having loved His own who were in the world, He loved them unto the end. And during the Supper, when the devil had already put it into the heart of Judas Iscariot, Simon’s son, to betray Him, Jesus, knowing that the Father had given all things into His hands, and that He came from God and was going to God, rose from the Supper and laid aside His garments; and He took a linen cloth and girded Himself. After that He poured water into a basin and began to wash the disciples’ feet and to wipe them with the linen cloth with which He was girded. Then He came to Simon Peter; and he said to Him: Lord, do You wash my feet? Jesus answered and said to him: What I am doing you do not know now, but you shall understand it afterward. Peter said to Him: Never—ever—shall You wash my feet! Jesus answered him: Unless I wash you, you have no part with Me. Simon Peter said to Him: Lord, not my feet only, but also my hands and my head. Jesus said to him: He who has been washed needs only to wash his feet; he is completely clean; and you are clean, but not all. For He knew who would betray Him; therefore He said: You are not all clean. So after He had washed their feet and taken His garments and sat down again, He said to them: Do you know what I have done to you? You call Me Teacher and Lord, and you speak rightly, for so I am. If I then, your Lord and Teacher, have washed your feet, you also ought to wash one another’s feet. For I have given you an example, that just as I have done to you, so you also shall do.
\end{quote}

\bigskip

The hour had come. Jesus knew the Lamb was to be slain for the sins of the world; He knew that the dreadful hour of suffering was to begin, and that it was to begin with the deepest humiliation for Him; for the devil had already put it into the heart of Judas Iscariot, Simon’s son, that he should betray Him. Jesus knew also that of the little flock of eleven chosen apostles there was not one who would stand firm that night. Utterly alone, without any friend on earth for strength or support, He was to suffer what must be suffered for the salvation of the world.

But He knew also that this was the way from this world to the Father, and that He who had come from God and entered the world now left the world again and went to the Father. Therefore His soul was lifted up with blessed joy in this hour of humiliation, so that He could say: Now is the Son of Man glorified, and God is glorified in Him. It was but a little while that He must pass through the valley of the shadow of death; yet through anguish of soul and bodily pain, through crushing shame and mockery from the devil and his tools, the way led forward to eternal and infinite glory, which He was to obtain not for Himself alone, but also for all His weak and wavering friends, whom He had loved while He was in the world and loved unto the end.

Yes, Jesus loved those who had followed Him until now, and who soon were to stand face to face with a hostile world, in which they were to continue the work and the conflict without having Him visibly among them. Jesus loved them, and in His humiliation and in His exaltation thought more of them than of Himself. He would help them in every way to endure in the conflict against the enemies whom He saw threatening them. Therefore He instituted the Supper that night; therefore also He undertook the washing of feet. For He knew that of all the enemies that threatened them, none was more dangerous than the devil of pride, who had just caused a dispute among them as to which of them should be regarded as the greatest. If only this enemy might be kept away from the circle of the Lord’s friends! For truly the experience of the Church through the ages has sorrowfully confirmed what Jesus foresaw, that of all the enemies of God’s congregation on earth none is more dangerous than the spirit of pride and vanity, which stirs in the old Adam so long as there is a shred of strength in him.

There was absolutely nothing of this spirit in Jesus. When the prince of this world came, he had nothing in Him. Jesus had come to serve, and precisely on this night He was to do the fallen world the greatest of all services: to give His life as a ransom for all. Terrible and immeasurable therefore is the difference between Him who is and was and ever shall be the greatest, and His chosen disciples, who even this night could dispute about who was greatest.

Therefore, when the Passover meal had been prepared and the moment had come when, according to Jewish custom, a servant should bring water around to those who reclined at table that they might wash their hands, Jesus rose and went and took the water, as though He were the servant or the slave who should carry it around to the guests. Yet it was not enough for Him to be the servant among the disciples by bringing them water for their hands. That would indeed have shamed them, but it would not have sufficed also to humble and raise them up after their fall.

He would help His own now and for all time by showing them the way and giving them the power to walk in it. Therefore He took the water and went to the one who this time perhaps had received the lowest place at the table, Simon Peter, and was about to kneel at his feet in order to wash them. Then it dawned upon Peter what was to happen here, and in dismay he cried out: Lord, do You wash my feet? He felt the sting in Jesus’ action, but he did not clearly see how bad things were among the disciples and with himself, and he would rather avoid the deep shame which he felt would come if Jesus carried out His purpose.

But Jesus was the steady and strong physician who knew what was needed here. He saw that Peter felt the reproof, but He saw also that Peter did not understand the sin, and how it alone could be taken away. Therefore He answered with the precious word which so often has become a comfort in heavy hours when the way of love seemed so strange: What I am doing you do not know now, but you shall understand it afterward.

Yet Simon Peter still thought that he must prevent the incomprehensible and shameful thing from happening: Never—ever—shall You wash my feet! He supposed that he understood it well enough; he had already learned what it meant, and he would show that he understood his position and had grasped the meaning. He was ready now to throw himself down before Jesus and wash His feet; indeed, if Jesus had now given him the basin and asked him to wash all the disciples’ feet, Peter would gladly have taken Jesus’ place, if only he might escape this humiliation, that Jesus truly should kneel at his feet and wash them. Thus the soul so often resists allowing Jesus to perform the work which He alone can do.

Jesus does not release Peter with half repentance and half restoration. He saw the danger of his beloved disciple’s soul, that he would hasten to come away from his real need without receiving real help. Peter was willing to do something for Jesus now, if only he might avoid that Jesus should do everything for him. Therefore Jesus takes the matter so seriously; He saw that if Peter were allowed to escape in this way, he was not fully prepared in his sorrow and therefore would not draw the true comfort from the right source of comfort, Jesus. Unless I wash you, He says, you have no part with Me.

Then Peter once more awakens with living, storming love for Jesus; then he will have not only feet but also hands and head washed, if it is so that Jesus alone must do all, and that all his fellowship with Jesus rests upon complete surrender and yielding to Him. Yet now it is Jesus who restrains and tempers Peter’s impetuous zeal and lets him see the deep meaning of the symbolic act. What He has willed that Peter should understand, and what Peter now in his willing surrender to the Savior is able to understand, is this: that both he and the other disciples need a cleansing which only Jesus can give them. Even if they belong to Him and already are clean, yet the daily walk in the world will soil their feet, not only because the world around them is unclean, but because the world’s unclean mind still stirs in themselves and is awakened in them through contact with the world. It had indeed just shown itself in the dispute they had with one another about who was greatest. But if Jesus may serve them with His cleansing grace, with the daily forgiveness of sins, then they need not fear that their daily falls shall rob them of their state of grace and their standing as God’s children. If they walk in faith, then the heart is the Lord’s, and He who has cleansed them shall daily cleanse them, so that sin will no longer rule their lives. But one among the twelve is no longer clean; therefore, you who stand, see that you do not fall. Judas is not cleansed from the wickedness of his sinful heart, even though his feet are washed with water by Jesus Himself. For he no longer loves; he no longer needs Jesus. He hates Jesus now, since he has sold Him, and the world’s lust has taken the place of love.

Thus Jesus completes His cleansing service of love upon all His disciples. And never has there been, nor can there be, revealed greater humility, self-abasement, and self-sacrifice than precisely this: that He who is the Lord against whom we sin is Himself the One who bears our sins in His body on the tree; Himself the One who takes away our sins by forgiving and blotting them out; yes, Himself the One who begs to be allowed to serve us in this way, since we so sorely need it. Not because He needed His disciples, but because they needed Him, therefore He lays Himself down at their feet and renders them the service of love.

In truth this deed of Jesus is serious and moving for all believing souls. And yet there are those who do not heed it as they ought. There are those who once needed Jesus and His grace in order to become clean, but who no longer need Him for that matter, because they think they sin no more. Is not this the same error as Peter’s: Never—ever—shall You wash my feet? They manage that matter themselves. But oh, if you would heed the gripping word which Peter received: Unless I wash you, you have no part with Me.

And again there are those who do not take to heart that he who is washed needs only to wash his feet, but is entirely clean. They seem constantly to wish to feed on the excitement of awakening and never to come to the believers’ quiet state and their firm and unshakable standing as God's children. Unhappy souls, who easily can become a prey to the devil’s dreadful power, since they always require the unrest of the first awakening in order to have the violent feelings which have become to them a sign of their conversion. How quickly such souls become like the restless waves of the sea, which are tossed by every wind. They run from one stormy movement into another; and they forget that salvation does not consist in being cast up and down by the foaming waves and now and then touching the Rock with one hand and then slipping out again. No, friend, you must climb up onto this Rock with your feet and take full and complete foothold upon it; then you first experience the peace and assurance which Jesus gives, and then you understand what it is that we have become partakers of Christ, if we hold fast our first firm confidence unto the end.

But Jesus is not finished with His disciples merely because He has washed their feet. He adds an admonition, and how greatly it is needed! What He has now done, in that He has bowed down before each of the disciples and washed their feet, shall be for them an example of humility. We bow in the dust before You, Lord Jesus, and sigh in our hearts: O Lord, how impossible for us to imitate You!

And yet, brothers and sisters, if we are to remain in the fellowship with Jesus and are to have true congregation in spirit and in truth, then we must learn from Him who is meek and lowly of heart. It will not do in the old way, that we rejoice over the neighbor’s faults and exalt ourselves at his expense and thank God that we are not like this publican. It is indeed hard to hear that we are Pharisees and self-righteous, but we are so if we do not wash one another’s feet and from the heart forgive each his brother his trespass. It requires humility to ask one’s brother for forgiveness; but he who has had spiritual experience and has tested his own heart somewhat in the seriousness of life knows that more humility is required, more of Christ’s mind, truly to forgive his brother.

Where shall we obtain this power of humility, this precious grace? With Jesus at the table of the Supper. He gives us Himself in the Supper, that through His body and blood we may become partakers of His life and Spirit. But if we receive Jesus Himself, then we receive His mind and His love, so that we can do toward the brothers what He has done toward us. Oh how differently would God’s life and God’s congregation flourish among us if we became more hungry and thirsty for the food which only Jesus can give, and if thus we truly came to experience that His flesh is truly food and His blood is truly drink.

Therefore come in poverty of spirit and receive from Jesus’ riches, and you shall find that His commandments are not burdensome, since He Himself pours out His love into your heart. Amen.

