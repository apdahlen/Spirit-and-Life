
\section{Epiphany: Not by might and not by power, but by my Spirit}

\begin{quote}

Matthew 12:15–21. And great crowds followed him, and he healed them all. And he strictly charged them not to make him known, in order that what was spoken through the prophet Isaiah might be fulfilled, who says: Behold, my servant whom I have chosen, my beloved in whom my soul delights; I will put my Spirit upon him, and he will proclaim justice to the nations. He will not quarrel nor cry out, nor will anyone hear his voice in the streets. A bruised reed he will not break, and a smoldering wick he will not extinguish, until he brings justice to victory. And in his name the nations will hope.

\end{quote}

\bigskip

Jesus had his own quiet manner of working while he walked here on earth in the days of his flesh. He was a king who had come to establish the kingdom of God and to spread it over the whole earth. He was to conquer the world and bring it under himself. Yet he advanced in an entirely different way from all other kings and conquerors. He did not march at the head of an army; he did not raise a people to an outward struggle for freedom. Around him there resounded neither the raw cries of triumph of victorious warriors, nor the lament of the defeated, the groans of the wounded, the pitiful wailing of the dying. Nor was he the great ecclesiastical prince who gathered around himself the great and small ambitions of the church and by their help advanced his own cause and that of his church.

According to our text, Jesus’ way was wholly different from the way of the world and of the worldly church. And since Jesus’ way is the only right way both in the individual soul and in the congregation and its mission, it is all the more necessary to heed it, the rarer it is that it is actually trodden by those who wish to be Jesus’ disciples and followers.

Jesus works this way: he uses his divine power to help those who truly need it. He does not seek his own honor through his miraculous works. Nor does he seek to satisfy the honor-seeking and vain Jews who waited for the Messiah with such carnal impatience, because through him they hoped to attain the fulfillment of their earthly expectations. Jesus walks the path that leads him to the cross, that incites the Jews against him, but that makes him a Savior for all who labor and are burdened, all who are poor in spirit, all who hunger and thirst for righteousness. For him, everything comes down to this: he will not break the bruised reed or snuff out the smoldering wick.

Think only what victories Jesus could have won if he had called legions of angels to his aid and with them struck down his enemies! Think what enthusiasm and jubilation he could have awakened among the Jews if he had placed himself at their head in the struggle against their oppressors! Think what a jubilant multitude of friends would have followed him if he had grasped earthly power and distributed lands and offices and riches and temporal pleasures to his followers! Think how quickly he could have stretched his scepter over the nations if it had merely been a matter of dominion!

But behold, the greatness of the earth and the glory of the world were far too small for him; and the perishable joy that the world gives was too paltry a gift for his disciples. He would not bow down to the god of this world to gain its kingdoms; nor could he allow his followers to become slaves of the world. He had come to destroy the dominion of the devil, not to fortify it. He had come to free souls, not to enslave them.

Therefore he went forward in a way altogether different from what the Jews desired; therefore his way was so unlike the way of the world. He wished to give God’s salvation in human hearts and eternal life in their souls; he did not wish to delude them with an empty appearance and a continual earthly enjoyment. Therefore he did not step forth in dazzling splendor, nor did he gather around himself the great and rich and learned and righteous and esteemed, so that such respect might surround him and his following that the wretched, miserable sinners would have to shrink back in fear and shame and hide themselves from this glorious one and his proud retinue. Had Jesus chosen what was esteemed in the world and exalted among human beings, had he allied himself with the self-righteous Pharisees, then he would also have driven back into sin and despair the poor, struggling souls who longed for true righteousness and for God’s peace and joy.

Jesus healed them all, says our text; yet at the same time he strictly commanded them not to make him known. He so willingly wished to help; he did not wish thereby to win any honor among human beings. He wished to save; he did not wish to awaken any earthly zeal and fanaticism. He wished to exercise divine mercy toward souls; he did not wish to gather a party against those who said that his work was of the devil. If poor sinners were truly to benefit from his mercy, they must not become vain partisans who boasted in an earthly way of their great leader. Therefore Jesus proceeds in the Spirit of God, and with the most inward tenderness he bends down and raises up the bruised reed and blows upon the smoldering wick, that the blossom may live and the fire burn.

This is the true power, which alone is great enough to save. For this love and Spirit of God alone are mighty to give human hearts true joy and real blessedness. Therefore it is this inner, hidden power of Christ’s death and resurrection that alone effects the salvation of a soul and the spread of God’s kingdom to the ends of the world. Therefore Jesus’ course is so quiet; therefore the working of his Spirit is so mighty. Therefore he is the true glory of Israel and the true light of the nations. Therefore he casts down the pride of the mighty; therefore he lifts up the lowly from the dust.

Jesus’ way is always good. And it is the only way that is good for our souls. Every soul that wishes to experience his salvation must walk this way of his. It is a way of lowliness and humiliation; yet there is no one who can truly become a partaker of God’s glory unless he becomes like the bruised reed and the smoldering wick, an object of the inward mercy of God and of our Lord Jesus Christ. The soul’s way to life and salvation passes through the painful experience of one’s own sin, one’s own unworthiness, and one’s own inability.

So long as a human being lives in his own power and seeks his own honor, greatness, happiness, and joy, he has not yet entered upon Jesus’ way. For all their contempt for the Pharisees, such people are nevertheless very much Pharisees themselves. They do not wish to hear of Jesus; they do not need his grace; they help themselves. And just as they despise grace, so they also despise those who need grace and seek grace with Jesus. They have no thought for the bruised reed and the smoldering wick; they despise such things; they believe in strength and ability; what is weak and sick and fearful and suffering is, in the opinion of these capable and powerful people, of no value. It is not viable, so let it die. That is the cold, Pharisaic wisdom of the world; there is nothing in it of Jesus’ love and God’s truth. Unless God bends such people’s greatness and pride, they will, with all their ability and righteousness, nevertheless have no share in the kingdom of God.

If we are to enter upon Jesus’ way, the way of life and salvation, then we must through a painful experience of our own inability descend into the valley of humility, in order to walk there and remain there. For only those who walk there are on the right way to heavenly glory. God opposes the proud, but gives grace to the humble. But when a human being experiences his sin, his unworthiness, his inability; when he is made to taste what it is to struggle to become righteous by his own power and cannot become so; when he is made to taste what it is to labor to attain honor before God and is continually put to shame; when he breaks all his strength in the struggle against death and condemnation and yet accomplishes absolutely nothing—then he truly becomes a bruised reed and a smoldering wick. Then the proud head is bowed, then the ambitious eye is lowered, then a human being collapses, and he is worth nothing and worthy of nothing in the world; but then, in all his unworthiness, he is an object of the saving grace of Jesus Christ.

A father has a child who possesses the whole love of his soul, and it is to him dearer than the world and heaven and God and all. If such a child is laid upon a sickbed and sinks quietly and unceasingly into the arms of death, and the father sets everything into motion to fight the unequal battle and to conquer the creeping illness and the threatening death, how will it look in the despairing father’s heart when the final hour comes for the beloved child, and it closes its eyes for the last time, and no earthly love can accomplish anything more for it? It is a picture of the soul that strives for eternal life by its own strength—and fails. It is a bruised reed. But Jesus does not break it; he has healing for the wounded soul, he has salvation and life for the crushed human being. His grace shall save a sinner, and his mercy shall bind up the wounds of the heart.

Friend, if you have entered upon this way of Jesus, then walk upon it! Remain in the valleys of humility and persist in poverty of spirit. And if you then continually feel yourself to be a bruised reed, then know that the Lord has a loving Spirit and hand, and that he takes pleasure in saving where it is impossible for human beings.

Jesus’ way is the same also with his congregation and within his congregation. He does not intend to make it great in the eyes of the world; nor does he wish it to become great in its own eyes. God’s congregation is the people who walk with Jesus and are healed by him. And even if they are many, the Lord nevertheless wills that they should not boast of their great number and think that therein lies their strength. The congregation’s way is only then the right way when the congregation walks in Jesus’ footsteps and does not seek to draw to itself the great unconverted multitude of the world by its visible greatness and power, but much more has the mind of Christ, which has compassion upon the bruised reed and the smoldering wick.

If we are to be the body of Christ, then we must follow our head. “He will not quarrel nor cry out, nor will anyone hear his voice in the streets.” It was not Christ’s way to seek outward exaltation and worldly power. By such means he could indeed force the world to pay him homage as a mighty lord, but he could not compel the salvation of a single soul. What does all the greatness of the world avail, if we lack the Spirit of Christ? But this is the mark of the Spirit of Christ: that we have his inward love, so that we seek what is lost, that it may be saved.

Away therefore with all ambition and vanity in our congregational work! Away with the desire to glitter in the pulpit, away with the desire to be foremost in the congregation! Away with laboring to become a great congregation without the Spirit of Christ! Let us follow Jesus in his inward love’s self-sacrificing work for the raising up of fallen human beings, for the salvation and blessedness of poor sinners.

Jesus’ way is the true way of mission. All the armies and navies of the world together can accomplish nothing for the salvation of the nations. All the merchant fleets of the world and all its means of communication can accomplish nothing to give the nations hope of eternal life. If culture and civilization are never so loud and boastful, over against the darkness and despair of paganism they stand powerless. It is only Christianity that can enter into this black night full of devilry and misery and, with the Spirit of Christ and the light of the gospel, seek out the lost souls in order to show them the way and give them life.

Not by might and not by power, but by my Spirit, says the Lord, shall the work be done.

God’s kingdom in the individual soul, in the building up of the congregation, in the work of mission, is always advanced in the one and same way. The Spirit of the Lord, who without worldly greatness and outward glory heals the wounds of souls, creates life, imparts love, and makes hearts glad in the Lord, is the hidden power that drives the blessed work which from day to day brings us nearer to the goal: the gospel of the kingdom shall be preached to all nations, and then the end shall come.

Do not aspire to the high things, but keep to the lowly; let Christ’s humble and unassuming way, with its healing power for sick and wounded souls, with the world’s proud contempt and bitter hatred, but with God’s blessed promises of eternal joy and glory, be our way, and we shall one day reach the point of seeing the King in his beauty and all God’s children in their blessedness. 

Amen.

