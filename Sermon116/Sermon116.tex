

\section{First Sunday after Easter: The Risen Jesus}


\begin{quote}

Luke 24:36–43: And while they were speaking, Jesus himself stood in the midst of them and said: Peace be with you. But they were terrified and seized with fear and thought they were seeing a spirit. And he said: Why are you troubled, and why do such thoughts arise in your hearts? Behold my hands and my feet, that it is I myself; touch me and see, for a spirit does not have flesh and bones as you see that I have. And when he had said this, he showed them his hands and his feet. And while they yet believed not for joy and marveled, he said: Have you here anything to eat? And they gave him a piece of broiled fish and of a honeycomb. And he took it and ate before their eyes.


\end{quote}

\bigskip


“The Lord is truly risen!” 

This the eleven already knew and confessed when the two disciples returned from Emmaus and told that they had seen the risen one. For the eleven greeted them at their entrance with this cry, which since then has sounded unceasingly within the congregation of God: “The Lord is truly risen and has been seen by Simon.”

Yet our text shows us how difficult it is to believe it exactly as it stands that Jesus is truly risen, bodily risen, as Peter later says on the day of Pentecost: “His soul was not left in the kingdom of the dead, nor did his flesh see corruption.”

For when Jesus himself stood in the midst of the disciples, who had just confessed his resurrection, they were afraid and thought that they saw a spirit. And when Jesus had said: “See my hands and my feet, that it is I myself; touch me and see; for a spirit does not have flesh and bones, as you see that I have,” they still did not believe for joy; so that at last Jesus even took a piece of broiled fish and of a honeycomb and ate it before their eyes.

As easily as it is said: “The Lord is truly risen,” so difficult it is to believe it altogether as the words sound. And yet only this is the true and genuine Christian faith.

In our days, and indeed always since the times of Christ, there have been certain very lofty spirits who have readily believed and taught that Christ has spiritually risen from the dead. By this they mean chiefly that his life and work, his suffering and death, have called forth a movement and stirring in the world which has continued until this day and will doubtless continue until the end of the world. Nor indeed can this be denied.

Others again speak more freely and say: Christ has risen in his disciples, in his congregation; the Jews put him to death, but behold, his congregation, which is his body, lives. Nor can this be denied either.

Yet neither of these is the whole, true, Christian faith. It is far simpler, and therefore far more demanding, but therefore also far more blessed. It believes in Jesus Christ crucified, dead, and risen—that is, bodily risen—so that “his soul was not left in the realm of the dead, nor did his flesh see corruption.”

The Christian soul finds its salvation in Jesus himself. His bodily death upon the cross is the atonement for our sins; his bodily resurrection is the mighty testimony that he is the Son of God and the victor over death. If he has died, he who bore our sins, then our sin is atoned for, then there is forgiveness of sins through faith in his blood; if he is risen, then death is overcome, then there is eternal life for sinners through faith in him.

Therefore this is the simple Christian faith which is mighty to save our souls: that “Christ died for our sins according to the Scriptures, and that he was buried, and that he rose again the third day according to the Scriptures, and that he was seen by Cephas, then by the twelve.”

In this there is salvation for the soul, in nothing else. Not in the understanding’s insight into God’s order of salvation; not in the will’s exertion to improve itself and live a life according to the Lord’s law; not in the feeling’s sorrow and joy. But the living Savior, dead for you and risen for you, the worker of wonders, he himself is the Savior, who can say with the power of Spirit and life to your soul: Peace be with you! For he alone has authority over sin and death, because he has suffered, innocent for the guilty, righteous for the unrighteous, and has come forth from grave and hell with victory over them both. Therefore he can forgive sin; therefore he can save from death.

The only thing that matters is to find him, who has the keys of hell and of death, who opens and no one shuts, and shuts and no one opens. If you find him, then he shall say to you: “Fear not! I am the first and the last and the living one; and I was dead, and behold, I am alive forevermore, Amen.” And his peace shall descend into your heart, because in that same moment you know that he is the one who understands all your distress, since he himself has suffered it, and who is mighty to relieve it, since he has gone forth victorious out of suffering.

But where is he to be found? Yes, where did his disciples find him? The two who went to Emmaus, the eleven who were gathered in Jerusalem—where did they find him? It was he who came to them as they went with weeping in their hearts and spoke of his suffering and death. It was he who came to the eleven as they sat frightened behind closed doors with his words and works in remembrance. He came to them. Therefore do not fear, you who earnestly long for salvation and peace; he will come to you also, if you seek him in his Word and in his congregation. It is as it stands in the hymn:



% 3rd verse of: To Guds Venner vandre silde by Johannes Neunherz
% https://hymnary.org/text/to_guds_venner_vandre_silde#Author

\begin{quote}
Where two in God take counsel,
There he himself the third;
He makes their counsel open,
And speaks the comforting word.\footnote{Johannes Neunherz, \textit{To Guds Venner vandre silde} (3rd verse)}

\end{quote}

Or as it stands in another hymn:

% 1st verse of Se hvor Jesus allevegne by Thomas Kingo 
% https://hymnary.org/text/se_hvor_jesus_allevegne

\begin{quote}
See how Jesus everywhere
Is present with each one
Who will inscribe him in the heart
And hold himself near him!
In the midst of the disciples
He lets himself be gloriously seen,
As they, full of sorrow in the hall,
spoke of him in sweet discourse.\footnote{Thomas Kingo, \textit{Se hvor Jesus allevegne} (1st verse)}
\end{quote}

If you seek him in full earnest, then you shall find that he is not far from any of us. You shall find him, and he shall find you, and your soul shall be refreshed by his peace. For he can give you all that you lack. You are sinful; he gives you righteousness. You are mortal; he grants you eternal life. You are lost and condemned; he gives eternal salvation and the hope of life. You bear a corruptible and sensual body; he gives incorruptibility and the glorification of the body. For all that he has won as the Son of Man in glory for his human nature, that he has won in order to give it to us; for he himself had enough in the glory which he had with the Father before the foundation of the world was laid.

Thus it is only the risen Jesus who can be our Savior. What we need is not merely a teacher who can show us the way through life, but a Savior who can show us the way through death. As it is written: “Our God is a God of continual salvation, and to the Lord God belong deliverances from death.” For “we ourselves also, who have the firstfruits of the Spirit, we ourselves groan within ourselves, waiting for adoption, the redemption of our body.” To full salvation belongs also the salvation of the body in incorruptibility; therefore Jesus’ bodily resurrection is so precious a pearl in the wreath which Christian faith possesses. God gave his image in body and soul; he raises it again in both through his Son.

Therefore the believing congregation of God is a blessed fellowship. For when death breaks the brother‑ and sister‑chain with its strong hand, yet it is only a separation to the eye, not to the heart. For the soul lives, and the body rises by God’s power, which he revealed when he raised Jesus from the dead. Though therefore we stand with tears in our eyes beside bier and grave, yet we sing glad in mind and soul:

% 3rd and 4th verses of "Med sorgen og klagen hold maade" by Aurelius Clemens Prudentius
% https://hymnary.org/text/med_sorgen_og_klagen_hold_maade

\begin{quote}
Let heart and eye then break,
And body laid in coffin lie,
The hour shall not be far away
When he shall call it back to life.

By death from all distress set free,
The bodies mingle with the dust,
But glorious they shall rise again,
With life‑breath and with beauty filled.\footnote{\textit{Med sorgen og klagen hold maade}, 3rd–4th verses; hymn based on Aurelius Clemens Prudentius.}
\end{quote}

The Lord be praise and glory for the resurrection and the hope of resurrection. May we all do as Paul: “Forgetting what lies behind and reaching forth to what lies ahead, I press toward the goal for the prize of the high calling of God in Christ Jesus.”

