

\section{Good Friday: Your Hour and the Power of Darkness}


\begin{quote}

Luke 22:39—71; 23:1—46.

And he went out and went, as was his custom, to the Mount of Olives; and the disciples also followed him. And when he came to the place, he said to them, Pray that you may not fall into temptation. And he withdrew from them about a stone’s throw, and knelt down and prayed, saying, Father, if you are willing, remove this cup from me; nevertheless not my will, but yours be done. And there appeared to him an angel from heaven, strengthening him. And being in agony he prayed more earnestly; and his sweat became like drops of blood falling to the ground.

And when he rose from prayer, he came to the disciples and found them sleeping for sorrow. And he said to them, Why do you sleep? Rise and pray, that you may not fall into temptation.

While he was yet speaking, behold, a multitude came, and he who was called Judas, one of the twelve, went before them and drew near to Jesus to kiss him. But Jesus said to him, Judas, do you betray the Son of Man with a kiss? And when those who were about him saw what would follow, they said, Lord, shall we strike with the sword? And one of them struck the servant of the chief priest and cut off his right ear. But Jesus answered and said, Let it go this far. And he touched his ear and healed him.

Then Jesus said to the chief priests and captains of the temple and elders, who had come against him, Have you come out as against a robber, with swords and clubs? When I was daily with you in the temple, you did not stretch out your hands against me; but this is your hour, and the power of darkness.

And they seized him and led him away and brought him into the high priest’s house; and Peter followed at a distance. And when they had kindled a fire in the midst of the palace and sat down together, Peter sat among them. And a certain servant girl, seeing him as he sat in the light, looked intently at him and said, This man also was with him. But he denied it, saying, Woman, I do not know him. And a little later another saw him and said, You also are one of them. But Peter said, Man, I am not. And about an hour later, another confidently affirmed, saying, Truly this man also was with him, for he too is a Galilean. But Peter said, Man, I do not know what you are saying. And immediately, while he was still speaking, the cock crowed. And the Lord turned and looked upon Peter. And Peter remembered the word of the Lord, how he had said to him, Before the cock crows today, you shall deny me three times. And he went out and wept bitterly.

And the men who held Jesus mocked him and beat him. And they threw a cloth over him and struck him, saying, Prophesy, who is it that struck you? And many other things they spoke against him, blaspheming.

And when it was day, the assembly of the elders of the people gathered together, both chief priests and scribes; and they led him away into their council, saying, If you are the Christ, tell us. But he said to them, If I tell you, you will not believe; and if I ask you, you will not answer. But from now on the Son of Man shall be seated at the right hand of the power of God. And they all said, Are you then the Son of God? And he said to them, You say it, for I am. And they said, What further testimony do we need? For we ourselves have heard it from his own mouth.

And their whole crowd rose up and led him to Pilate. And they began to accuse him, saying: This man we have found misleading the people and forbidding the payment of tribute to Caesar, and saying that he himself is the Christ, a King.

And Pilate asked him, saying: Are you the King of the Jews? And he answered him and said: You say it.

And Pilate said to the chief priests and to the multitude: I find no guilt in this man. But they insisted strongly, saying: He stirs up the people, teaching throughout all Judea, beginning from Galilee even to this place.

But when Pilate heard of Galilee, he asked whether the man was a Galilean. And when he learned that he was under Herod’s authority, he sent him to Herod, who himself also was in Jerusalem in those days.

And when Herod saw Jesus, he rejoiced greatly; for he had long desired to see him, because he had heard much concerning him, and he hoped to see some sign done by him. And he asked him many questions; but he answered him nothing.

And the chief priests and the scribes stood and accused him vehemently. And Herod and his soldiers mocked him, put a white robe on him, and sent him back to Pilate. And on that day Pilate and Herod became friends with one another; for before they had been at enmity.

And Pilate called together the chief priests and the rulers and the people, and said to them: You have brought this man to me as one who is misleading the people; and behold, I, having examined him before you, find no guilt in this man concerning those things of which you accuse him. Nor has Herod either; for I sent you to him, and behold, he has done nothing deserving death. Therefore I will chastise him and release him.

Now he was obliged to release one to them at the feast. But they shouted as one, saying: Away with this man, and release to us Barabbas—who had been cast into prison for an insurrection that had taken place in the city, and for murder.

Again Pilate addressed them, desiring to release Jesus. But they shouted in reply, saying: Crucify, crucify him!

And he said to them the third time: Why, what evil has this man done? I find no guilt deserving death in him; therefore I will chastise him and release him. But they pressed upon him with loud cries, demanding that he should be crucified; and their shouting prevailed.

And Pilate pronounced that their demand should be granted. And he released the one they were asking for, who had been cast into prison for insurrection and murder; but Jesus he handed over to their will.

And as they led him away, they seized one Simon of Cyrene, coming from the country, and laid the cross upon him, that he might bear it after Jesus.

And there followed him a great crowd of the people, including women who were mourning and weeping for him. But Jesus, turning to them, said: Daughters of Jerusalem, do not weep over me, but weep over yourselves and over your children. For behold, days are coming in which they shall say: Blessed are the barren, and the wombs that did not bear, and the breasts that did not nurse. Then they shall begin to say to the mountains: Fall upon us; and to the hills: Cover us. For if they do these things in the green tree, what shall be done in the dry?

Two other evildoers were also led out with him to be put to death. And when they had come to the place called the Skull, there they crucified him, and the evildoers, one on the right and the other on the left.

And Jesus said: Father, forgive them; for they do not know what they are doing. And they divided his garments and cast lots.

And the people stood looking on; and the rulers also mocked him, saying: He saved others; let him save himself, if he is the Christ, the Chosen of God.

And the soldiers also mocked him, coming up and offering him sour wine, and saying: If you are the King of the Jews, save yourself.

And there was also an inscription over him, written in Greek and Latin and Hebrew: This is the King of the Jews.

And one of the evildoers who were hanging there railed at him, saying: Are you the Christ? Then save yourself and us. But the other answered and rebuked him, saying: Do you not even fear God, since you are under the same sentence? And we indeed justly; for we receive what our deeds deserve; but this man has done nothing amiss.

And he said: Jesus, remember me when you come in your kingdom. And he said to him: Truly I say to you, today you shall be with me in Paradise.

And it was now about the sixth hour, and there was darkness over the whole land until the ninth hour, and the sun was darkened and the veil of the temple was torn in two.

And Jesus, crying with a loud voice, said: Father, into your hands I commit my spirit. And having said this, he gave up his spirit.

\end{quote}

\bigskip


This is the week of silence. And Good Friday is for us the quietest of all these days of sorrow. In Norway it was the custom that on that day neither organ nor bells were used. But for Jesus it was no quiet day. It was long—and loud.

First a bitter struggle in his soul, so that the sweat fell like drops of blood, while his disciples could not watch one hour with him; at last, since they might not use weapons, they fled from him.

Then an onslaught against him by the chief priests, the captains, the elders, and a band of servants with swords and clubs and torches—as against a robber. Then before Annas, then to Caiaphas with mockery and scorn and blows and false witnesses in both places; later before Pilate and before Herod and back again to Pilate, while the chief priests and the elders and the whole people cried out as one: Crucify! Crucify! Give us Barabbas! His blood be upon us and upon our children! If you release him, you are not Caesar’s friend! Crucify! Crucify! Until even the proud Roman was overcome by the noise and frightened by the fury of the people, so that he handed Jesus over to be crucified, and the cries of the people and of the chief priests prevailed.

The Lord himself gathers all this tumult of wickedness and injustice into one word: "This is your hour and the power of darkness."

So the servants of darkness and of Satan also have their time and their hour; and silent as a lamb that is led to the slaughter, Jesus humbled himself under this hour and power of darkness, though he could have commanded legions of angels for his defense. He did so that he might suffer it all—wholly innocent—and thus become the Lamb who bears and takes away the sin of the world.

There is something wondrous and beyond us in this providence of the righteous and almighty God to permit the powers of wickedness such injustice and, as it were, to appoint for them an hour and a fixed time in which they have complete dominion, so that they could trample under their feet and cast down into the depths of shame this holy and innocent Son of God. Yet it shows us also that, whether we understand it or not, it was necessary — yes, it was pleasing to you, Father — in order that we might see that God means it in dead earnest when he says that he loved the world so greatly that he gave his only-begotten Son, that we should be saved when we believe.

How often does not the question arise in our hearts in the course of daily life, when we see great injustices committed without punishment: Why does God permit this? — so that we are often shaken by it and tempted to doubt God’s almighty and righteous love.

And yet, what are even the most dreadful injustices either in history or in our own personal experience compared with this people’s rebellious conduct against their own Savior and Messiah? And if God did not spare his own Son, but gave to the prince of darkness and his servants, the elders and chief priests, an hour and a time which they could use freely to satisfy their heart’s wickedness and bloodthirst, and he did it with a purpose that lay far, far beyond any human understanding to grasp or devise — that, as we see on this day, innumerable human souls should be eternally saved — how ought we not to bow humbly in the dust before the eternal and all-seeing One, when he still today in the midst of us permits things that are offensive to our sense of justice and simple faith, and when this holy and grave word again sounds over us from the Savior’s mouth:

"This is your hour and the power of darkness."

For it serves us for good, even if we do not understand it, as it then became our eternal salvation. It was then also in order that the power of darkness and its servants should have no excuse. Had they not had their hour? Had not God given them leave to use the whole might of their wickedness? And what was the fruit thereof? Jesus raised from the grave; Satan and all his kingdom overcome. And what had they to say when the Lord’s hour came, according to the word of Jesus: "Weep not for me, but for your children; for the days shall come when you shall say to the mountains, Fall on us, and to the hills, Cover us!"

As surely as God gives to the power of darkness its hour, so has he also appointed a day on which he will judge the world in righteousness; and he had already appointed an hour on which this people’s children, who had cried "crucify," should see their sanctuary laid in ruins, and the holy city where they had shed innocent blood leveled to the ground, and they themselves scattered as slaves and strangers over the whole earth unto this day.

Therefore the Savior can on this day of torment and anguish say so calmly: "Let them go so far; this is your hour and the power of darkness." He knew what significance it already had and what his Father would accomplish through him:
"Shall I not drink the cup which my Father has given me?"

And now, my friend, you who have been brought over from darkness to light, from the power of Satan to God, now the same Good Friday question comes to you which Jesus put to James and John:

"Are you able to drink the cup that I must drink?"

For you must indeed drink it, according to the word that no one can be glorified with Jesus who does not suffer with him. When you have labored yourself weary and spent your strength in vain and to no purpose; when all your love and devotion have been despised and misinterpreted; when it seems as though your whole building shall fall into ruins, and enemies trample upon you and would blot even your name from remembrance in violence and shame — can you then still say with the prophet:

"Yet my right is with the Lord, and my recompense with my God," and add with confidence:

"This is your hour and the power of darkness," because you know that your shame shall be God’s glorification, and your humiliation the salvation of many souls — then you have today had your Good Friday with Jesus and shall in the Lord’s time be raised and exalted with him.

Then you have today beheld him, the author and finisher of our faith, who for the joy that was set before him endured the cross and despised the shame; and you shall not grow weary nor faint in your soul, but shall sit with him at the right hand of God’s throne, and, when the hour of darkness is past, inherit the unfading crown of glory. For yet a very little while, and he who is to come will come and will not delay.

Be therefore not of those who, because of the power of darkness and the hour of wickedness, draw back to their own destruction, but of those who believe in the midst of Good Friday’s anguish and darkness unto the salvation of the soul.

For truly, he will not delay either with salvation or with judgment. And as the day came, dreadful with fire and destruction upon the city and the temple, so that not one stone was left upon another, so shall it surely come in the Lord’s appointed time to strike down and cast into eternal darkness those who now rejoice in the hour of the power of darkness and again trample the Son of God’s blood under foot, persecuting and misleading his little ones, mocking the Spirit of grace, or sinking into lukewarmness and being spewed out of the Lord’s mouth. For recompense belongs to God; I will repay, says the Lord.

And therefore the Good Friday voice sounds as a holy admonition also to the impenitent:
"This is your hour and the power of darkness."

Friend, shall it sound in vain?

