

\section{Good Friday: Your Hour and the Power of Darkness}


\begin{quote}

Luke 22:39—71; 23:1—46.

And he went out and went, as was his custom, to the Mount of Olives; and his disciples also followed him. And when he came to the place, he said to them: Pray that you may not enter into temptation. And he withdrew from them about a stone’s throw, and fell upon his knees and prayed, saying: Father, if you are willing, remove this cup from me — yet not my will, but yours be done. And there appeared to him an angel from heaven, strengthening him. And being in agony, he prayed more earnestly; and his sweat became like drops of blood falling down upon the ground. And when he rose from prayer and came to his disciples, he found them sleeping for sorrow.

And they led him away to the high priest’s house. But Peter followed at a distance. And when they had kindled a fire in the midst of the courtyard and sat down together, Peter sat among them. But a certain servant girl, seeing him as he sat in the light and looking intently at him, said: This man also was with him. But he denied him, saying: Woman, I do not know him. And a little later another saw him and said: You also are one of them. But Peter said: Man, I am not. And about the space of one hour after, another confidently affirmed, saying: Of a truth this man also was with him; for he is a Galilean. But Peter said: Man, I do not know what you are saying. And immediately, while he yet spoke, the rooster crowed. And the Lord turned and looked upon Peter. And Peter remembered the word of the Lord, how he had said to him: Before the rooster crows today, you shall deny me three times. And Peter went out and wept bitterly.

And the men who held Jesus mocked him and struck him. And when they had blindfolded him, they struck him on the face and asked him, saying: Prophesy, who is it that struck you? And many other things they spoke against him, reviling him.

And as soon as it was day, the elders of the people and the chief priests and the scribes came together and led him into their council, saying: If you are the Christ, tell us. But he said to them: If I tell you, you will not believe; and if I also ask you, you will not answer me nor let me go. Hereafter shall the Son of Man sit on the right hand of the power of God. Then they all said: Are you then the Son of God? And he said to them: You say that I am. And they said: What further need have we of witness? For we ourselves have heard it from his own mouth.

And the whole multitude of them arose and led him to Pilate. And they began to accuse him, saying: We found this man perverting the nation and forbidding to give tribute to Caesar, saying that he himself is Christ, a king. And Pilate asked him, saying: Are you the King of the Jews? And he answered him and said: You say it. Then Pilate said to the chief priests and to the people: I find no fault in this man. But they were the more urgent, saying: He stirs up the people, teaching throughout all Judea, beginning from Galilee to this place.

When Pilate heard of Galilee, he asked whether the man were a Galilean. And as soon as he knew that he belonged to Herod’s jurisdiction, he sent him to Herod, who himself also was at Jerusalem in those days. And when Herod saw Jesus, he was exceedingly glad; for he had desired for a long time to see him, because he had heard many things of him, and he hoped to see some miracle done by him. Then he questioned him with many words; but he answered him nothing. And the chief priests and scribes stood and vehemently accused him. And Herod with his soldiers treated him with contempt and mocked him, and arrayed him in a splendid robe and sent him again to Pilate. And the same day Pilate and Herod were made friends together; for before they had been at enmity between themselves.

And Pilate, when he had called together the chief priests and the rulers and the people, said to them: You have brought this man to me as one who perverts the people; and behold, I, having examined him before you, have found no fault in this man concerning those things whereof you accuse him; no, nor yet Herod; for I sent you to him, and behold, nothing worthy of death has been done by him. I will therefore chastise him and release him. For he must release one to them at the feast. And they cried out all together, saying: Away with this man, and release unto us Barabbas — who for a certain insurrection made in the city and for murder was cast into prison.

Pilate therefore, willing to release Jesus, spoke again to them. But they cried out, saying: Crucify, crucify him. And he said to them the third time: Why, what evil has he done? I have found no cause of death in him; I will therefore chastise him and let him go. And they were urgent with loud voices, requiring that he might be crucified. And the voices of them and of the chief priests prevailed. And Pilate gave sentence that it should be as they required. And he released unto them him who for insurrection and murder had been cast into prison, whom they desired; but Jesus he delivered to their will.

And as they led him away, they laid hold upon one Simon, a Cyrenian, coming out of the country, and on him they laid the cross, that he might bear it after Jesus. And there followed him a great company of people, and of women who also bewailed and lamented him. But Jesus, turning unto them, said: Daughters of Jerusalem, weep not for me, but weep for yourselves and for your children. For behold, the days are coming in which they shall say: Blessed are the barren, and the wombs that never bore, and the breasts that never nursed. Then shall they begin to say to the mountains: Fall on us; and to the hills: Cover us. For if they do these things in the green tree, what shall be done in the dry?

And there were also two others, criminals, led with him to be put to death. And when they were come to the place which is called The Skull, there they crucified him, and the criminals, one on the right hand and the other on the left. Then Jesus said: Father, forgive them; for they do not know what they do. And they divided his garments and cast lots.

And the people stood looking on. And the rulers also with them derided him, saying: He saved others; let him save himself, if he is the Christ, the chosen of God. And the soldiers also mocked him, coming to him and offering him vinegar, and saying: If you are the King of the Jews, save yourself. And there was also a superscription written over him in letters of Greek and Latin and Hebrew: THIS IS THE KING OF THE JEWS.

And one of the criminals who were hanged railed on him, saying: If you are the Christ, save yourself and us. But the other answered and rebuked him, saying: Do you not even fear God, seeing you are under the same condemnation? And we indeed justly; for we receive the due reward of our deeds; but this man has done nothing amiss. And he said to Jesus: Lord, remember me when you come into your kingdom. And Jesus said to him: Truly I say to you: Today you shall be with me in Paradise.

And it was about the sixth hour, and there was a darkness over all the earth until the ninth hour. And the sun was darkened, and the veil of the temple was torn in the midst. And when Jesus had cried with a loud voice, he said: Father, into your hands I commit my spirit. And having said thus, he gave up his spirit.

\end{quote}

\bigskip





This is the Holy Week. And Good Friday is for us the quietest of all these days of sorrow. In Norway it was the custom that on that day neither organ nor bells were used. But for Jesus it was no quiet day. It was both long and tumultuous.

First an irreconcilable struggle of soul, so that the sweat ran like drops of blood, while his disciples could not watch one hour with him; at last, since they might not use carnal weapons, they fled from him.

Then an assault against him by the chief priests, the captains, the elders, and a band of servants with swords and staves and lamps as against a robber. Then before Annas, then to Caiaphas with mockery and scorn and blows and false witnesses in both places; later before Pilate and before Herod and back again to Pilate, while the chief priests and the elders and the whole people all together cried and shouted with tumult and roar: Crucify! Crucify! Give us Barabbas! His blood be upon us and upon our children! If you release him, you are not Caesar’s friend! Crucify! Crucify! Until even the proud Roman was overcome by the noise and frightened by the fury of the people, so that he delivered Jesus to be crucified, and the cries of the people and of the chief priests prevailed.

The Lord himself gathers all this tumult of wickedness and injustice into one word: "This is your hour and the power of darkness."

Thus, then, the servants of darkness and of Satan also have their time and their hour; and silent as a lamb that is led to the slaughter, Jesus bowed himself humbly under this hour and power of darkness, though he could have commanded legions of angels for his defense; it was in order to suffer all and suffer wholly innocent, and thus to become the Lamb who bears and takes away the sin of the world.

There is something wondrous and for us incomprehensible in this ordering of the righteous and almighty God to permit the powers of wickedness so great an injustice and, as it were, to appoint for them an hour and a fixed time in which they have complete dominion, so that they could trample under their feet and cast down into the depths of shame this holy and innocent Son of God. Yet it shows us also that, whether we understand it or not, it was necessary — yes, it was pleasing to you, Father — in order that we might see that God is altogether in earnest when he says that he loved the world so greatly that he gave his only-begotten Son, that we should be saved when we believe.

How often does not the question arise in our hearts in the course of daily life, when we see great injustices committed without punishment: Why does God permit this? — so that we are often assailed by it and tempted to doubt God’s almighty and righteous love.

And yet, what are even the most dreadful injustices either in history or in our own personal experience compared with this rebellious conduct of the Jewish people against their own Savior and Messiah? And if God did not spare his own Son, but gave to the prince of darkness and his servants, the elders and chief priests, an hour and a time which they might undisturbed use to satisfy their heart’s wickedness and bloodthirst, and he did it with a purpose that lay far, far beyond any human understanding to grasp or devise — that, as we see on this day, innumerable human souls should be eternally saved — how ought we not to bow humbly in the dust before the eternal and all-seeing One, when he still today in the midst of us permits things that are offensive to our sense of justice and simple faith, and when this holy and grave word again sounds over us from the Savior’s mouth:
"This is your hour and the power of darkness."

For it serves us for good, even if we do not understand it, as it then became our eternal salvation. It was then also in order that the power of darkness and its servants should have no excuse. Had they not had their hour? Had not God given them leave to use the whole might of their wickedness? And what was the fruit thereof? Jesus raised from the grave; Satan and all his kingdom overcome. And what had they to say when the Lord’s hour came, according to the word of Jesus: "Weep not for me, but for your children; for the days shall come when you shall say to the mountains, Fall on us, and to the hills, Cover us!"

As surely as God gives to the power of darkness its hour, so has he also appointed a day on which he will judge the world in righteousness; and he had already appointed an hour on which this people’s children, who had cried "crucify," should see their sanctuary laid in ruins, and the holy city where they had shed innocent blood leveled to the ground, and they themselves scattered as slaves and strangers over the whole earth unto this day.

Therefore the Savior can on this day of torment and anguish say so calmly: "Let them go so far; this is your hour and the power of darkness." He knew what significance it already had and what his Father would accomplish through him:
"Shall I not drink the cup which my Father has given me?"

And now, my friend, you who have been translated from darkness to light, from the power of Satan to God, now the same Good Friday question comes to you which Jesus put to James and John:
"Are you able to drink the cup that I must drink?"
For you must indeed drink it, according to the word that no one can be glorified with Jesus who does not suffer with him. When you have labored yourself weary and spent your strength in vain and to no purpose; when all your love and devotion have been despised and misinterpreted; when it seems as though your whole building shall fall into ruins, and enemies trample upon you and would blot even your name from remembrance in violence and shame — can you then still say with the prophet:
"Yet my right is with the Lord, and my recompense with my God," and add with confidence:
"This is your hour and the power of darkness," because you know that your shame shall be God’s glorification, and your humiliation the salvation of many souls — then you have today had your Good Friday with Jesus and shall in the Lord’s time be raised and exalted with him.

Then you have today beheld him, the author and finisher of our faith, who for the joy that was set before him endured the cross and despised the shame; and you shall not grow weary nor faint in your soul, but shall sit with him at the right hand of God’s throne, and, when the hour of darkness is past, inherit the unfading crown of glory. For yet a very little while, and he who is to come will come and will not delay.

Be therefore not of those who, because of the power of darkness and the hour of wickedness, draw back to their own destruction, but of those who believe in the midst of Good Friday’s anguish and darkness unto the salvation of the soul.

For truly, he will not delay either with salvation or with judgment. And as the day came, dreadful with fire and destruction upon the city and the temple, so that not one stone was left upon another, so shall it surely come in the Lord’s appointed time to strike down and cast into eternal darkness those who now rejoice in the hour of the power of darkness and again trample the Son of God’s blood under foot, persecuting and misleading his little ones, mocking the Spirit of grace, or sinking down into lukewarmness and being spewed out of the Lord’s mouth. For recompense belongs to God; I will repay, says the Lord.

And therefore the Good Friday voice sounds as a holy admonition also to the impenitent:
"This is your hour and the power of darkness."

Friend, shall it sound in vain?

