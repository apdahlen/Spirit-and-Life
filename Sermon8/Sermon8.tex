

\section{Third Sunday of Advent: The Way of the Lord}


\begin{quote}

Luke 3:1–6. In the fifteenth year of the reign of Emperor Tiberius, when Pontius Pilate was governor of Judea, and Herod was tetrarch of Galilee, and his brother Philip tetrarch of the land of Iturea and Trachonitis, and Lysanias tetrarch of Abilene, while Annas and Caiaphas were high priests, the word of God came to John, the son of Zechariah, in the wilderness. And he went throughout the whole region around the Jordan, proclaiming a baptism of repentance for the forgiveness of sins, as it is written in the book of the words of the prophet Isaiah: “The voice of one crying in the wilderness: Prepare the way of the Lord, make his paths straight. Every valley shall be filled, and every mountain and hill shall be brought low, and the crooked shall become straight, and the rough ways shall become smooth, and all flesh shall see the salvation of God.”

\end{quote}

\bigskip


When the Lord comes to his people, the question is whether his way is prepared for him, or whether the people are prepared, so that they are a people made ready for the Lord.

This preparation of the way and of the people was so necessary that the prophet Isaiah had already proclaimed that a crying voice should sound in the wilderness before the coming of the Lord; and it was foretold by the prophet Malachi that the messenger of the Lord should come before him, to prepare the way before his face; and again, that Elijah should come before the day of the Lord’s coming, so that the day might not become a curse instead of salvation.

And of John it was said, before he was born, that he would be this forerunner, this crying voice, this preparer of the people in the spirit and power of Elijah.

And in our text it is told that John himself, in the wilderness, received the word of the Lord. We do not know how the Lord spoke to John; nor is that of any importance; but we know both that it was the Lord himself who called him, and we know that it was in order to prepare the way of the Lord, and so that the coming Messiah might be revealed to Israel, that he was called.

And John walked the hard path of the divine calling, which led him to death. He went to the secure and defiant Israel and preached a baptism of repentance for the forgiveness of sins. In this way he was to prepare the way of the Lord; in this way he was to make people ready for the Lord.

But did Israel then need preparation? This chosen people of the Lord, with the Law and the Prophets, with the light of the Word and the call of the Spirit—did it need preparation for the coming of the Lord? They were indeed “children of Abraham” and “children of the promise”; the kingdom of God had been promised to them, the Messiah had been promised to them; and now they had waited so long and suffered so much—should they still need preparation? Were they not ready to receive the Lord, who would bring them vengeance upon their enemies and restore to them the glory of David? Would they not gladly see the day of the Lord and rejoice to meet their deliverer?

Alas, alas, Israel was not ready to meet its God; Israel was not prepared to receive the kingdom of heaven. Their mind was earthly and fleshly; with anxious expectation they looked for improvement in their earthly conditions, for earthly glory, for worldly power and freedom. But the kingdom of heaven was not the longing and desire of the heart. Therefore the messenger of the Lord had to go before his face and cry: “Repent, for the kingdom of heaven is near!” Turn your hearts from the world to God, from the earth to heaven; for it is not the kingdom of David, but the kingdom of God that is coming; not an earthly kingdom, but the heavenly kingdom that is coming. It is not “all the kingdoms of the world and their glory” that are to be given to fleshly hearts for sensual enjoyment; but it is the kingdom of heaven that comes to broken hearts with heavenly peace and healing. It is a kingdom that no one can see unless he is born again of water and the Spirit.

And John, who knew his people’s insatiable desire for the glory of the world, and who knew that the Lord does not give his people a sensual and perishable happiness, but an eternal and imperishable glory, stepped forward with the preaching that the Lord himself had laid in his heart in the solitude of the wilderness. He came to his people and preached a baptism of repentance for the forgiveness of sins.

Thus the way of the Lord was to be prepared; thus the valleys were to be filled and the hills leveled; thus the crooked was to become straight and the rough to become smooth.

\textbf{Is the same still needed?}

\textbf{Yes, precisely the same is still needed for the whole people and for every individual soul.}

Why do the people struggle against one another in our land? Why is the bondage so great and the zeal so burning and the struggle so hard and the envy so bitter? Is it for a heavenly glory and for a heavenly crown that they fight and run and rush so eagerly? Or is it not far more earthly advantage and happiness and power that are pursued with such insatiable desire? Stop for a moment and consider! Look at the teeming crowd around you and ask what all these are seeking. But ask above all after the desire and longing of your own heart. Is it not earthly, is it not sensual?

What voice is it that presses among us? It is the voice of John, which should sound over city and countryside, into house and home, into heart and soul: “Repent, for the kingdom of heaven is near!” It is repentance that is needed among us. It is a baptism of repentance that we require, in order to escape the judgment of the Lord, in order to become a people prepared for the Lord.

The Lord is still coming to us; are our hearts ready to receive him? We speak of building a free church of free congregations; we speak of a renewal of God’s church among us; and this is indeed a coming of the Lord, a day of favor, one of the days of the Son of Man for us. Are we also a people ready for the Lord, so that he can use us in his service? Oh, it is time for the voice of John to sound among us: “Prepare the way of the Lord, make his paths straight!” It is time to go to the river Jordan and confess our sins and receive a baptism of repentance for the forgiveness of sins. Will you not do it now, while it is the time of visitation?

We speak of unity and of peace in God’s church; we rejoice with joy toward the good days that are to dawn. Let us take care that we do not err. The good days are the days of the Lord, and they do not become our days unless we open to him who stands at the door and knocks, and say to him: “You blessed of God, come in!”

We speak of Christmas and Christmas joy, of gladness and peace, of angel song and heavenly jubilation. Then it is time to turn the heart from the world to God, from the earth to heaven. Then it is time for sorrow and brokenness and humility before the face of the Lord; then it is time to “repent and turn again, that our sins may be blotted out, and that times of refreshing may come from the presence of the Lord.”

Then we shall see the salvation of God; then our hearts shall rejoice with imperishable joy.

Repent, for the kingdom of heaven is near!

