% Finally feels like the emotional volatility is correct. More hammer less explanation.

\section{Trinity Sunday: Jesus Has the Authority}

\begin{quote}

Matthew 28:16–20. But the eleven disciples went to Galilee, to the mountain where Jesus had appointed them. And when they saw him, they worshiped him; but some doubted. And Jesus said to them: All authority has been given to me in heaven and on earth. Go therefore and make disciples of all nations, baptizing them in the Name of the Father and of the Son and of the Holy Spirit, and teaching them to keep all that I have commanded you. And behold, I am with you all days to the end of the age. Amen.

\end{quote}

\bigskip

Strange indeed! The disciples did what Jesus had commanded them; they went to Galilee to meet him there after his resurrection; and behold, while the others worshiped him, there were still some of the eleven who doubted.

So it is written of the Lord’s own disciples in this grave and holy hour: some doubted!

How narrow and hard this road must be—to believe that the Crucified is risen! What power Satan and the flesh must wield over us frail children of men, when even in such a glorious hour of revelation doubt could steal into some hearts!

How many among us have not said or thought: Yes, if only I could see Jesus alive before me, then I would surely believe! — and perhaps added: Had I been in the place of the disciples, I would not have doubted.

Friend, be not proud, but fear!

When Mary Magdalene came to the eleven and declared to them that Jesus lived and had been seen by her, they did not believe it (Mark 16:11); and when the two from Emmaus also came and told the same, neither did they believe (Mark 16:13), so that Jesus, when he revealed himself to them as they sat at table, reproached them for their unbelief and hardness of heart. And when Thomas would not believe unless he might see and touch the Savior, he was crushed when the Lord showed himself to him and reproved his unbelief; My Lord and my God! — that was all he could say. But the Lord says: Because you have seen me, you have believed, Thomas; blessed are they who have not seen and yet have believed.

You who would not have doubted had you been in the disciples’ place — bow and see who stands before you! It is the same who stood before the disciples and said:

All authority is given to me in heaven and on earth.

All authority!

Who can understand this? And who would not collapse into the dust when he has grasped it and remembered his sin!

Scripture says: These shall make war with the Lamb, and the Lamb shall overcome them, for he is Lord of lords and King of kings; and

The Father raised Christ from the dead and set him at his right hand in heaven, far above all principality and authority and power and dominion and every name that is named, not only in this world but also in that which is to come, and put all things under his feet; and

The Father has also given the Son authority to execute judgment, because he is the Son of Man.

This is the Lord who stands before you today and stood before the disciples then and spoke his Word of authority.

When the disciples remembered their doubt and unbelief, how must they have felt themselves shaken with fear and awe, like Moses before the bush and Elijah when the still small voice passed by his cave and God himself spoke to him.

But how must they also have felt themselves exalted and blessed when they remembered that this almighty God and Lord was their Savior, their Friend, their Brother! He did not use his authority to judge them—though he had every right—but to crush their doubt and to lift them up in grace and trust them with a task holy beyond measure: to be Christ’s messengers and warriors, to overcome kings and nations and to subject to him the whole world — not with human might nor with the sword, but with the power of the Holy Spirit, with Christ’s cross, with the power of the Word and with the quiet message of peace of the gospel.

Because he has all authority, Jesus gives his disciples this command and this commission:

\begin{quote}
Go therefore and make disciples of all nations by baptism into the Name of the Father and of the Son and of the Holy Spirit, and teaching them to keep all that I have commanded you.
\end{quote}

Baptism and the Word!

Baptism, according to the Word: He who believes and is baptized shall be saved; and

The Word, the two‑edged sword which reveals the secret counsels of hearts — the incorruptible seed of God which begets anew and renews lost children of men; the Word which is Christ himself near my heart and my mouth, the Word of faith which we preach.

These are the two means the Father and the Son, through the Holy Spirit, have entrusted to his disciples in order thereby to subject to Jesus all the peoples of the world through repentance and the forgiveness of sins.

They look small. Powerless. Just as unimpressive as Jesus himself appeared when he walked among the Jewish people, and as he still stands today exposed to mockery and contempt.

From that hour on, there was no more doubt among the disciples; quietly they gathered in Jerusalem; trustfully they chose the twelfth apostle, ready when the hour should come for God’s power to be revealed in them; calmly and expectantly they awaited the feast of Pentecost; and when the Spirit had filled them as with a consuming fire, they boldly struck the first blow with the Word among the peoples and won a glorious victory: Three thousand souls who received the Word were baptized and were added to the congregation.

And thus the Lord’s disciples have gone forward in the power of the Holy Spirit, from century to century, from people to people with baptism and the Word, and have laid the nations under the kingdom of Jesus, until their beautiful feet that proclaimed peace and good tidings have also reached down to our time and to our hearts.

What is it that can give God’s servants, the Lord’s disciples with David’s simple weapons, courage and power to go against the whole world and subdue it? What is it that gives God’s children boldness when they are mocked, when they are trampled underfoot, when they are persecuted and slain, so that even when they die they sing and conquer? It is this Word from the mouth of Jesus that he not only has all authority in heaven and on earth, but that with this authority he will be with his disciples all days unto the end of the world.

If God is for us, who can be against us?

That is the secret. The hidden and yet before the world so gloriously manifested power of Christ’s Church and of his disciples; this is the Christian’s hidden strength, which gives joy in sorrow, victory in defeat, and life in death: that the almighty Son of God by his Holy Spirit dwells in my heart and gives to me, poor frail child of sin, through faith his authority and his protection.

Who, then, could harm me?

% FIXME: Clarification on Baal and blood or fire and blood.

For nearly two thousand years the Lord’s disciples have gone forward to fulfill this entrusted commission to all peoples; mockery and persecution have often been their lot; fire and blood have often marked their path; where Satan has been compelled to lie under Christ’s cross, he has again and again pressed in as an angel of light into the congregation itself to rob the cross of its offense and the gospel of its foolishness and simplicity. Craft and power, together with princes and leaders of the world, he employed to make void this Word of God: All authority is given to me in heaven and on earth, and behold, I am with you all days unto the end of the world.

But to no avail; the blood of witnesses has become the Church’s seed; and where the world slew one disciple and seduced ten, there the Lord has raised up hundreds in their place; nation after nation has been subjected to the gospel; from north to south, toward east and west the Word has pressed forward, until today it is proclaimed in nearly three hundred different languages. Instead of weakening and being exhausted with the course of centuries, the power of Christ’s word has become mightier, so that in our century there has even occurred as it were a new Pentecostal wonder, and the gospel is now proclaimed with greater zeal and in wider extent than ever since the days of the apostles.

But you who doubt Christ’s divine authority, what do you say? Has not his word, both concerning his authority and his preservation, been gloriously confirmed? Be not proud, but fear! For to him is also given authority to judge, and soon, soon he comes!

And you who call yourself by Christ’s Name and belong to his congregation, what do you say? Is it not time for you to awake out of sleep, to read the signs of the times and hasten to take part in the struggle for the spread of the kingdom to the ends of the earth before it is too late and the Lord has no other use for you than for the salt that has lost its savor?

And you servant of God, you are often downcast and fainthearted and it seems that the Word bears no fruit when you preach; lift up your eyes and see what is taking place round about; the end of all things draws near, and the Lord confirms his word gloriously; therefore strengthen the hanging hands and the feeble knees, go with new courage and new strength into the struggle; the victory is near, and when you are weak, then you are strong; for the Lord still has all authority, he sends you, and he is still with you.

And you quiet, unnoticed child of God, man or woman, who in secret have life in God, but often grow discouraged and begin to doubt whether you may belong to the Lord, so sinful and frail you are. 

Oh, how the Lord needs you! 

His power is perfected precisely in weakness. Shake off all chains of sloth and come courageously into the struggle; there is a soul near you whom the Lord will win by you. For remember that if you shall have the blessing and salvation of this promise: Behold, I am with you all days — then you must not forget that the almighty Son of God also commands you: Go out and make all peoples my disciples!

Come, friends—all of you. Work. Pray. Bear witness. Give, that the gospel may press on through heathen, Jews, and slothful Christians unto the ends of the earth, and that we may soon celebrate the great feast of the Trinity in heaven and together with the innumerable multitudes of every people and tongue, with white robes and palms, sing the new song of God and of the Lamb:

Glory be to God the Father, God the Son, and God the Holy Spirit, who has been and always remains one true God from eternity to eternity. Amen.

