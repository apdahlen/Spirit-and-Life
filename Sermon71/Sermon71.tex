

\section{Quinquagesima Sunday: God’s Lamb — God’s Son}


\begin{quote}

John 1:29–34: The following day John saw Jesus coming toward him and said: Behold the Lamb of God, who bears the sin of the world. This is he of whom I said: After me comes a man who has been before me; for he was before me. And I did not know him; but in order that he might be revealed to Israel, for this reason I came baptizing with water. And John bore witness and said: I saw the Spirit descend as a dove from heaven, and it remained upon him. And I did not know him; but he who sent me to baptize with water said to me: He upon whom you see the Spirit descend and remain, he it is who baptizes with the Holy Spirit. And I have seen it and borne witness that this is the Son of God.
\end{quote}

\bigskip

God’s Lamb is God’s Son, and God’s Son is God’s Lamb, who bears the sin of the world.

Here, in a single sum, is the whole wondrous truth of the Gospel: a foolishness and an offense to the world, but a power of God unto salvation for those who believe.

For this is the preaching of the cross.

If Jesus is God’s Lamb, who bears the sin of the world, then the sin of the world is too heavy for the world itself to bear; it is a burden that drags the world down into the depths of perdition, as the stone sinks into the sea.

But if Jesus is God’s Lamb, who bears the sin of the world, then the heavy burden has been laid upon shoulders strong enough to carry it. For God’s Lamb is God’s Son, and what casts the world into the abyss of perdition, the Son of God can bear without perishing.

This is the mightiest testimony to the sin of the world: that the Son of God had to become the Lamb of God in order to bear it.

This is also the strongest testimony to God’s eternal love: that the Son of God took the sin of the world upon himself and was slain as the sacrificial Lamb in order to atone for it.

And if it is an offense to the world to hear that it is sinful, and an even greater offense to hear that its sin has been atoned for by the blood of the Lamb, yet it is blessed and sweet for the one who knows his sin, that the Son of God has become the Lamb of God and has borne it.

Or is there a heavier burden than sin? Of all the cries of pain that sound from this poor earth, where tears moisten the eye and pain pierces the soul, this is the most grievous cry: “My sin, my sin, my sin!”

O the one who has writhed in the distress of sin, who has seen God’s wrath over himself, and who has felt it in the innermost depths of the soul, that this is an incurable ruin, a torment that will endure through the eternity of eternities — for him all other pain has become small and all other sorrow light.

No heavier burden exists than the burden of sin; no more bitter pain exists than the hellish torment of an evil conscience.

Have you ever known it, soul? Or have you until now been so afraid to perceive it that you have fled and fled from the Word of God and the Spirit of God, lest your sin should come to light before you?

Poor human being, who flees from the Gospel of the cross because it reveals your sins to you: you cannot flee from death and judgment; and your sin will overtake you and drag you down into eternal perdition, when there is no longer any way to find reconciliation by the blood of the cross.

Though it is dreadful as death to acknowledge one’s sin, it is nevertheless better to know one’s disease and seek healing for it than to carry it until it is too late.

Come out from your deceitful hiding place and confess that you are a sinner. Come out of your frivolity and your cowardice and place yourself beneath the cross of Jesus and see what your sin has done to the Son of God. Behold the Lamb who was slain. \textbf{Behold the blood that flowed for you!}

When the Spirit of the Lord convicts you of sin, there will arise a moment of joy in the night of your pain, when the Spirit glorifies Jesus in your heart, and you begin to glimpse with the awakening eye of faith that the Son of God has become the Lamb of God for you.

Atonement, reconciliation, forgiveness in the blood of Jesus for all my sin — oh how blessed for the one who is crushed by sin and wrath. “He has borne our sicknesses and carried our pains; the punishment lay upon him, that we might have peace, and by his wounds we have received healing.”

For this purpose, then, the Son of God has come into the world: that he should bear the sins of the world. For this purpose he is baptized by John: that he who knew no sin should be made sin for us, that in him we might become righteous before God. The baptism with water in the Jordan is the consecration to the baptism of blood on the cross, so that our baptism might become a bath of cleansing, in which we receive forgiveness of sins and the gift of the Holy Spirit. Our baptism is a baptism into Christ’s death, in which the blood of the Lamb cleanses us from all our sins; for it is the blood of the Son of God.

Purchased with the blood and baptized into death, we belong to Christ, friends. Our life is no longer our own, but his who loved us and gave himself for us. If my sin cost you, my Savior, your life, then I am yours in life and in death.

And when soon the hour of death comes, and I must walk through the valley of the shadow of death, then this is my refuge and my confidence: “The blood of Jesus Christ, the Son of God, cleanses me from all my sins”; for the Son of God has become the Lamb of God for me.

Thus it is better to die with Christ in order to live eternally with him, than to live with the world in order to die eternally with it. O souls who have been purchased with the blood of Jesus, hasten to listen to the Baptist’s glorious testimony: “Behold the Lamb of God, who bears the sin of the world”; for this Lamb is the Son of God.
