
\section{First Sunday after Trinity: Self-Denial and Renunciation of the World}

\begin{quote}

Matthew 16:24–27: Then Jesus said to his disciples: If anyone wants to follow me, he must deny himself and take up his cross and follow me. Whoever wants to save his life will lose it; but whoever loses his life for my sake will find it. For what does it profit a man to gain the whole world and lose his soul? Or what will a man give in exchange for his soul? For the Son of Man will come in the glory of his Father with his angels, and then he will repay each according to what he has done.

\end{quote}

\bigskip


The old text for the First Sunday after Trinity we all know; it is about the rich man and Lazarus. The rich man’s joyful life ended in eternal torment; Lazarus’s wretched existence ended in eternal blessedness. In our text today the reason is explained why it came to so different an end with the rich man and Lazarus.

Whoever seeks eternal glory must deny himself, renounce the world, and follow Jesus. These three things are required for the one who desires to reach eternal life.

Anyone who wants to satisfy his own desire and seize the world’s constant pleasures cannot expect that eternal glory will crown a life which has been spent in the pleasures of sensuality and the bondage of vanity.

The narrow way of life is self-denial. But it is self-denial in the footsteps of Jesus, and it is wholly different from that self-denial which the world also knows. Many have punished themselves in many directions, made themselves suffer, relinquished pleasures and comforts for the sake of worldly aims. The world admires this, if it is crowned with success; it is often adorned with honorable names; it is called strength of character and perseverance, it is called frugality and self-sacrifice; but it is not the full and unconditional self-denial of which Jesus speaks. That goes infinitely further. It does not consist in denying oneself in certain directions in order thereby to attain greater enjoyments elsewhere; but it means giving up myself—my own will, my own claim—so that God becomes my all in all. For to deny oneself in the Christian sense is to lose one’s life.

The natural man lives for his own pleasure shut up inside the sensual and the temporary. His sorrows and cares, his joys and satisfactions, his desires and strivings are all confined to that same narrow sphere. And because the natural man in all this, in work, in sorrow, in joy, in distress and in desire, lives without God and transgresses his commandments, therefore this whole natural life is a life in sin, be it ever so outwardly upright. Whoever is to be saved must lose this life; not simply by dying the bodily death; for he who by compulsion loses that to which his heart clings does not become glad, but sorrowful; he does not become happy, but miserable; he does not become blessed, but accursed; he is not saved, but lost. So it is not bodily death that frees us from the bondage of self-indulgence with its pain and its pleasure; but it is the death of self-denial at Christ’s cross.

Go with him to Golgotha and learn what it means to stand as a sinner before God; see what has happened to God’s Son, to the green tree, and tremble when you think what will happen to you—the dry wood. Open your eyes and see that your life in sin and sensuality, that life which you live for your own pleasure, is a life under God’s wrath and ends in eternal torment. Learn what it means that the world passes away with its lust; see what it is to fall into the hands of the living God. 

Then your proud heart will be bowed and your self-willed will broken. Then you will receive a broken heart and a crushed spirit. Then you lose your old life in the service of sensuality and vanity. Then you begin to feel the dread and the shuddering of eternity. Then you see your own helplessness when chasms open around you on every side. Then you learn what true self-denial is, when you sink helpless, broken, lost down at Christ’s cross and lift your praying hands to him: 

\textbf{Lord, save me, I perish!}

That is the hour of self-denial; then the whole old life is lost. From then on it remains a life of self-denial, a life in fellowship with Christ’s sufferings.

Thus the narrow way of life becomes also a way of renunciation of the world, a way of the cross. For he who dies with Christ to his old self, he also dies to the world. Eternity has become his all; to escape eternal ruin, to attain eternal blessedness, that is what matters, and nothing else. He has become a child of God. Heaven is his homeland, and in order to reach it he turns his back upon the world. But to live outside the world and yet be in the world, that is the inner cross which belongs to all Christians. And to bear the world’s contempt and scorn because one has despised the world, that is the outer cross at which no Christian should be scandalized. For if you turned your back upon the world in order to go toward heaven’s blessedness, can you expect anything else than that the world rages and hates, when it has spread out its glory before you in vain and enticed you with its pleasure?

And it is not enough that the Christian feels what it costs to renounce the world. It is not only this that a Christian must bear, that inwardly loosed from the world he becomes the world’s mockery and men’s contempt. But since it will often happen that the deceitful heart again fastens itself upon the temporal and perishable things, the Lord deals graciously with his children and spares them temptations and chastens them for falls, in that he lets poverty and sickness and sorrow fall upon them, so that they become as Lazarus, full of sores. This is no burden for the one who presses on toward eternity’s glory; it is a relief for the spirit because it oppresses the flesh. It drives away from this perishable world forward toward the glory, if it is borne in Christ’s spirit. The Lord takes away what has become for us a temptation which we could not endure; let us beware lest sorrow over the loss become a new temptation in which we fall and are lost.

But the third thing which belongs to the narrow way of life is to follow Jesus. Without this there is no gain either in self-denial or in renunciation of the world. For to follow Jesus is to live the life of love after him; to love the Father in heaven, to love the people on earth; to love God who hates sin, to love sinners who are perishing in their sins under God’s wrath. If our self-denial and renunciation of the world are not borne and governed, not permeated and animated by Christ’s love, then they are nothing; as Paul says: If I distribute all my goods to the poor, and if I give my body to be burned, but do not have love, it does me no good.

It is a wretched life to live in self-denial and renunciation of the world without Christ’s love. It becomes a hard and cold, a bitter and evil, a heartless and spiritless existence, which confidently claims to be Christianity, but alas, alas, it is not. It has Christianity’s pain, but it lacks its joy; it has taken the bitter, but it has neglected the sweet; it has learned to know sin and death and perdition; but it has not experienced grace and restoration and blessedness. 

O what misery—what a wretched state! 

You have halted halfway, dear soul; you have let go of the world, but you have not grasped Christ; you have lost your life, but you have not received God’s life instead. For God’s life is the life of love in the Holy Spirit.

It is not enough to deny yourself; you must also sacrifice yourself for Christ’s cause; it is not enough to renounce the world, you must also take all the worldly and earthly things you have and use them in Christ’s service. It is useless if you do not delight in riches and the world’s goods, if you let your wealth lie unused where it could clothe the naked and feed the hungry. It does no good if you live simply and frugally, if you only let it be a dead treasure with you, which others consume in their pleasures. O no, follow Jesus’ example and step in his footsteps! See how they are marked by blessing! Then your life in self-denial and renunciation of the world will also become a blessed life, when it is also a life in following Christ.

When your heart closes itself seriously, with courage and firm resolve against the world’s pleasure, O let it spread itself freely and blessedly in living love toward your heavenly Father and toward your fellow human beings. When you are as a bronze wall and an iron pillar against the powers of evil, O be loving and gentle toward the poor sinner, man or woman, whom Satan has bound in his chains and ensnared in his nets. When you hate sin, yet have compassion on the sinner. When you rebuke Jerusalem’s sin, O do it with tears in your eyes and weeping in your heart, because Jerusalem did not know the time of its visitation.

Self-denial and renunciation of the world must be purified and understood in following Christ; for it is no Christianity to die with him if we do not also live with him. But Christ’s life is Christ’s love. May it become our life, and then it will also become our blessedness, when the Son of Man will come in his Father’s glory with his angels and repay each according to what he has done.

