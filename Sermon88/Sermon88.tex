\section{Mary’s Annunciation Day: The Wonderful Ways of God}

\begin{quote}

Luke 1:46–55. And Mary said: 

My soul magnifies the Lord, and my spirit rejoices in God my Savior; for he has looked upon the low estate of his servant. 

For behold, from now on all generations shall call me blessed. The Mighty One has done great things for me, and holy is his name. 

And his mercy is from generation to generation toward those who fear him. He has shown strength with his arm; he has scattered those who are proud in the thoughts of their hearts. 

He has brought down the mighty from their thrones and exalted the lowly. The hungry he has filled with good things, and the rich he has sent away empty. 

He has taken up his servant Israel, remembering his mercy, as he spoke to our fathers, to Abraham and to his offspring forever.
\end{quote}

\bigskip

“My thoughts are not your thoughts, and your ways are not my ways, says the Lord.” Truly, he had chosen a strange way to take up his servant Israel—whom he had engraved on both his hands—and to remember Abraham and his offspring forever.

With the strength of his arm he scattered those who were proud in their thoughts and cast the mighty down from their thrones; but the lowly he exalted.

He let the rich go away empty, but the hungry he filled with good gifts.

Mary, a poor and unnoticed Jewish maiden, saw this beforehand in the Spirit and therefore praised God. Eighteen hundred years later we look back with astonishment upon the mighty and wondrous deeds of the Lord and mingle our weak voice with Mary’s song of praise.

In a secluded valley among the mountains of Judea the glory of the Lord is revealed. Two women, who both in wondrous ways were to become mothers, meet and greet one another; the power of God’s Spirit seizes them and stirs their hearts from the depths with joy and thoughts of salvation.

“Blessed are you among women,” Elizabeth exclaims, “and blessed is the fruit of your womb; and how has it happened to me that the mother of my Lord should come to me?”

And Mary answers: “My soul magnifies the Lord, and my spirit rejoices in God my Savior, because he has looked upon the lowliness of his servant; for behold, from now on all generations will call me blessed.”

None of the contemporary pagan historians mention this meeting between the two Jewish women. Yet it is more important than all the meetings and battles of emperors and kings combined. The words they exchanged outweigh all the writings of philosophers together.

For this meeting, as it were, introduces the birth of the Forerunner and the Savior, and the words contain the Gospel for all peoples—but all in God’s own wondrous way, in a way that seems impossible to us.

On David’s throne sat Herod, of Esau’s lineage, half Jew and half pagan—Jew in order to rob power from the promises of God’s people, pagan in order to sneak his way into power from the Roman world empire. In this way, through hypocrisy toward Israel’s sanctuary and servitude to worldly power, through crimes and cunning, Herod had become one of the world’s mighty. In his pride he imagined he could restore David’s kingdom by earthly means. The two women in the mountains of Judea knew better than the mighty king of the Jews, who nevertheless had built Solomon’s temple more splendidly than it had ever been before.

In vain he sought to ensnare the wise men from the East with his old hypocrisy in order to destroy “the King of the Jews” who had been born; in vain he let the blood of innocent children flow in Bethlehem for the same purpose; in vain his son had the Forerunner murdered and, together with Pilate, crucified the Savior, the King of the Jews; in vain his descendant had the apostle James executed and Peter imprisoned. The Lord was stronger than Herod and his house; he laid his heavy hand upon them, cast them down from their thrones, let them be eaten alive by worms, scattered the whole house, and caused their kingdom to be utterly destroyed.

But the lowly he exalted. The humiliated, despised, crucified Jesus he tore out of the hand of his enemies, out of the hand of death and of Satan, so that the house of Israel might know with certainty that God has made him both Lord and Christ—this Jesus whom they had crucified—and has given him a name that is above every name, so that at the name of Jesus every knee shall bow, of those in heaven and on earth, and every tongue confess that he is Lord, to the glory of God the Father. This has been done by the Lord, and it is wondrous in our eyes.

“On Moses’ seat sit the scribes and the Pharisees.” They thought themselves rich beyond all others in spiritual treasures. They were Abraham’s offspring; they had the covenants, the worship, the promises, and the giving of the law; they had all knowledge and all insight; they had the commandments of the law explained many times over and could fulfill them all and boast of righteousness before God. What wonder, then, that in the pride of their riches they took offense at the Forerunner, who bore witness to them of sin, and conceived a deadly hatred toward the lowly-looking Savior, who had come only to seek the lost and to give them forgiveness of sins!

What wonder that when the Savior rebuked them for their sins, they in hatred and wrath fulfilled his words and said: “There is the heir; come, let us kill him and take the inheritance ourselves!” They killed him and hung him on the tree of the curse; but they received no inheritance; they had to go away in shame with nothing in their hands. The Savior’s words were fulfilled: that the kingdom would be given to a people who would bear its fruit. From east and from west they would come—poor, hungry, miserable, despised—and recline at table with Abraham, Isaac, and Jacob, while the children of the kingdom would be cast out, where there is weeping and gnashing of teeth.

These rich Pharisees, who had abundance in everything and lacked nothing, and who therefore were angered at the poor Savior when he offered them forgiveness of sins by faith—how empty-handed and miserable and poor they had to depart from the presence of the Lord whose Son they had slain. “Crucify! crucify! His blood be upon us and upon our children!” they had persuaded the people to cry, but how deeply they had to sink. “We have no king but Caesar!” was their humiliating confession. And when they had finally crucified him whom death could not hold and persecuted his disciples everywhere, how they had to drink the cup of disgrace to the last drop, when the Lord with his mighty hand left not one stone of the temple upon another and scattered them, temporally and spiritually stripped and empty-handed, over the whole earth unto this day.

But the hungry he gave good gifts. Sinners, tax collectors, Gentiles, who in their distress turned to the Lord—on them he had mercy and gave them forgiveness of sins, life, and salvation by faith in the crucified one; he gave them all things with Jesus Christ. Have we not then received riches enough? And has not Mary’s song of praise come to glorious fulfillment over us, who by faith have found mercy?

Yes indeed, the Lord has taken up his servant Israel in remembrance of mercy toward Abraham and his offspring—not offsprings, as though of many—as Paul says—but of one: Christ, and those who by faith and baptism are clothed with him and have become children of God. As it went with Herod and the Pharisees, so it has gone from century to century, and it is wondrous in our eyes: the mighty cast down, the rich sent away with nothing, but the lowly, the crushed and bowed down in heart, these the Lord has exalted; and the poor, the poor in spirit, he has made rich beyond all description, so that though they have nothing, they yet possess all things.

Thus he has, in the midst of the season of fasting and sorrow, when the congregation of God with bowed hearts follows Jesus on his heavy path of suffering, prepared for us a table in the wilderness, a moment of joy in affliction, so that our hearts together with the two women in the mountains of Judea might lift themselves up in praise, our spirit rejoice in God our Savior, and our soul magnify the Lord: “My soul, bless the Lord and forget not all his benefits, who forgives all your iniquity and heals all your diseases, who redeems your life from the pit, who crowns you with mercy and compassion.”

Come here in comfort, you who hunger for righteousness, you who have lost all joy in the world, yes, who are rejected and despised by the world and see nothing before you but an eternal judgment—you poor, lowly, wretched sinners who fear that you are too great transgressors for the Lord to have mercy on you—come with assurance: our Savior has descended deeply enough to pay for it all. “Can a woman forget her nursing child? I, I have not forgotten you, says the Lord.” Come here in comfort once more; let your hearts in childlike faith with the crucified one experience the truth of these words: “The Lord has exalted the lowly and filled the hungry with good gifts.”

Come, let us unite with Mary in praise and thanksgiving: “The Lord has done great things for us; holy is his name, and his mercy endures from generation to generation over those who fear him.”

