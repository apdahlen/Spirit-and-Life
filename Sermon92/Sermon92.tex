

\section{Palm Sunday: The Grain of Wheat}


\begin{quote}

John 12:20–33: But there were some Greeks among those who had come up to worship at the feast. These then went to Philip, who was from Bethsaida in Galilee, and asked him: Sir, we wish to see Jesus. Philip comes and tells Andrew this, and Andrew and Philip tell it again to Jesus. But Jesus answered them:


\textcolor{oxblood}{The hour has come that the Son of Man should be glorified. Truly, truly I say to you: Unless the grain of wheat falls into the ground and dies, it remains alone; but if it dies, it bears much fruit. He who loves his life shall lose it, and he who hates his life in this world shall keep it unto eternal life. If anyone will serve me, let him follow me; and where I am, there shall my servant also be; and if anyone serves me, him shall the Father honor.}

\textcolor{oxblood}{Now my soul is terrified; and what shall I say? Father, save me from this hour! Yet for this reason I have come to this hour. Father, glorify your name! Then there came a voice from heaven: I have both glorified it, and I will glorify it again. The crowd that stood and heard it said then that it had thundered; others said: An angel has spoken to him. Jesus answered and said: This voice did not come for my sake, but for your sake. Now judgment comes upon this world; now the ruler of this world shall be cast out. And I, when I am lifted up from the earth, will draw all to myself. But this he said in order to signify what death he was to die.}

\end{quote}


\bigskip


Our text tells us that some Greeks, who had come to Jerusalem to worship at the Passover feast, came to Jesus’ disciples and said: “We wish to see Jesus.”

When the disciples came to Jesus with this message from these foreigners, he answered them: “The hour has come for the Son of Man to be glorified.”

% "That hour" crescendo

This Passover feast, then, is the Savior’s hour of glorification. Now that hour has come which he has awaited; that hour in which the eternal counsel for the salvation of the human race is to be fulfilled; that hour from which blessed effects shall proceed for all times and all eternities.

The Greeks’ simple words, “We wish to see Jesus,” led the Savior occasion to call it the hour in which he was to be glorified; for the salvation of the nations is his glorification, and this Passover feast is the hour in which our salvation is won.

But the hour of glorification is an hour of suffering. At this Passover feast the Passover lamb is to be slain, the Lamb of God who bears the sin of the world. Therefore the Savior says: “Unless the grain of wheat falls into the ground and dies, it remains alone; but if it dies, it bears much fruit.”

This, then, is Jesus’ glorification: that he dies in order to become the Savior of the nations; that he gives his life over into death, so that through his resurrection he may bring life and incorruptibility to light.

Therefore the hour of glorification is a dreadful hour, of which the Savior says: “Now my soul is terrified; and what shall I say? Father, save me from this hour! Yet for this very purpose I have come to this hour. Father, glorify your name!”

It is the Son’s struggle to do the Father’s will and to reveal his counsel of love for our salvation. So deep is the abyss of suffering into which the Son must descend for the sake of the fallen race that he prays for a moment to turn back—but not to refuse the path. Overwhelmed by the dreadful guilt and offense of our sin, he looks only to the glorification of divine love through the atoning death; and though he feels the full bitterness of death, he prays that it may come upon him, so that the Father’s name may be glorified and the fallen race be reconciled with God.

His struggle with suffering ends in complete surrender to the Father’s will; and from heaven comes the Father’s testimony concerning the Son, that the hour of suffering, bitter though it is, is nevertheless the hour of glorification. This gives blessed assurance of victory, and the Savior testifies that the hour of his suffering and death is his hour of victory, in which power is taken from the ruler of this world; and henceforth the exalted Savior from the cross will stretch out his arms toward all people to draw them into the embrace of grace. There is no power strong enough to keep a human being bound in the bondage of sin any longer, if that person is willing to follow the drawing of the crucified Savior. Satan is bound, death is conquered, sin is atoned for, wrath is stilled.

What do you think, O soul—shall your own resistance nevertheless become your perdition, despite the Savior’s drawing? Shall all of God’s love and all of Christ’s suffering be wasted on you, because you will not come when the Savior calls? Your own heart’s resistance is now the only obstacle to your salvation. Shall it prevail—and you lose, lose the salvation which an eternal Father-love prepared for you?

% Christ's drawing?

O no, do not resist him! Open to him who knocks at the door of your heart, and God’s name shall be glorified in you through your salvation. You too shall become a grain of wheat that is gathered into the granary, if you let Christ’s drawing power overcome you, so that you are united with him, who himself is the true grain of wheat that was laid in the ground in order to spring up from it and bear a blessed harvest.

Already we see how the wheat grows and the heavy head is filled. Already we see God’s kingdom grow forth like a mighty tree from the little seed that was cast into the field of the world. Where are you, O soul? Are you among the noble fruit of the bloody sowing, or are you only chaff, ready to be burned? God’s kingdom can go forward without you—can you do without God’s kingdom?

O that we all might be drawn in to Christ’s cross, so that we all might become fruit of the grain of wheat and honor God’s love by being saved by his grace! But Christ’s way to glory is also the way of all his servants. \textbf{As he died, so must we die, in order to live.} Our entire old, sinful life, our entire corruptible and sensuous life with all its enjoyment and desire, we must give over, so that we die from sin and the world in order to live for God.

It is the hour of suffering, but it is also the hour of salvation. When a person gives up everything in order to win salvation in Christ’s death and resurrection, that person suffers the pain of crucifixion, but also gains an eternal life and an eternal glory. Let go of everything you have, and everything to which your heart clings among earthly, sensuous, and corruptible things; then you will indeed feel the agony of crucifixion, but if you then grasp Christ, dead and risen, you will also taste the blessed joy of the resurrection. Only in this way do you share in the life of the grain of wheat; only in this way do you yourself become a new grain of wheat, sprung forth from the true grain of wheat, a living member of Christ’s body.

But not only that. If you wish to be Christ’s servant and go to work in his vineyard, then know that it is the same way for you as for him: the way of self-surrender and self-sacrifice. If there is to come fruit from your life, fruit for God’s kingdom, then prepare yourself to suffer hardship as a soldier of Jesus Christ. You must give your entire life if there is to be any blessed fruit. From half surrender, half heart, half life, and half death there comes only imperfect fruit.

Be unsparing in your surrender in God’s work and do not spare yourself; then you too shall become like a grain of wheat that falls into the ground and dies, and from which much fruit grows. Your life in God is not given to you so that you should live for yourself alone and win for yourself an eternal glory. If you are to be to the praise of God’s love, then let your life be a service of love as an offering, in which you live for God’s people and God’s kingdom, so that there may be blessing with you wherever you go, and blessed fruit from your labor.

See what fruit there is from the lives of those who truly have followed in the Savior’s footsteps and sacrificed themselves wholly for the Lord’s cause! Let us also follow in the same footsteps; and if we suffer with him, it is nevertheless only incorruptible honor to suffer for him who suffered for us. And our brief hour of suffering shall become a bright day of eternal glory for us.

Awake, you soldiers of the Lord.

And let there be a whole spirit of sacrifice among us, so that the cold spirit of selfishness of the world must retreat in shame before the power of Christ’s love, and the Lord’s field shall put forth a glorious harvest among us, to the praise of God’s love and to the salvation of human hearts.

Stake your life, and you shall win an eternal life; stake your life, and you shall see blessing and rich fruit from your life. Do not fear the pain: the way of glory is a way of tribulation. For God’s love is revealed in Christ’s death on the cross for the sin of the world, and only through Christ’s cross are our hearts filled with God’s love, so that we also are able to present ourselves as a living, holy, and God-pleasing sacrifice.

