

\section{Second Day of Christmas: The Pain of Christmas and the Joy of Christmas}


\begin{quote}

John 1:1–14. In the beginning was the Word, and the Word was with God, and the Word was God. He was in the beginning with God. All things came into being through him, and without him not one thing came into being. In him was life, and the life was the light of men. And the light shines in the darkness, and the darkness did not grasp it. There was a man sent from God; his name was John. He came as a witness, to bear witness to the light, so that all might believe through him. He was not the light, but came to bear witness to the light. This was the true light, which enlightens every human being coming into the world. He was in the world, and the world came into being through him, and the world did not know him. He came to his own, and his own did not receive him. But to all who received him, he gave authority to become children of God, to those who believe in his name; who were born, not of blood, nor of the will of the flesh, nor of the will of man, but of God. And the Word became flesh and dwelt among us, and we saw his glory, glory as of the only‑begotten from the Father, full of grace and truth.

\end{quote}

\bigskip

The angels proclaimed great joy to the shepherds, and added: “It shall be for all the people.” Today’s text shows us why the joy is so great that angels must sing it out over the earth. The child who is born to us, the Son who is given to us, the child in the manger, is “mighty God, everlasting Father” (Isaiah 9:6). The text also shows why not all people receive the joy that God intended to grant to “all the people.”

The Word, the Word of life, is the gospel contained in Christ himself, or the complete and all‑sufficient, all‑embracing revelation of God’s eternal being and his infinite love. Christ is the gospel, for in him the glory of God is revealed, so that all flesh may see it, that glory which is full of grace and truth.

What believers in the Old Testament saw only dimly in a mirror, that God was a Father with a father’s love for all his children, Christ revealed when he himself became a small, poor child among human beings, when he became like us in all things except sin, so that we, through faith, might become wholly like him.

How infinitely great this Jesus is, even when he lies as a small child in the manger! He is God from eternity. He was with God, and he was God, in heavenly glory before the foundation of the world was laid. And when the world was created, he was the one through whom all things came into being; nothing came into being, and nothing could come into being, apart from this Son who has now come down to the earth and become one of us. He commanded heaven and earth to come into being; he caused the sun to roll across the heavens; he set the stars upon their appointed paths. Infinite is he in power and wisdom from eternity to eternity. In him the Father’s being had its true image, and God’s glory its radiance. Therefore the Father spoke to him and said: Let us make a human being in our image, after our likeness. He was all that the human race was meant to be according to God’s purpose, and infinitely much more.

But sin entered the world; mankind tore itself loose from him who was its life and its light. Then humanity’s true life was shattered; then its light was extinguished. In death and darkness, with blinded heart and wandering spirit, the poor human race wandered over the earth, groping for a way it could not find, striving for a goal it could not reach.

And yet the light was there, even though humanity had closed its heart and eyes to it. It was God’s eternal Word, which shone and gleamed like “a light in a dark place” through Abraham and Moses and all the prophets. Throughout the whole time of the Old Testament there was a dawn‑glow of the coming sun, which shone across the dark earth and called and beckoned to human souls. But they had turned their backs on the rising sun; they stared and stared at the shadows that lay across the earth, and they kept the darkness in their eyes and in their souls, although it was only necessary to turn around in order to see “the light from on high,” which shone through the Word in the mouths of the prophets.

The darkness did not comprehend the light; the eternal, blessed light of God was shut out, and human beings walked in their worldly thoughts and desires, in the impurity and wickedness of their hearts, in pagan idolatry and abomination, or in Jewish self‑righteousness and self‑deification. Darkness closed in everywhere, and hearts did not find the way back to God and his radiant kingdom.

Then came John the Baptist. God sent him so that he might bear witness to the light, point to the Savior, and say to all people: Behold the Lamb of God! For light and life for sinners were to be found only with the Savior who laid down his life for them, just as the lambs died for Israel’s firstborn in Egypt. John was to bear witness so that the worldly-minded might turn back before the Lord came, and believe in him who came with salvation for all who seek salvation, but with crushing judgment for all who loved darkness more than light.

The forerunner did his work, and how much fruit it bore was shown best when the Lord himself came. For a few—oh, all too few—John became a faithful guide who with joy led souls to their heavenly bridegroom; but most paid no heed to it. They perhaps rejoiced for a little while in his light, and then returned to their daily routines and their worldly life.

The Son of God came to God’s people, and he found resistance, indifference, unbelief, contempt, hatred, persecution, and death. He came to his own, and his own did not receive him.

What infinite pain lies in these words. It is the sorrow that has followed the gospel from the very beginning. Those who ought to have been the first to receive the Savior, those who said they had waited so long for him, were the first to reject him. “Oh that my head were waters, and my eyes a fountain of tears, that I might weep day and night for the slain of the daughter of my people,” says the prophet Jeremiah (Jeremiah 9:1). And Jesus himself wept over Jerusalem and said: Oh, if you had known on this your day what serves for your peace, but now it is hidden from your eyes. Thus sorrow also accompanies the joyful Christmas message; and it was not long before the angels’ song of praise on the fields of Bethlehem was replaced by “weeping and loud lamentation,” when the little children were murdered in Bethlehem by the cruel King Herod, who sought Jesus in order to kill him.

As then, so now. The gospel is met with resistance, contempt, and unbelief in the midst of a dead Christendom. For most, Christmas brings no joy in Jesus and no song of thanksgiving for the wondrous child in the manger; no, for the vast majority who are called by the Christian name, Christmas is a time of worldly joy and sinful revelry, by which the life in God is often killed in the hearts of many young people. Still today God’s Word continually utters this heart‑rending complaint over God’s people: 

\textbf{He came to his own, and his own did not receive him.}

Therefore all true Christians also feel a living pain precisely at Christmas over the dreadful indifference and worldliness within Christendom. Can it not become otherwise? Must God’s grace in the Son always meet with such resistance from sluggish, rigid, and defiant human hearts? Wake up, then, as you ought, you congregation of God; wake up, you worldly soul who sleeps so heavily in sin; and Christ, the light from on high, will shine upon you and lead you into God’s peace and God’s blessedness.

For truly there is joy, true and real joy at Christmas for those who receive him. It is not with Jesus as it is with other people. You receive them, you may have a pleasant time with them for a while, and when they leave you again there is perhaps a sense of loss for a little time, and then the waves of forgetfulness close over them, and it has no further significance for you that you met them. But with Jesus it is different. If he is truly received by a human heart, received with the whole distress and need of a grace‑hungry, salvation‑seeking heart, received as only one receives who has learned to hunger for love and receives it for the first time, then there also comes a great and joyful transformation in such a person.

Receive Jesus, and there will be a change in you that has significance for time and eternity. You become a child of God. No one can open the heart to Jesus with the broken soul’s longing for relief and salvation without the Savior also granting salvation and life. Where he enters, he brings his grace and his gift with him, and his gift is God’s own eternal life. You receive him in the bitter sorrow of repentance, with shame over your sin and yet with such deep longing for his grace; and behold, when he enters your heart, joy streams in like warmth with the sun, and you rejoice with unspeakable jubilation. This is the true joy of Christmas. Only the joy of becoming a child of God is the right and true joy over the child of God, Immanuel. Then he receives his honor and his praise. Then the true Christmas song sounds in the heart, a song that is not completed until you stand saved before the throne of the Lamb among the number of the blessed. You begin to see Jesus’ glory as that of the only‑begotten from the Father in the moment of time when you become a child of God; and when you one day see him face to face, you will recognize him, because you see the same glory so infinitely much more clearly.

Do you have this experience of being God’s child, and do you have this Christmas joy, and do you have this living hope? Then Jesus has become your Savior. Amen.

