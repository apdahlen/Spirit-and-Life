

\section{Sunday after Christmas: The Fulfillment of the Promises}


\begin{quote}

Luke 1:68–75

Blessed be the Lord, the God of Israel, that he has visited and redeemed his people, and has raised up for us a horn of salvation in the house of David, his servant, as he spoke through the mouth of his holy prophets from of old: salvation from our enemies and from the hand of all who hate us; to show mercy toward our fathers and to remember his holy covenant, the oath which he swore to our father Abraham, to grant us that we, being delivered from the hand of our enemies, might serve him without fear, in holiness and righteousness before him all the days of our life.

\end{quote}

\bigskip



These words are the beginning of the glorious song of praise which the aged Zacharias lifted up when he had received his speech again on the day when his son John was circumcised and given his name. For a long time he had been mute because of his unbelief; but during his long and heavy silence he had thought and struggled and believed, and now he spoke, “filled with the Holy Spirit.” As Elizabeth his wife, and Mary the mother of Jesus, had already earlier believed and confessed and praised God for salvation through the child who was to be born, so now Zacharias also had become one of those who believed and confessed that the day of salvation had dawned and that the hope of Israel—the hope promised by the Lord and awaited by the fathers—had at last arrived.

And it is no wonder if joy arose in the heart of the aged priest when it finally stood firm for him in the Holy Spirit: now the day of salvation had dawned, now the promises were to be fulfilled for God’s chosen people. Whoever knows what it is to have waited long, waited patiently, waited under great afflictions, waited often under doubt and fear and inward temptation—and then at last to reach the goal of all one’s longings—he can indeed grasp something of the overflowing joy that filled the heart of Zacharias and made its way over his lips in song of praise and jubilation. And yet the unspeakable joy in the soul of believing Israel cannot be fully grasped, for the hope of the fathers was now fulfilled; indeed longing and expectation and tension are multiplied and multiplied again when it is a whole people who wait and hope and long, when it is the expectation of millennia that comes, and when it is the entire plan of the Lord’s salvation that is to be revealed in a single moment.

Can anyone wonder that in such an hour the Spirit of the Lord finds a believing heart wide open, so that he may fill soul and mind and mouth with words from God?

\textbf{It is the promise to the house of David, of which the holy prophets have spoken, that is fulfilled. It is the oath to Abraham that is now kept. The ancient, faithful words of the Lord, upon which the fathers had placed their trust, now showed themselves to be true, as the Lord himself is true.}

The promise to David had pledged a “horn of salvation” for Israel. This means a saving power that could strike down all enemies, that could break all resistance. It was the eternal kingdom that the Messiah was to possess, the kingdom of God that was to overthrow the kingdoms of the world, fill the earth, and itself endure forever. Of this Nathan spoke to David; of this Isaiah and Daniel and all the prophets prophesied. Now it was to come through the child whom the virgin was to bear. This is the wondrous ability of true faith, to see the great in the small, to see the power of God in what the world calls frail. Blessed is she who believed! Blessed also are Elizabeth and Zacharias, who believed with her! Blessed is every soul that believes, believes in the Lord Jesus Christ, and finds in him a horn of salvation against all its enemies, a fortress in times of distress.

And the oath to Abraham—we read it in Genesis 22:16–18:

“By myself I have sworn, says the Lord, because you have done this and have not withheld your son, your only son, I will surely bless you and will surely multiply your offspring as the stars of heaven and as the sand that is on the seashore; and your offspring shall possess the gate of his enemies; and in your offspring shall all the nations of the earth be blessed, because you have obeyed my voice.”

This was the great word that Abraham and his descendants had received from the Lord on that dreadful day when Abraham offered his son and received him back from God as one raised from the dead. And now it was truly to be fulfilled. Zacharias saw in the Spirit that the child in Mary’s womb was that offspring of Abraham who was to possess the gate of his enemies. He saw that his own son John was to be the forerunner and the greatest of the prophets, who was to prepare for the Lord a people set in readiness. Zacharias saw how an innumerable multitude of blessed children of Abraham and children of God would find salvation—peace and life, righteousness and holiness through faith in the promised Messiah, who had already come.

Brothers and sisters, do we also have the faith of Zacharias? Has the Holy Spirit also filled our hearts and our tongues with the joy of salvation in this blessed feast of Christmas? Or do you still doubt, dear brother? Who is a God like our God, who proclaims salvation to Abraham for all the peoples of the earth, and who fulfills his word and remembers his covenant, as we see it with our own eyes? In Jesus it is all fulfilled—everything that the Lord had promised, everything that the fathers had hoped for. If you long with the fathers for a Savior, then come and see; seek your salvation in Jesus, and you will find it.

For in him and through him and with him it is also granted to you to serve the Lord without fear, in holiness and righteousness before him all the days of your life. And this is the proof that he is truly the Savior of the world. If he can save you, you poor child of sin, whom then could he not save among poor sinners?

Without fear you shall serve the Lord when Jesus gives you the forgiveness of all your sins and makes you a beloved child of God. It is only the child with an evil conscience who fears his father; it is only the sinner with unforgiven sin who must tremble and quake before God. When the blood of Jesus blots out all your sins, and the Spirit bears witness with your spirit that you are a child of God, then you no longer fear; then you love God. This is blessedness, this is salvation from all enemies, this is victory over the world and sin and death. Fear is not in love, for perfect love casts out fear. But when love is born and created in the heart, then there is righteousness and holiness. For what is born of God does not sin. And holy is the one who is like God, and God is love.

Thus a sinner is saved by the Lord Jesus Christ. Thus the image of God is restored in us. And what further salvation is needed? If we are beloved children of God who love God because he loved us first, who then shall separate us from the love of Christ? Tribulation, or distress, or persecution, or hunger, or nakedness, or danger, or sword? In all these things we are more than conquerors through him who loved us.

But if Jesus has become your Savior, then you know that in him there is salvation for all peoples. Yet to all the blessing of Abraham has not yet reached. Therefore Christmas is the time for mission. Let the light of the Gospel shine to the ends of the earth; this is Christmas’s urgent call to all who share in its joy. Lift the light high at home and send it far out into the darkness of the world and unbelief, that God’s salvation may be made known to the ends of the earth. If you are glad, then share your joy with the many who are afflicted; you yourself will become gladder thereby, and the kingdom of Jesus Christ will be extended, and his love will receive the reward of its suffering.

Come then, take part in the work that the blessing of Abraham may come to all peoples!

