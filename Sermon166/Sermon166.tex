
\section{Fourth Sunday after Trinity: A New Righteousness}

\begin{quote}

Matthew 7:1–6. Do not judge, so that you will not be judged. For the judgment you pronounce will be pronounced on you, and the measure you use will be used on you. Why do you see the speck in your brother’s eye but fail to notice the beam in your own? How can you say to your brother, “Hold still, I will remove the speck from your eye,” when there is a beam in your own? You hypocrite! First remove the beam from your own eye, and then you will see clearly to remove the speck from your brother’s eye. Do not give what is holy to dogs or throw your pearls before swine, lest they trample them and turn on you and tear you apart.

\end{quote}

\bigskip

There are three things which the Savior through these words admonishes the disciples and all Christians against: to judge others hastily; to blame the faults of others before one has seen and corrected one’s own; and to deal lightly with what is holy.

All three are useful and necessary admonitions; and with good reason every preacher will employ this text earnestly to rebuke that censoriousness, fault-finding spirit, and irreverence in the handling of holy things which, not only among members of the congregation in general but also among converted children of God, often appear in grievous measure.

Mutual suspicion and distrust are not unknown in the congregations. Among those who bear Christ’s name and have been baptized into him, people speak about one another, judge one another, and explain one another’s actions. At times there is a heartlessness—a spirit of self-satisfaction, even of bitterness. It is enough to make one recoil, remembering the Master’s command to love one another and the doctrine of Christianity that its confessors belong to the same Father’s house.

The children of God therefore often divide themselves into small circles and cliques, which mutually sigh over one another, shake their heads over one another, and are continually anxious and troubled with doubt whether “the others” are in truth children of God; one can imagine what becomes of love under such conditions.

Many members of the congregation go to the Holy Supper twice a year just as regularly as they attend to spring sowing and autumn harvest, and not with a tenth part so much seriousness. They send for God’s servants an hour before a near relative dies and are yet fearful that they have “troubled” the pastor too early; and while in many places they stand outside the church door after the service has begun and chatter and trifle and trade, and breathe easier once the sermon is finished, yet they fear that the Baptism over their children has not had the proper power if the pastor has forgotten to make the sign of the cross over them.

Such half-heathen, half-Catholic remnants of superstition often come to light, coupled with indifference toward that which in truth can save a man: the Word of God and faith. So much rashness, self-righteousness, and Christian untruth confront a servant of God that he often does not know where to begin or where to end when he considers this text.

And yet not even the strictest moral sermon, held without respect of persons, can be all that Jesus intended by these remarkable words about judging and measuring and correcting one’s neighbor. We must look deeper. The Savior has assuredly another purpose than merely to admonish against the daily faults and falls to which we are all exposed.

Rational Christianity would gladly see in the Sermon on the Mount a collection of moral precepts of the highest and purest kind, and therein places Christ’s merit, that he, like Confucius, Buddha, and other founders of religion, was able both in life and teaching to present a purer morality than his age. The Rationalists have therefore always confessed that in Jesus of Nazareth they saw an ideal moral teacher. Already this will make it evident to every Christian that Jesus by the Sermon on the Mount intended something altogether different than composing a moral textbook, even a very lofty one. One does not come to know Jesus and his teaching by reason alone.

But first of all it was precisely in opposition to the Pharisees and their moral precepts, stretched into the finest consequences, that the Lord appeared and spoke. It is a misunderstanding when anyone supposes that after an almost two-thousand-year study of the Law of Moses the Pharisees lacked moral commandments of the highest and purest kind as well as of the more ordinary sort. Their error was not that they lacked commandments—indeed, they had plenty, and many of them holy and excellent. That is why Jesus says, “The scribes and the Pharisees sit in Moses’ seat; therefore do and observe whatever they tell you” (Matt. 23:2–3).

Their fundamental error, their spiritual blindness, lay here. They meant and taught that when anyone kept the commandments, as they had earnestly and extensively expounded and interpreted them, such a one thereby became righteous before God.

And as particularly the commandments here mentioned by Jesus, about judging and finding fault with others, are concerned, they are, considered merely morally, neither of higher nor purer sort than that they are incorporated into the common moral consciousness of most peoples in the form of proverbs and the like, such as: Sweep first before your own door; do not stir in other men’s refuse, and so forth.

The Savior intended neither to give, as it were, a brief compendium of the best and noblest found in humanity’s moral consciousness — as the Rationalists assert — nor to supplement the Pharisees’ already overfilled collection of laws with some new commandments.

He came to proclaim a new righteousness.

Therefore he says to his disciples at the beginning of the Sermon on the Mount: Unless your righteousness exceeds that of the scribes and Pharisees, you shall by no means enter into the kingdom of heaven. And the righteousness of a Gamaliel, a Hillel, even a Paul was, humanly speaking, not to be despised; there are many today who believe that it can even stand before God and only wish that they were so pure and blameless as those named. And immediately before he says: I have not come to abolish the Law and the Prophets; I have not come to abolish, but to fulfill.

That is the point. He has come to fulfill the Law—to present it in its full measure, in its true, holy, divine content, an image of God’s own holy being, realized in human form in the person of Jesus Christ. 

In this form the Law is presented not so that men might have a moral ideal to dream about, and the best of them chase after it with abstinence, mortification, and extraordinary worship—things that may have a show of wisdom, but in reality serve only the flesh.

Jesus presents the Law as he does in order to convict of sin, so that sin through the commandment might become exceedingly sinful. He would have us understand that if a law had been given that could give life, then righteousness would indeed be by the Law; but now Scripture has shut up all under sin. The Law cannot save. Therefore the merciful God has prepared a way of salvation by sending his only-begotten Son in the likeness of sinful flesh, and for sin, to suffer and die for us—so that by the Holy Spirit and faith in us the Law’s demand may be fulfilled in us.

Therefore Jesus set forth the Law as he did in the Sermon on the Mount, in order to bring men to the knowledge of sin, to repentance, and to faith in a new righteousness, the righteousness by faith in Christ.

When we compare our text with the old one for the same Sunday in Luke (6:36–42), we find the same. For there the admonition against judging is introduced with these words: Therefore be merciful, as your heavenly Father also is merciful, and judge not. What man can be merciful as God without one: the man Jesus? And God in heaven can admit into his fellowship none but him who is perfect like him. Thus there is required a greater righteousness than that of the Law, than that of the Pharisees, than that of the highest moral teacher; there is required nothing less than Christ’s righteousness, the righteousness of faith.

The parable of the mote in the brother’s eye is in Luke introduced with this question: Can a blind man lead a blind man? Shall they not both fall into the pit?

This casts light over the whole passage. It is precisely in opposition to the Pharisees’ self-righteousness that Jesus gives the Law the interpretation which necessitates a new righteousness. It is the scribes and Pharisees who are blind; the Law, instead of being to them a mirror of God’s holiness to bring them to the knowledge of sin, has in their eyes become a beam that makes them spiritually stone-blind, so that they know not sin, but only faults, transgressions, and trivialities, indeed of the very finest kind, yet still only human, and especially in “others.”

But in this condition there arises precisely the greatest frivolity toward what is holy and the bitterest intolerance toward those who think differently. Where God’s Law is transformed into — be it thousands of — human commandments, or where God’s doctrine of salvation is transformed into human learning, there what is holy is thrown to dogs and pearls have come to swine; and if you dare to say this to them, they will turn on you and tear you apart, as they also did to Jesus when they hung him on the cross.

Well then, Brother and Sister, where do you stand? Have you used God’s Word and God’s holy things as a cloak for your self-righteousness and trampled the blood of the Son of God under your feet?

Or have you by God’s Spirit allowed the two-edged sword of his Law to pierce your heart to reveal sin and bear witness of righteousness and judgment? Has he given you grace to remove the beam from your own eye and truly see? In the shame of your nakedness and under the weight of condemnation, have you fled to Golgotha, to be washed in Jesus’ precious blood and receive a new righteousness?

Have you thus had Christ’s love poured out in you by the Holy Spirit, so that you no longer judge your brother, but love him and would even be willing to be cut off, if only one heart might be won for Jesus?

How different things would be among us then.

Brother and Sister, where do you stand?

