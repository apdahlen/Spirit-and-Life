

\section{Ascension Day: He opened their understanding, that they might understand the Scriptures}


\begin{quote}

Luke 24:44–53: But he said to them: “These are the words which I spoke to you while I was still with you, that all things must be fulfilled which are written in the Law of Moses and in the Prophets and in the Psalms concerning Me.” Then he opened their understanding, that they might understand the Scriptures, and said to them: “It is written, and the Christ must suffer and rise from the dead on the third day, and that repentance and forgiveness of sins be preached in his name to all nations, beginning from Jerusalem. And you are witnesses of these things. And behold, I send the promise of my Father upon you; but remain in the city of Jerusalem until you are clothed with power from on high.”

And he led them out as far as Bethany, and he lifted up his hands and blessed them. And it came to pass, while he blessed them, he was parted from them and carried up into heaven. And they worshiped him and returned to Jerusalem with great joy, and were continually in the temple, praising and blessing God. Amen.

\end{quote}

\bigskip


“I came forth from the Father and have come into the world; again, I leave the world and go to the Father.”

Thus Jesus said to his disciples (John 16:28), after he had first spoken a word to them which they did not understand: “A little while and you shall not see me; and again, a little while and you shall see me, because I go to the Father.” “What is this that he says,” they said, “a little while? We do not understand what he speaks.” That was the state of the disciples’ spiritual understanding on the night before his death.

But when he then uttered the words with which this sermon begins, they declared: “Now you speak plainly; now we know that you know all things, and now we believe that you came forth from God.”

There was therefore in this word such a full and simple Gospel, such a heavenly light, that it penetrated the covering which the humbling Law had laid over their eyes, and they could glimpse something of the perfect way of salvation which the words and life of Jesus intended to open for them and for the world. Yet death still remained, which again would lay mist over their eyes; there still remained the realization of that word, “to come from the Father and to go back to the Father,” in order that they might see with their eyes and become true eyewitnesses of the glory of God, the perfection of the Gospel, and thereby be enabled to look back upon the Scriptures and see them in a new light, both to understand them themselves and to explain them to poor sinners.

This took place in the Ascension.

The Ascension, like the Resurrection, is not an event in the spiritual world alone; it is a real, perceptible fact, by which the disciples could see and verify with their own senses, in that they not only saw Jesus after he had risen, but Thomas had to put his hand into his side and feel the marks of the nails, and Jesus ate fish and honeycomb before them. And as it happened with his Resurrection, so also with his Ascension; it is expressly written that “he was taken up while they beheld, and a cloud received him out of their sight” (Acts 1:9). This event had a profound effect on their spiritual understanding.

In our text, between the Resurrection and the Ascension, it is said that he opened their understanding to understand the Scriptures, as he explained to them that thus it is written—in the Law of Moses and the Prophets and the Psalms concerning Me—that the Christ must suffer, rise again, and that repentance and forgiveness of sins be preached in his name to all nations, beginning from Jerusalem, after he had first sent them the Father’s promise and clothed them with power from on high. It was this that took place through Jesus’ Ascension, by which he, seated at the Father’s right hand, sent the Holy Spirit to the disciples.

For, as Paul says (1 Cor. 2:10): “God has revealed it to us by his Spirit; for the Spirit searches all things, even the deep things of God; for who knows what is in a man except the spirit of the man which is in him? Even so no one knows what is in God except the Spirit of God.” And the Lord himself says that “he, the Spirit of truth, whom the Father shall send, he shall guide you into all truth.”

This is the meaning of Christ’s Ascension: that through it Jesus sent and sends his Holy Spirit, who opens our understanding, that we understand the Scriptures according to the word: “You have an anointing from the Holy One, and you know all things.”

Did the disciples then not understand the Scriptures? No. For Jesus says to the two disciples on the way to Emmaus (Luke 24:25): “O fools and slow of heart to believe all that the prophets have spoken; must not the Christ suffer these things and enter into his glory?” And beginning from Moses and from all the Prophets, he expounded to them in all the Scriptures the things concerning himself.

Herein lay a veil over the Law for the Jews, as it does to this day for all men according to their natural understanding. For “My words,” says the Lord, “are spirit and are life,” and the natural man does not receive the things that belong to the Spirit of God; they are foolishness to him, and he cannot know them, because they are spiritually discerned.

All that is written in the Old Testament concerning God’s dealings with the evil and rebellious yet chosen people—that all this truly is a prophecy of Christ; that not only God’s glorious deeds with them in Egypt and in the wilderness, but also their deep humiliation and heavy judgments, which they had indeed deserved on account of their sins, were a promise of Christ, his suffering and exaltation; that every single thing in Israel’s history—the Tabernacle, the Ark of the Covenant, the worship, yes, that the Rock from which the water flowed was Christ (1 Cor. 10:4), and that the bronze serpent was Christ (John 3:14), and that the Law was not a way of salvation but only a tutor leading to Christ (Gal. 3:24)—this was hidden even from the disciples, so that they were offended and terrified at his shameful suffering and death, until the Ascension cast light upon the whole, and the outpouring of the Spirit at Pentecost thereafter entirely removed the veil, so that they could see clearly.

Therefore that same Peter, who in carnal fear had denied his Savior, could on that day stand forth before the whole people and say: “This is what was spoken by the prophet Joel,” and that God had thus made Jesus of Nazareth both Lord and Christ, until the people had their eyes opened with terror and asked, “Men and brethren, what shall we do?” And thus Stephen could in his speech review the whole history of Israel and show that it all pointed to Christ, and cry out: “You stiff‑necked, you always resist the Holy Spirit.” And Paul in his first sermon (Acts 13:38) concludes thus: “Be it known to you therefore, that through Jesus of Nazareth forgiveness of sins is proclaimed to you; and from all things from which you could not be justified by the Law of Moses, by him everyone who believes is justified.”

Thus was their understanding opened by the Ascension and the outpouring of the Spirit to understand the Scriptures, and “we have therefore the prophetic word made more sure, and you do well to give heed to it as to a light that shines in a dark place, until the day dawns and the morning star arises in your hearts” (2 Pet. 1:19).

And now, brothers and sisters, has the Ascension accomplished this in your heart—not only to make you heavenly‑minded, so that you desire to be with Christ, but that you also daily, by the powerful working of God’s Spirit, truly ascend with Christ, where he sits at the right hand of the Father, and from that place with the Spirit’s light look back upon all that God has said and done in his Word, and thus more and more have the eyes of your understanding opened to understand the Scriptures, so that you see that it is all about Christ and about his body on earth, his congregation, his new, holy Israel—and that you therefore, whoever you are, as a witness must both in word and life in his name proclaim forgiveness of sins to all nations and carry the message to the ends of the world, as it began from Jerusalem? If you daily receive such blessing from the Ascension of the Son of God, then you shall also daily from these heavenly heights receive all light and all power, not only to see and know the spiritual host of wickedness under heaven, but also to overcome all the powers of Satan that stir in and around you; and one day you shall stand saved and blessed at Jesus’ side, after having overcome all. And you will not lack those who receive you into the heavenly dwellings.

But if Christ’s Ascension remains for anyone a doubtful or indifferent thing, then there will gradually be laid a thicker and darker veil over the Scriptures; and all that you read and hear without the heavenly light of God’s Spirit will only further close the Scriptures to you and darken your eye. See therefore that the light which is in you is not darkness! But come, just as you are, and look up to the Ascended One, and he shall send you by his Spirit the light of faith and of life into your heart.

