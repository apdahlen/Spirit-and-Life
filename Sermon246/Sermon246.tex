
\section{Twenty-fourth Sunday after Trinity: They knew neither the Scriptures nor the power of God.}


\begin{quote}
Luke 20:27–40. Then some of the Sadducees, who deny that there is any resurrection, came to him and questioned him, saying, ‘Teacher, Moses prescribed for us that if a man’s brother dies, having a wife, and he dies childless, his brother shall take the wife and raise up offspring for his brother. Now there were seven brothers. The first took a wife and died childless; and the second took the wife, and he also died childless; and the third took her, and likewise all seven; they left no children and died. Last of all the woman also died. In the resurrection, therefore, whose wife will the woman be? For the seven had her as wife.’ And Jesus said to them, ‘The children of this world marry and are given in marriage; but those who are considered worthy to attain that world and the resurrection from the dead neither marry nor are given in marriage, for they cannot die anymore, because they are like angels and are children of God, being children of the resurrection. But that the dead are raised, even Moses indicated, in the passage about the bush, where he calls the Lord the God of Abraham and the God of Isaac and the God of Jacob. Now he is not God of the dead, but of the living, for all live to him.’ And some of the scribes answered and said, ‘Teacher, you have spoken well.’ For they dared not ask him anything further.
\end{quote}

\bigskip


The Sadducees were of that kind of people whose whole religion was denial of the resurrection. Just as in our own day there are those who deny that Jesus is God, always ready to dispute and to use their supposed learning to demonstrate that Jesus is never expressly called God in Scripture, so the Sadducees, with their worldly wisdom, were always prepared to prove that Moses teaches nothing about the resurrection of the dead; and thus their entire religion was built on the denial of this doctrine of God.

This suited the carnal life perfectly, to be able to convince themselves and others that there is no life after this one, that they have no accountability, and to encourage one another in the service of their belly with these words: ‘Eat, drink, and be glad, for tomorrow we die.’

They imagined themselves superior, and thought they demonstrated their superiority when they could dispute with Jesus about the resurrection.

But it went with them as with most other learned religious arguers. With all their sharp-witted learning, they were ignorant of the main thing, of the catechism, so to speak; they knew neither the Scriptures nor the power of God (Mark 12:24). Whoever seriously applies himself to these two will have better things to do than to dispute.

A man who was not in every respect faithful to the truth wished to have an excuse and a defense for his falsehood. He thought that if he could first prove that an emergency lie was justified, then he could himself decide when a lie was an emergency lie.

He said to a friend: ‘A madman with an axe in his hand was pursuing another and stopped where the street divided in two, not knowing which way the pursued man had gone. There he met another person and said to him, “Tell me which way my enemy went, or I will kill you.” What should this person do? If he keeps silent, he himself will be killed; if he tells the truth, the other will be killed. Was it not necessary here to use an emergency lie?’

The friend replied: ‘I will answer yes, if you will promise me never to speak an untrue word until the above-mentioned situation occurs.’

Jesus, however, answered the Sadducees in a far more grave and searching manner, when they, in order to create a semblance of proof for their fleshly denial of the resurrection, produced an equally unreasonable story about the woman and the seven brothers who had been married to her.

He did not engage in their foolish and dishonest example. As was his custom, he went straight at their hearts and souls.

\textbf{You Sadducees, he would say, deny the resurrection of the dead, and the reason is not that you possess any sound proof for doing so; the reason is that in your carnality and spiritual blindness you know neither the power of God nor the Scriptures.}

If you knew the power of God and were not so clouded by your own sensuality, then you would know that life in this world and life in the world to come are two entirely different things. In this world, life—even the very best—is in the flesh, corruptible, and mortal. Since we all must die here, and since God’s plans of salvation could not be realized without the continuation of the human race, the Lord from the beginning gave this command: ‘Be fruitful and multiply and fill the earth.’ Therefore it belongs to life in this world to marry and to give in marriage.

\textbf{But there, where God has revealed his power in a man and raised him from the dead, there begins a life in Spirit, in incorruptibility and immortality. Then death is swallowed up in victory, and the man can no longer die, but becomes like the angels, with a transfigured, glorified body, whose nourishment is the vision of the Lord’s face. There they neither marry nor are given in marriage, because each has an eternal existence of his own.}

But neither do you know the Scriptures, despite your imagined learning. You strain out the gnat and swallow the camel. The most important thing escapes your blinded eye. For it stands written in Moses, in a well-known place, that God is the God of Abraham, Isaac, and Jacob. Can the living God be God of dead people? Then Abraham, Isaac, and Jacob, who are dead, must, according to Scripture’s own testimony, be risen to new life. ‘Therefore you are greatly mistaken.’

No wonder that both the people and the Pharisees marveled at his simplicity and authority, and that from that time on they did not dare to ask him anything further.

Friend, you are not like the Sadducees? You do not dare to deny that there is a life after this; you still believe in the resurrection of the dead, and you do well. But have you also considered what the resurrection of the dead means for you?

Do you know the Scriptures and the power of God?

Do you know that the Son of Man has been given authority to carry out judgment, and that the hour is coming when all who are in the graves will hear the voice of the Son of God and come forth—those who have done good to the resurrection of life, and those who have done evil to the resurrection of judgment (John 5:24–29)?

Thus the Scripture sounds, and thus the power is exercised.

Friend, are you prepared?”
