
\section{Second Sunday after Trinity: On Following Christ}

\begin{quote}

%FIXME: Correct when the DAN 1871 section is complete.

Luke 9:51–62: When the time drew near for him to be taken up, he set his face toward Jerusalem. He sent messengers ahead of him, and they went into a Samaritan village to prepare for him. But the people did not receive him, because he was on his way to Jerusalem. When his disciples James and John saw this, they said, “Lord, do you want us to command fire to come down from heaven and consume them, as Elijah did?” But he turned and rebuked them. “You do not know what spirit you are of. For the Son of Man has not come to destroy souls, but to save.” And they went on to another village.

As they went along the road, someone said to him, “Lord, I will follow you wherever you go.” Jesus said to him, “Foxes have holes, and birds of the sky have nests; but the Son of Man has nowhere to lay his head.” He said to another, “Follow me.” But he said, “Lord, let me first go and bury my father.” Jesus said to him, “Let the dead bury their own dead; but you, go and proclaim the kingdom of God.” Another said, “Lord, I will follow you, but first let me say farewell to those at my home.” Jesus said to him, “No one who puts his hand to the plow and looks back is fit for the kingdom of God."

\end{quote}

\bigskip


What is here related took place while Jesus with his disciples was on his last journey to Jerusalem. For in the fifty-first verse it stands written that "when the days were fulfilled that he should be taken up, he steadfastly set his face to go to Jerusalem." His disciples also understood that this was a decisive going, which, according to their thought, would at last bring the expected glory. Therefore they were filled with lofty—almost proud, triumph-confident—thoughts. They already had time to consider "which of them should be greatest"; they were strangely angered at one who cast out demons in Jesus’ name and "followed not with them"; and John and James were so inflamed at the boldness of the Samaritans in refusing Jesus lodging that they asked permission to command fire to fall from heaven and consume them.

Each of those words must have been like a barb in Jesus’ heart, driving him to bow his head still lower! He answered them lovingly and meekly: "The Son of Man has not come to destroy men’s souls, but to save,"—and continued the way with Gethsemane and Golgotha before him.

Ah, poor disciples! They were still far from understanding what a path of suffering this journey was for Jesus, still less that their own reward of "glory" should consist in following after him upon the way of the cross and of renunciation. Their thoughts were yet fleshly.

Under these circumstances, and under the influence of that tension and enthusiasm with which hearts were filled, a man comes to him and says: "Lord, I will follow you wherever you go."

In truth a fair and right confession, and surely sincere! And yet, how deceptive! Not so much for others as for oneself!

Jesus’ answer shows it:

"The foxes have holes and the birds of the heaven nests; but the Son of Man has not where he may lay his head."

It is a grave answer—meant to shatter every fleshly dream about following Christ, and to force us to ask what it really means to "follow him." For there is a state of which Jesus himself says, “the last is worse than the first.”

In the centuries which have run their course since Jesus laid this weighty admonition upon a hasty confessor’s, a newly awakened disciple’s heart, human nature has not changed a hair’s breadth. While it may swiftly be brought to rejoice and be exalted by faith that Christ by his suffering and death upon the cross has paid for all sin, it recoils and withdraws when the Lord reminds: What I have given up, you also must give up; what I have suffered, you also must suffer.

And so, even in our own days there are multitudes of awakened Christians who after a shorter or longer time fall away—amiable, eager, sincere confessors, for whom the cross and renunciation have come so unexpectedly upon their longings for heaven and their hallelujah cries that they have become a stumbling-block to them, and the last has become worse than the first. They did not heed this weighty word of Jesus: "Whoever does not bear his cross and come after me cannot be my disciple." The Son of Man has not where he may lay his head.

But have you considered this?

In contrast to the hasty and fervent yet unreflecting confession which does not reckon "what it will cost to finish a tower," and therefore runs the danger of soon falling away, there is presented in the two following examples a class of Christ’s followers who are also ready and willing, but who are still governed by considerations which in themselves are wholly justified, yet which become a snare, a stumbling-block, and a rock of offense when they are placed alongside of, or even before, the following of Christ.

The Lord says to one: "Follow me." He is ready, but he will be allowed "first to go and bury his father."

It is not wrong to bury one’s deceased father; on the contrary, among heathen and Jews and Christians, among all peoples who yet retain the poorest remnant of religion, it has been regarded as one of the holiest duties to give one’s nearest and dearest a seemly burial. The greatest disgrace—more dreaded by the noblest men than death itself—was to remain unburied, a prey to dogs and wild beasts. The history of Ahab and of Tobit shows fully how highly a seemly burial was esteemed among the Jews.

Nothing is more dangerous or more difficult to break with than customs handed down from the fathers. Nothing angered the most zealous Jews more than that Jesus healed on the Sabbath; and within almost every Church there are customs, mere and simple customs, which have awakened more bitter strife than the most important truths.

But fearlessly Jesus breaks every regard to custom when it comes side by side with, or even in the way of, following him. "One thing is necessary; and first the kingdom of God and his righteousness"—this is his doctrine.

"Let the dead bury their dead," he said. You need not trouble yourself about your dead father; there will be enough others to see that he is buried decently. The more one is dead in himself and without hope for eternity, the more he seeks to deceive others and himself by trying to dress up death with a fine funeral. You must even be ready to bear the shame of neglecting so sacred a duty as burying your own father; for you, everything else must be set aside for these two necessary things: to "follow me" and "to proclaim the kingdom of God."

It was a hard saying; and we are not told that the disciple could accept it. But as Jesus says to one, "Give all that you have to the poor," to another, "Let your father lie unburied," so there is for you, whoever you are, one thing which you either reluctantly or not at all will give up, which by every conceivable means you seek to place beside or in harmony with your Christianity, one thing to which your heart clings more than you will confess or know, so that this thing becomes first, and Jesus second. But the Lord asks for that very thing. Search your heart. Be honest with yourself, so that you are not cheated out of the hope of eternity.

"Let the dead bury their dead!"

There is no bond so holy, no love so strong, no object so precious, that you must not be ready to relinquish it in order wholly and undividedly to follow Jesus. This is what Jesus declares in the third example, where another disciple expressed his willingness to follow Jesus if only he might first take leave of those in his house. It is not mentioned whether they were parents, wife, or children; it is the same. Jesus’ answer is like a two-edged sword exposing the hidden counsels of the heart. It shows in this case that the disciple, perhaps himself unaware, had his first, his strongest love in his home; that he at once beforehand must become clear about this, receive grace to relinquish it according to that word from Jesus’ mouth: "If anyone comes to me and does not hate his father and mother and wife and children and brothers and sisters, yes, and even his own life, he cannot be my disciple," and count even the dearest things on earth among what will be added, and for which a follower of Christ should be free of anxious care; else he would come into the same danger as Lot’s wife, look back to the glory of Sodom and become a pillar of salt.

"No one who lays his hand upon the plow and looks back to what is behind him is fit for the kingdom of God."

After incomprehensible sufferings and mighty manifestations of God’s power, the Israelites found courage to break free from Egypt’s bondage and went forth bravely until they came to the Red Sea. There they stood: the sea before and Pharaoh’s host behind. It might well bring the bravest to despondency. "Go forward," said the Lord; and they went straight into the sea and sang the Lord’s praise on the other side, while Pharaoh’s host lay buried in the deep. But soon it was all forgotten, and hearts stole back to Egypt’s flesh-pots and rebelled and hardened themselves against the Lord. "They were all under the cloud and all passed through the sea; they all ate the same spiritual food and drank the same spiritual drink; but with most of them God was not well pleased; they were overthrown in the wilderness."

They had put their hand upon the plow, but looked back.

Remember Lot’s wife. Remember the serpent-bitten Israelites. Remember your own deceitful heart, and follow Paul’s counsel:

Forgetting what lies behind and straining toward what lies ahead, I press on toward the goal, toward the prize of God’s heavenly call in Christ Jesus.