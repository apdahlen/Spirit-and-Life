
% 28 Feb 26: Remove KJV voice and set emotional volatility.
% 01 Mar 26: Still more KJV voice removal.


\section{First Sunday in Advent: The Year of the Lord’s Favor}


\begin{quote}
Luke 4:16–22. 
And he came to Nazareth, where he had been brought up, and according to his custom he went into the synagogue on the Sabbath day and stood up to read. And the book of the prophet Isaiah was handed to him, and when he opened the book, he found the place where it was written: 

\begin{quote}
The Spirit of the Lord is upon me, because he has anointed me to proclaim the Gospel to the poor; he has sent me to heal those who are brokenhearted, to proclaim release to the captives and recovery of sight to the blind, to set the oppressed free, to proclaim the year of the Lord’s favor. 
\end{quote}
And he closed the book and gave it back to the attendant and sat down, and the eyes of all in the synagogue were fixed on him. And he began to say to them: Today this Scripture has been fulfilled in your hearing. And all bore witness to him and marveled at the gracious words that proceeded from his mouth.
\end{quote}

\bigskip

Will the new church year that begins today be a year of the Lord’s favor for our church?

The answer depends on two things: first, whether the Lord will come—“according to his custom”—into our synagogues, that is, our congregations, and let his Word be heard among us; second, whether we give that Word room in our hearts, so that it becomes for us a power unto salvation.

Praise be to the Lord: the year of the Lord’s favor, which began with his preaching in Nazareth, has not yet come to an end. From that time it has been the year of the Lord’s favor, since then the Gospel has gone out across the world. From Sunday to Sunday, from year to year, from century to century, preaching with power from God and the Holy Spirit has resounded over the earth in thousands upon thousands of places.

Praise be to the Lord that it still resounds. And that it resounds with its old, simple, and life‑giving message. Even if it has at times been darkened and overshadowed by human rules and teachings, it has nevertheless not fallen silent; but as the sun breaks forth again from mist and clouds, so the clear radiance of the Gospel has by the Lord’s grace again shone forth. Still the Lord, through his Gospel, offers the kingdom of God with its grace and gifts to the poor; still he offers healing and restoration to broken hearts; still he offers sight to the blind and freedom to the oppressed; still he opens his arms and says: “Come to me, all you who labor and are burdened, and I will give you rest.” Grace for every poor sinner, free grace, unmerited grace, remains the sum of the preaching, as it always was. The same old distress that again and again afflicts and presses human hearts can still be relieved by the same old faithful remedy, the precious blood of Jesus, which blots out our transgressions and atones for our guilt.

Up to now it has been the year of the Lord’s favor for all of us who dwell in lands that are illumined by the true light of the Gospel, where the Lord’s pure Word and blessed Gospel sound forth in the congregation from the mouths of the Lord’s faithful witnesses. But there are also places from which the Lord has removed the lampstand, and where the Gospel no longer sounds. Where now are Nazareth and Capernaum? Where are Jerusalem and Bethlehem? Where are Ephesus and Antioch and Smyrna? Where are even Rome and Alexandria? Therefore, you who have the light, do not grow proud, but stand in holy fear. Thank the Lord, who has allowed you to keep the light; but pray in deepest humility that he will allow you to keep the light “yet this year.”

And blessed be the Lord, we dare, despite our sins and our lukewarmness, believe that he will still, “according to his custom,” come into our synagogues every Sunday and let the Word be read and preached for us. We dare say that the Lord will still show patience, that he will let the new church year be “the year of the Lord’s favor.” He has not forgotten to show mercy, and he will not close the door of grace. Therefore we still proclaim, in faith and in the Spirit of Christ, the year of the Lord’s favor with Word and Baptism and the Lord's Supper, and with the blessed preaching of Jesus’ witnesses. 

\textbf{But will it also become the year of the Lord’s favor for you?} Dear soul, consider this question carefully. There are many—very many—who are part of God’s congregation and who take part in its work, but who do not share in its joy and blessedness. Why should this be so? If you bear the congregation’s burdens, then also take its goods and joys, and so it will become the year of the Lord’s favor for you.

How can this come to pass? Perhaps you could learn something by simply following our text. Jesus went, “according to his custom," on the Sabbath day into the synagogue. You should begin with the same custom; perhaps you do not yet have this custom. One often speaks so contemptuously of habitual Christianity, and it is indeed true that it is a poor Christianity. But there are habits that are useful for the Christian life. Among them is this custom of Jesus Christ, to go into the synagogue on the Sabbath day. Take it up in earnest in the new church year, and do not let laziness or sluggishness or bad weather hinder you from gathering with God’s congregation in the Lord’s house. And even if the pastor cannot be present in the congregation’s assembly every Sunday, let that not hinder the congregation from gathering for its edification. The Word can still be read—and it can still do its work.

But it is not accomplished with a good habit alone. If this year is to become the Lord’s year of favor for you, then the question is whether you will become one of the poor and brokenhearted, for whom the Gospel offers peace for the heart. If you go through the year content and self-satisfied, then it will not become better for you than all the other years. If your heart is great and strong and hard and whole, so that the Word finds no room in it, then it will not become otherwise than it has been; for with every day that passes, and with every word you hear without repentance and faith, you are storing up wrath for the day when God’s judgment is revealed. If the new church year is to become for you a year of the Lord’s favor, then bow, you proud heart, under the Lord’s law that judges your sin; then let Christ’s love melt the ice around the heart, so that you flee broken and poor to the cross and the blood.

“Can a sinner do anything for his salvation?” you ask. Must he wait until the Spirit of the Lord takes hold of him so powerfully that he cannot resist? Must he wait until a storm of awakening sweeps over land and people? — But have you not heard of him who stands at the door and knocks? Shall he stand outside even longer? He brings salvation with him; he brings healing with him; he brings peace for the heart and blessedness with him; you can take nothing—absolutely nothing—but you can receive everything from him; oh, open up, open up, that he may come in. Whoever has ears to hear, let him hear. If you truly hear, the Word gives faith and peace and life.

A man stands in the shade and freezes and trembles; a friend comes and says to him: “On the other side there is sunshine and warmth—go there.” And the man goes; he may not feel the warmth at once; but if he waits a little while, then the warmth of the sun begins to do him good. Have you tried the same with God’s Gospel? You go to church a single time; you leaf casually through your Bible; you perceive nothing. But remain standing in the light; give the Word time; hear it diligently and read it yet more diligently, and it shall itself plow the ground and break the stone crust and lay living seed down in the field of the heart. Let it be watered with tears and weeping over your own hardness, and the seed shall grow the better. Pray, and wait in faith and hope, and the year of the Lord’s favor shall become your own year of favor, your day of salvation and the times of refreshing from the presence of the Lord.

May the Lord grant us a blessed church year with new glad tidings for our sorrowing and wounded hearts.
