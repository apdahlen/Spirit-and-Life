\section{First Day of Easter: Fear and Joy}

\begin{quote}

Matthew 28:1–8: And when the Sabbath was past, as it began to dawn toward the first day of the week, Mary Magdalene and the other Mary came to behold the tomb. And behold, there was a great earthquake; for the angel of the Lord descended from heaven and came and rolled the stone from the door and sat upon it. And his appearance was like lightning, and his raiment white as snow. And the guards trembled for fear of him and became as dead men. But the angel answered and said to the women: Do not be afraid; for I know that you seek Jesus, the crucified. He is not here; for he is risen, as he said. Come, see the place where the Lord lay. And go quickly and tell his disciples that he is risen from the dead. And behold, he goes before you into Galilee; there shall you see him. See, I have told you. And they departed quickly from the tomb with fear and great joy and ran to tell his disciples.

\end{quote}

\bigskip

In the three hours immediately before Jesus gave up his spirit on the cross, Luke tells that the sun was darkened and that the veil of the temple was rent in two. At his resurrection Matthew reports in our text that there was a great earthquake, and that the angel of the Lord descended from heaven as lightning and rolled the stone from the tomb. The Lord’s death and resurrection was an event that set heaven and earth in motion to show that he who is Lord over death is he who has all authority both in heaven and on earth.

But every revelation of this divine power, small or great, over an individual or over the whole of humanity, works terror and fear in every sinful heart, which thereby comes to know both its impotence and its judgment.

When the angel was revealed in the temple to Zacharias, he was afraid; and when he came to the Virgin Mary, who should become the Mother of God, it was the same with her, wherefore he also said to her: “Fear not, Mary, for you have found grace with God.” When Jesus was revealed to the disciples walking upon the sea, they were terrified; and when after the resurrection he suddenly stood in the midst of them, they were seized with fear, so that he had to calm them by showing them his hands and his feet, and when their fear thereafter passed over into so exceeding a joy that it would almost take from them their faith, he calmed them again by eating before their eyes.

What wonder if fear and terror seized not only the guard, but also the women who on Easter morning came to seek the crucified, when the forces of nature were shaken and the power of the Lord’s glory was revealed and the angel spoke to them: “He is not here; he is risen.”

It is sin which awakens fear when I am at once placed face to face with the holy and almighty God, whether this be accompanied by extraordinary manifestations or whether it occurs through the Spirit’s awakening and testimony in the heart.

My sin! my sin! my sin! — that too becomes a child of God’s continual lament here below, as it weighed upon the Apostle Paul and made him cry out: “O wretched man that I am! who shall deliver me from this body of death?” And although also a carnal and unconverted man will be filled with fear and terror when God’s nearness is revealed to him in some visible and supernatural manner and will tremble before the judgment which he knows in his heart, yet it is a child of God who most deeply feels fear in the face of the revelation of God’s power, not only because a child of God most deeply knows his sin and God’s holy righteousness, but because he has also known the foretaste of the blessedness of eternal life and fears, so long as he walks in sinful flesh, to lose this incomprehensible treasure. For “let him who stands take heed lest he fall;” “I go in peril where I go, and never know myself secure.” What wonder therefore if these women, to whom the angel had proclaimed the blessed tidings of Easter morning: “Christ is risen!” went away from the tomb with fear. Can fear be separated even from the most blessed joy here in the world, unless a man forgets to search his heart at every time and to remember his sin?

I recall an Easter morning when I was to preach. I had been taken up with the thought of the victorious power of the resurrection and had prepared a sermon which in my opinion should resound with that triumph which as a shout of joy ought to pass through the whole congregation on the morning of the resurrection and sound up to heaven to be met by the angels’ answer. I fear that there had entered fleshly enthusiasm into my proclamation and that I had forgotten fear before the revelation of the Lord’s glory. I ascended the pulpit, as it seemed to me, festively disposed and well prepared. It was to become a true “Easter sermon.” I prayed God for his blessing. But when I had come into the midst of reading the old, simple text in Mark and again met the words: “Who shall roll us away the stone from the door of the tomb? And when they looked, they saw that the stone was rolled away,” then I was seized by a strange feeling; my breast began to heave, my throat was constricted, and my eyes were filled with tears despite all my efforts. The letters ran together before me, and my tongue could not bring forth a word. So I stood for a while mute and in a state of unspeakable humiliation and brokenness. The Lord’s holiness and power had been revealed to me, and I had been made to remember my sins. When at last I as it were tore myself free and read the remainder of the text, my whole sermon was gone, and I would gladly have crept under the earth for shame. O how I wished that I had never been a priest! But preach I must, and I know not to this day what I spoke in my fearful and poor condition. Only this I know, that some days thereafter, while I still felt cast down and humbled over the events of Easter Day, a man came to me, opened his heart in many tears, and asked me to show him the way to salvation. He had been awakened by that Easter sermon over which I was ashamed. When he was gone, I felt myself tenfold more confounded and would, like Peter, have cast myself at the Savior’s feet and said: “Depart from me, for I am a sinful man!” But the Lord met me halfway and said: “Take courage; I make you a fisher of men.” It was for me an Easter Day of humiliation, but also of blessing, and I began to understand the women’s feelings when they went from the tomb with fear; but I also came to understand how they nevertheless went away with great joy.

“Love casts out fear,” says the Apostle. No one knows God’s love except he who has been made to see his sin and in his fear and distress has found grace through faith in the crucified and risen one. Where sin abounded, grace abounded yet more; the deeper one is allowed to descend into his own sin, the more firmly he clings to the Savior, the more he grows in faith, and the more blessedly he feels the joy of being saved. Therefore Paul also says after having lamented: O wretched man that I am! — “There is therefore now no condemnation for those who are in Christ Jesus; for if we have died with Christ, we believe that we shall also live with him; for death has no more dominion over him; the death that he died, he died once for sin, but the life that he lives, he lives to God.”

If we cannot approach Christ’s opened tomb without fear when we consider our sin and God’s omnipotence and grace, then neither can we draw near to the risen one except with an unspeakable joy, praise and thanksgiving; for he who was dead, behold, he lives; and we who were dead, behold, we live! If he is risen, then we too are risen; if he has ascended, then we shall be allowed to follow him; and if he sits at the Father’s right hand, then we also shall live and reign with him in all eternity. Precious soul, do you believe this?

O if you must indeed be humbled like Peter and terrified like Mary Magdalene, what unspeakable blessedness to come again early on Easter morning to Jesus’ tomb, and and so believe and know that Jesus is risen — and that you are risen with him — and go away not only with fear, but also with great joy because of God’s love which is poured out in you by his Holy Spirit!

Then there remains only this: like the women, with the conviction of faith and childlike trust, to go out among brothers and sisters and proclaim this truth, which is eternally new and ever powerful: 

\begin{quote} Jesus is risen! 

He is risen! \end{quote}

And as it stands written: “We are buried with him by baptism into death, that just as Christ was raised from the dead by the glory of the Father, so we also shall walk in newness of life” — so begin this Easter morning with earnestness and sincerity, not only with confession of the mouth, but above all with a Christian life, a life in daily fear and poverty of spirit, but also a life in great joy and victory over the flesh, to proclaim first among your own, then to all others, this saving truth — if you yourself have experienced it — “Christ is risen!” From poor sinners, from tried Christians, yes, from the angels in heaven there shall sound back the answer: 

\begin{quote} He is truly risen!\end{quote}
