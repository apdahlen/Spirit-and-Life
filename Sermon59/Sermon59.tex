

\section{Sixth Sunday after Epiphany: From Tabor to Golgotha}


\begin{quote}

Matthew 17:1–9: Six days later Jesus took Peter and James and his brother John with him and led them apart, up onto a high mountain. And he was transfigured before their eyes, and his face shone like the sun, and his garments became white as the light. And behold, Moses and Elijah were seen by them, speaking with him. Then Peter answered and said to Jesus: “Lord, it is good for us to be here; if you will, we will make three dwellings here, one for you, one for Moses, and one for Elijah.” While he was still speaking, behold, a bright cloud overshadowed them, and behold, a voice came out of the cloud, saying: “This is my Son, the Beloved, in whom I am well pleased; listen to him!” And when the disciples heard this, they fell on their faces and were greatly afraid. And Jesus came forward and touched them and said: “Rise, and do not be afraid.” And when they lifted up their eyes, they saw no one except Jesus alone. And as they were coming down from the mountain, Jesus commanded them, saying: “Tell no one of this vision, until the Son of Man has risen from the dead.”

\end{quote}

\bigskip


This was Christ’s course here on earth. And this is also every child of God’s path of pilgrimage in the land of exile.

But this is a great offense to flesh and blood.

For before the Lord went up onto the Mount of Transfiguration, he had begun to teach his disciples that the Son of Man must go to Jerusalem and suffer many things from the elders and chief priests and scribes, and be killed, and rise again on the third day.

The eager but uncomprehending Peter took offense. He likely thought that Jesus, for all his heavenly insight, did not truly understand earthly matters; that he saw everything in far too dark a light; that he went far too far, and so on, as such reasoning always goes.

At any rate, it is written of him that he took the Savior aside and even rebuked him.

Peter rebuked Jesus!

“Lord, spare yourself,” he said; “this shall never happen to you! You go too far in your zeal; you will ruin the whole cause for us.”

The Lord answered him as his fleshly misunderstanding deserved: “Get behind me, Satan! You are a stumbling block to me; for you do not set your mind on the things of God, but on the things of men.”

These were hard words, crushing words; and yet that was not all. It is not enough that I must suffer; you yourselves must also suffer, he says, and drink the same cup which I must drink.

“If anyone would come after me, let him deny himself and take up his cross and follow me.”

This is the path laid out for a disciple of the Lord.

“For whoever would save his life will lose it; and whoever loses his life for my sake will find it.”

This is what a Christian must order his life by.

For what would it profit a person if he gained the whole world and forfeited his soul? Or what can a person give in exchange for his soul?

After such conversations and painful instruction, Jesus took three of his disciples, among them Peter, up onto the Mount of Transfiguration to meet the Lord’s glory.

Jesus needed it, and the disciples needed it, for the heavy journey and struggle that lay ahead.

And it was indeed a blessed hour.

Jesus, a man like you and me, truly human, yet transformed, transfigured, glorified—his face shining like the sun and his garments white as the light—and the heaven-taken Moses and Elijah in his company, speaking with him!

What wonder that the disciples were terrified, overwhelmed, so that they did not know what they were saying, and that Peter therefore cried out: “It is good to be here; let us build dwellings!”

Peter possessed an excellent human understanding. What would bring his flesh and blood pain, he recognized from afar: “Lord, this shall never happen to you.”

And what, on the other hand, let him glimpse the blessedness of heaven, he understood well enough must be seized at once, in order to avoid all future suffering and struggle. Then he would be proven right after all. “This shall never happen to you.”

Therefore he said: “Rabbi, let us build dwellings here for you and Moses and Elijah.”

But no. The Lord had said: “Get behind me, Satan! You set your mind on the things of men,” because he could not avoid suffering if men were to be saved. The Transfiguration was not given in order to avoid the struggle, but in order to be strengthened for it—both for him and for the disciples.

Therefore a cloud came and overshadowed them, and a voice sounded: “This is my Son, the Beloved, in whom I am well pleased; listen to him!”

That was the matter.

Not to avoid suffering and struggle, but to carry within them light in darkness, victory in struggle, life in death—that is why he took the disciples up onto the Mount of Transfiguration, and why the Lord’s voice sounded: He—Jesus—the man who shall be rejected and suffer and die—he is my Son, the Son of God; look to him, listen to him! And there is no danger amid the deafening roar of storm and waves; he stretches out his hand, and each time you are rescued from the devouring deep.

“And suddenly, when they looked around, they saw no one anymore, but Jesus alone with them.”

Thus the blessed hour of the Transfiguration was over.

From Tabor Jesus went down and set his face and his steps unwaveringly toward Jerusalem and Golgotha.

But in the hardest hour of the struggle, when the bloody sweat flowed, this was his strength: “You are my Son!”

And in the darkest hour of suffering, when he cried out: “My God, why have you forsaken me!” this was his light and victory: “You are my Beloved.”

And thus he overcame death and the devil, loosed the bonds of hell, and rose victorious with a transfigured body to sit at the right hand of the Father.

Peter had to walk the same path from Tabor to Golgotha. In the darkest hour, when he denied his Savior; in the deep hour of humiliation and testing, when the Lord answered his unfaithfulness with the tenderest mercy: “Simon, son of Jonah, do you love me?”—what was his strength and his restoration if not this: “Jesus is the Son of God,” the voice he had heard on the holy mountain. And at Pentecost, before the council, on the cross in Rome—what was his light and victory if not this:

“We did not follow cleverly devised myths when we made known to you the power and coming of our Lord Jesus Christ, but we were eyewitnesses of his majesty; we heard the voice borne from heaven: ‘This is my Son.’” (2 Peter 1:16)

And so the apostle John testifies to the end: “We saw his glory, glory as of the only Son from the Father, full of grace and truth.”

So it is also with us. Every child of God has his Tabor—a blessed hour when, by the power of faith, he came to his Savior, saw the heavens opened and the Son of God shining like the sun at the Father’s side, and heard this blessed word from his mouth: “Your sins are forgiven,” and believed this blessed testimony from God’s mouth: “You are a child of God!” and learned to stammer “Abba!” and felt the whole blessedness of heaven flow through his heart. Oh, why did it last so briefly? Why could I not always remain on this mountain of blessedness?

\begin{center}
One is always willing, gladly,\\
to go with him up onto Tabor;\\
but unwilling, at times, to step down\\
into the Garden, where he lay,\\
full of anguish, fear, and distress,\\
cried out, prayed, and sweat blood,\\
in order to save me and others—\\
that path one is reluctant to walk.\\
\end{center}

And yet, every child of God also has his Golgotha, his labor and his measure of pain; there he must go.

Here Peter’s understanding avails nothing: “This shall never happen to you!” Nor is there room for the desire to linger in blessed feelings and, with Peter, build one’s dwelling on the mountain.

Down you must go—unceasingly down—into the dark valley of death’s shadow, until God in his time exalts you.

Here labor, struggle, and suffering are required; here you must serve as a soldier of Jesus Christ; here you must ‘fill up in your flesh what is lacking in Christ’s afflictions, for the sake of his body, which is the church’ (Colossians 1:24)—and rejoice in it.

Here it is not only a matter of working out your own salvation with fear and trembling, under painful trials and heavy struggles with flesh and world and Satan: ‘Not my will, but yours be done!’
It is also a matter of working and fighting and suffering like Peter and Paul and all the other witnesses, for Christ’s body, which is the church: to be light and salt and leaven in the world, in small matters and great; to be an example, so that God’s name may not be blasphemed; to encourage, lift up, and draw souls to the Lord; to seek out the poor and help the helpless; to proclaim the Lord’s acceptable year—all under contradiction, under mockery, suspicion, slander—all for his sake who endured the torments of hell for us; and to do so with praise and thanks and song, that it has been granted to us not only to believe in Christ, but also to suffer for his sake.

Oh, how heavy a path! How many fall beneath the cross! How many are covered with gloom and darkness and have forgotten the light from Tabor! How many built their own Tabor and wallowed in the pleasure of their own feelings in order to avoid the path to Golgotha.

But praise be to the Lord: many went courageously into the struggle and held out faithfully. Many kept their eyes unwaveringly fixed on the footprints of Jesus and went where he went and were comforted with his comfort. And when the weary soldier at times was allowed to sit down by the way and drink from the brook (Psalm 110:7), then the heavenly radiance of Tabor came as a new light in the soul and a new strength on the path: “I have the testimony of God’s Spirit in my spirit, that I am a child of God. Abba! Father!”

Friend, are you among these faithful laborers, these obedient soldiers? You long? You grow weary? See, it is drawing toward evening, and the day is declining. Soon your time of service is ended. Look there—Golgotha and the cross! Now soon he comes, God’s blessed Son, whom you saw so livingly by faith in your holy and blessed hour of new birth; now he comes soon with the same power by which God raised him, to lead you also up over Golgotha’s dark hill of death to the heavenly Tabor, where he has prepared a dwelling for you, eternal and imperishable, where you shall live and reign with him forever.

Oh, such blessedness is well worth a journey from Tabor to Golgotha!

\begin{center}
O Jesus, sweet,\\
My comfort in my need,\\
I pray to you in sorrow:\\
Help me, that until my death\\
I give my heart to you.\\
\end{center}
