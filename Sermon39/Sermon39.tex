

\section{First Sunday after Epiphany: Let the little children come to me}


\begin{quote}

Mark 10:13–16. And they brought little children to him, that he should touch them; but the disciples rebuked those who brought them. But when Jesus saw it, he was indignant and said to them: Let the little children come to me, and do not hinder them; for to such belongs the kingdom of God. Truly I say to you: whoever does not receive the kingdom of God as a little child shall never enter it. And he took them in his arms, laid his hands on them, and blessed them.

\end{quote}

\bigskip


Late one evening a man found a little child outside his house that had wandered off and could not find its way home by itself. The child could not tell where it belonged, but nevertheless bore a mark by which the man discovered where it belonged. But it was too late to bring it home that night.

What did he do with the child?

Perhaps he acted like the man the apostle James speaks of and said: “My dear child, go bravely out into the dark night; the Lord will surely lead you safely to your mother, if it is his will. Go on!”

Or perhaps he said: “I have no room in my house, and my wife is frail; I will set the child down at my neighbor’s door, ring the bell, and go my way. He can take care of the child; he has better means and more time than I.”

Or did he take the child to the police station and let the authorities see to bringing the child to its rightful mother?

No, this man did none of these things. He had read and learned a little about the merciful Samaritan. He took the weeping and helpless child into his house; his wife cared for it, spoke kindly to it, and wiped away its tears.

“Lie down now and sleep peacefully,” she said; “early tomorrow morning you shall come to your mother.”

And the child slept sweetly through the whole night and found its mother the next morning, who had almost been beside herself with anguish over the lost child.

Did the man not act rightly? And if he had acted otherwise, knowing to whom the child belonged, would he not have been worse than a wild beast?

Now then, father and mother, do you not also have a child — perhaps several — and do you not know to whom they belong? Are they not among those of whom Jesus says that “such belong to me”?

What then have you done to lead your child to Jesus?

I do not ask what you have done so that your child might fare well in the world.

Nor do I ask whether you have often wished: “If only my child might be saved!”

Let the frivolous flatter and deceive one another with their frivolity. I ask you in earnest: What have you done — to lead your child to him whom you know it belongs to, and who is more than a mother to it, yes — who has marked it and made it his own in the holy water of baptism; — what have you done, so that your child might come to Jesus? What have you done in the short time it was entrusted to you, from evening to morning — a little, short child‑life?

Oh, surely you have not hindered it? — It would be too dreadful even to think.

To hinder a poor little child from coming home again to its mother — its Jesus!

And yet the Lord himself says: “Do not hinder them!”

Then it must indeed be possible — to hinder.

Father and mother, could the terrible thing have happened, that you have hindered your own child — humanly speaking, your own flesh and blood — from coming back to its rightful Lord and Father, Jesus, who has acquired a right to it by his own precious blood?

Oh no, not hindered?

You lie down and rise up without Jesus in your thoughts or on your lips.
The child lies down and rises up, and Jesus becomes a stranger both to its thoughts and to its speech.

Is that to hinder?

At the midday table in a hotel, a son sitting beside his father was asked by the waiter what he would like to drink. “The same as my dad,” he said. The father looked at the waiter and then at the son. “Give me water,” he said.

For the first time in his life he truly felt the meaning and the heavy, incalculable responsibility of being an example.

Have you given your child a bad example? That is to hinder.

A harsh and unfeeling word, a heated and reckless act, an unforgiving or frivolous utterance or deed in your own house — where many often make use of a self-indulgent freedom — has often dug a grave of offense for your own child, so that it did not come to Jesus.

But even if you have not hindered your child — and who can lift the first stone? — you have still not come far in leading a child to Jesus.

“Let them come to me,” that is, work, strive, pray night and day that your child may reach where it belongs, that it may be saved.

Father and mother, what have you done?

Have you entrusted to your neighbors, or perhaps to society, the task of raising your child for heaven? There are Sunday schools and weekday schools where religion is taught; if someone sends his child there, is that all one is obligated to do for one’s child’s — immortal soul?

Oh, how heartless!

Was this what the Lord meant when he said: “Let the little children come to me” — that we should send them to religious school, and that be the end of it?

There you see a mother in the evening kneeling by her child’s bed. The small hands are folded and lifted upward. The little heart beats in blessed joy at the thought of heaven and Jesus, whom the mother has so often painted with the bright colors of faith, and to whom they now pray together.

That is what a mother has been granted grace to do.

The father takes the children in his arms, sings with them, tells of God’s wondrous deeds among his people, and lifts their gaze from what is passing to the imperishable above. The children ask and answer, so that one is often compelled to marvel. It is as though they were truly like Jesus in the temple, having come onto their own field, into their rightful house, when they speak of the “home” up there.

This is what every single father can do, in order to lead his child to Jesus.

And no one must say: “I cannot; I have no ability to teach children,” for then he must at the same time confess that he does not love Jesus.

It is not outward ability or lofty learning that matters.

There are two things that are required, in order to become leaders of one’s children to Jesus.

They are: first, to know and love the Savior oneself; second, to love and esteem the children.

Many do not esteem children higher than things or animals, which they would make Christian by coercion or beating.

Praise be to the Lord, that Jesus esteems both us and the children more highly, and loves us and draws us to himself with deep patience and mercy. Let everyone remember how the Lord has led him; in that way shall he lead his children.

“Let them come to me and do not hinder them!”

Others again do not esteem children as dearly entrusted treasures. They regard them more as a kind of personal possession, a sort of luxury and toy, from which they can draw purely human pleasure and satisfaction.

Rabbi Meir had two sons whom he loved above all earthly things, and a godly wife. One day, while he was in the synagogue, the two sons suddenly died. The mother had them carried up into the upper room, and with a compressed heart laid a cloth over them. The Lord made her strong.

When Meir came home, the wife asked whether one is not always obligated to return what has been entrusted. Meir looked sternly at his wife. “Can my wife ask such a thing?” he said.

Then she gently took him by the hand, led him upstairs, and showed him the two bodies.

“That was the dearest thing the Lord had entrusted to us,” she whispered; “now he has demanded them back.”

Tears ran quietly down the old rabbi’s cheeks; but he thanked God, because the children had come safely home.

For this reason children are entrusted to us — that we should lead them home to Jesus.

If this were always clear and alive before us, what children there would be! and what a new generation to bear God’s congregation and its message of salvation to the ends of the world!

“Let the little children come to me and do not hinder them!”


