% It is a living relation between God and the soul through Christ, sustained by Word and Sacrament.

% The word udvortes is problamatic. Mapping as:
%    Theological contrast → outward
%    Perceptual contrast → visible
%    Institutional contrast → external

% Problamatic use of fleshly vs carnal

\section{Second Sunday in Advent: The Kingdom of God Comes}

\begin{quote}

Luke 17:20–30
But when he was asked by the Pharisees when the Kingdom of God would come, he answered them and said: The Kingdom of God does not come in such a way that one can point to it. Neither will they say, See here, or see there! For behold, the Kingdom of God is within you. But he said to the disciples: The days will come when you will long to see one of the days of the Son of Man, and you will not see it. And they will say to you, See here, or see there! Do not go out, and do not follow them. For just as the lightning flashes from one end of heaven to the other, so will the Son of Man be in his day. But first he must suffer much and be rejected by this generation. And as it was in the days of Noah, so it will also be in the days of the Son of Man: they ate, they drank, they married and were given in marriage, until the day when Noah entered the ark, and the flood came and destroyed them all. Likewise also, as it was in the days of Lot: they ate, they drank, they bought, they sold, they planted, they built; but on the day when Lot went out from Sodom, fire and sulfur rained from heaven and destroyed them all; so it will be on the day when the Son of Man is revealed.
\end{quote}

\bigskip

Few things are as deeply misunderstood—or as thoroughly filled with alien, carnal content shaped by human sinful desire—as this word: the Kingdom of God.

The pope seized both spiritual and temporal power over all Europe and called it “the Kingdom of God.” The Mormons in our own days have founded a society in which outward power, the Word of God, and sensual desire are shamefully mingled together, and they do not hesitate to call it the Kingdom of God. The individual often pursues honor and power by means of the Word and the congregation, and yet seeks to persuade both himself and others that he does it for the sake of the Kingdom of God.

No wonder, then, that the Pharisees also had their concept of “the Kingdom of God.” By this they meant a kingdom at once spiritual and earthly, like that of David, only far greater in extent, power, and glory—a kingdom such as the papal church later attempted to realize: a world-kingdom in which the Jews would sit upon the thrones of dominion with all nations under their feet. And this kingdom, they thought, would come fully completed with the Messiah.

Naturally, then, Jesus—despite his blessed words of life and his divine miracles—was not much to the Pharisees’ liking, when in his simple, poor outward appearance there was nothing that promised them visible glory and power, but rather the opposite: mockery and affliction. If he were the Messiah, then they were disappointed in what they believed to be expectations grounded in the Word of God, concerning a king upon David’s throne; they therefore preferred to see in him a deceiver and a mocker of Israel’s rightful hope.

With such an earthly understanding of “the Kingdom of God,” they came to him and asked: “When does it come?”

He knew well that his own disciples, then as now, are only too inclined to share, to a greater or lesser extent, the Pharisees’ earthly expectations with regard to the manifestation of the Kingdom of God in the world; and he therefore directed his answer partly to the Pharisees, partly to his own disciples. We will consider them together.

In two respects, he says—and he speaks not only to them, but also to us—\textbf{you have misunderstood and overturned the concept of the Kingdom of God. First, you have forgotten that it is spiritual, and that its visible manifestation in the world is not in glory, but in lowliness. Secondly, that it indeed will one day also be revealed in outward majesty and glory, but that then it will not become dominion over the world, but judgment upon the world. This is to be considered.}

“The Kingdom of God,” he says, “does not come so that one can point to it.” The word that stands in the original text, and which cannot easily be rendered by an equivalent word, indicates that the Kingdom of God, in its proper meaning, does not have such visible boundary-marks with regard to extent or nature, that one could either with the bodily eye or with fleshly reason recognize them, point to them for others, and say: “here the Kingdom of God reaches, and no further; upon this visible thing the true Christianity depends, and upon nothing else.”

Just as little as circumcision, despite the Jews’ false trust in it, set any real boundary between those who belonged to the Kingdom of God and those who were outside, just as little can anyone rightly say: “all who live in a Christian country belong to the Kingdom of God,” or “all who belong to the congregation are in the Kingdom of God,” or “all who agree with us belong to the Kingdom of God.” For the Kingdom of God has no such visible or sensible boundaries; it is rather something within us, something spiritual: a living relation between each soul and the living God through the Savior, Jesus Christ—a relation which, by its very nature, stands opposed to all that is merely sensible and perishable. Therefore the Kingdom of God, in this sense, however hard it is for flesh and blood to acknowledge it, and however gladly we would evade it by making a compromise with the world, is nevertheless always in affliction, persecution, and lowliness. And insofar as the Kingdom of God, as it is on earth, must nevertheless also appear in an earthly form, with external organizations, buildings, and the like, all this is only the accidental and the transient, which, like the human body itself, will be brought to nothing; while that which is “within”—the power of the Spirit in the Word and the Sacraments, the life and fellowship in the Son and the Father—will only then shine forth in its full heavenly splendor, when all that is perishable and external has been stripped away.

But precisely on account of this opposition between the spiritual nature of the Kingdom of God and all that is earthly and perishable, on account of the afflictions and the unceasing struggle and persecution to which the Kingdom of God is exposed, there is also in all God’s children an unceasing sigh to see Christ again bodily and glorified, with majestic power in his hand to strike down the enemies and to establish his kingdom of glory here on earth.

“Guard yourselves against these voices and such temptations within you and around you,” the Lord says—and he says it especially to his disciples—“guard yourselves therefore; and when someone points to Christ here or there, then do not go out. For the Kingdom of God is always here below on earth something “within you,” not something that can be identified by outward power or glory. Guard yourselves therefore; for this is the leaven of the Pharisees.”

The Kingdom of God must always be in lowliness, and just as Christ himself first had to suffer much and be rejected, so must his body, which is the congregation—the Kingdom of God on earth—first suffer and be rejected; and then comes the glory, then comes Jesus, the carpenter’s son from Nazareth, in his full divine majesty and power.

“Yes,” he says to the Pharisees, “the Kingdom of God, such as you expect it, will indeed in its time also come—but woe to you when it comes.” The Lord does not delay, he says to his disciples, in coming; he comes in his time. But be ready when he comes! Remember Lot’s wife. For when Christ is revealed in his glory with the hosts of angels around him, then he comes in the clouds for judgment.

For judgment.

Swiftly like the lightning that flashes from one end of the earth to the other—and terrifyingly.

Unexpected and unforeseen he comes with his kingdom, when the great mass of humanity, as in the days of Noah and of Lot, has sunk into spiritual sleep and undisturbed enjoyment of all the goods of earthly life—when love has grown cold and faith has faded upon the earth—when there is crying, Peace, peace! and human beings rejoice and mirror themselves in the Kingdom of God which, with the mind of the Pharisees, they themselves have built by earthly power and strength; then the Son of Man stands there unawares like the lion over the sleeping one, and the humanity which just now was eating and drinking and delighting itself in its sensuality lies at once upon its knees under wailing lament and unspeakable terror, and hides its face so as not to see the Son of Man.

Thus, you Pharisees and you heedless disciples, who so gladly would remove the cross and serve both God and Mammon: the Kingdom of God comes, and it comes soon.

But are you also ready to receive it?

Have you remembered Lot’s wife?

