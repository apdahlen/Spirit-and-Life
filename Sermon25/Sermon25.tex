\section{New Year’s Day: And so this year as well}


\begin{quote}
Luke 13:6–9. And he spoke this parable: A man had a fig tree planted in his vineyard; and he came seeking fruit on it and found none. And he said to the vinedresser, Look, for three years now I have come seeking fruit on this fig tree and find none. Cut it down! Why should it also render the soil useless?

But he answered and said to him, Lord, let it stand yet this year also, until I dig around it and fertilize it— if it then bears fruit; but if not, you shall cut it down afterward.
\end{quote}

\bigskip


It is not difficult to see the connection between this parable and the two preceding statements of Jesus. Eighteen men had been killed at once when the tower in Siloam fell upon them, and Pilate had had some Galileans killed while they were offering sacrifice, so that their blood flowed together with the blood of the sacrifices. People commonly thought that God had thus punished these people because of grievous sins they had committed. But the Lord says:

Do you think that these people were sinners beyond all other Galileans, or all who live in Jerusalem?

\textbf{No, I tell you; but unless you repent, you will all likewise perish.}

This applies to every person: all must repent, and moreover each has been given a certain season of grace in which this must take place.

Thus the parable of the fig tree enters in, and it is especially to be pondered today, when you step into a new year—perhaps the final year of grace.

The Lord here, as in the parable of the sower, takes his image from daily life. A vineyard was generally a small garden from which a man lived much as from a farm among us; for the extraordinary fertility of the soil made it possible in Judea to reap great yield from what by our standards would be a very insignificant plot of land, half or a quarter acre. In return, the soil had to be cultivated with the utmost care and used to the utmost. And since daily wages were exceedingly low, a vineyard owner could afford to hire one or more laborers or vinedressers to keep the ground well cleared, fertilized, and productive. Where it could be done without harm to the vine cultivation, fruit trees were planted to make use of the soil. If they proved unfruitful, it was a double loss for the owner: first, time and labor wasted; second, precious soil rendered useless. And it lies in the nature of the matter that a gardener who works with fruit trees from their first planting onward binds a special affection to them, almost as to living beings. This can be observed among us as well. And thus the full meaning of the parable at once stood fairly clear before the hearers, and ought so to stand for us.

For, as the prophet says, “the vineyard of the Lord of hosts is the house of Israel, and the men of Judah are his delightful planting” (Isaiah 5:7).

The Lord’s vineyard is his dearly purchased congregation, and he comes today to find fruit in his vineyard.

You are the fig tree—whoever you are—who by baptism was grafted into Christ’s body and to this day has been warmed and watered by the sun and rain of his grace. To you—whoever you are who hear or read—comes today the holy and righteous God to find fruit in you.

He takes you aside alone and says: Come, let us set things right between us!

Is there anything I could have done to make you bear fruit that I have not done? Where, then, is your fruit?

There is only one fruit that the Lord requires of his planting: saved souls—no more nor less. For God has not placed his congregation and his children to help advance or improve the world, but that the world might be saved through them. God works by his Spirit and his Gospel through human beings, to bring peace to those far off and to those near, as it is written:

“How beautiful upon the mountains are the feet of those who bring good news, who proclaim peace, who bring glad tidings, who proclaim salvation, who say to Zion: Your God reigns.”

But in order to proclaim this glad message in such a way that souls are saved, one must first have experienced its blessedness in a true and living faith. Only thus does a person become able to proclaim his excellencies, who called us out of darkness into his marvelous light—that is, by word and life to draw souls to Jesus. For this reason God has placed us here; for this reason he has given us the heavenly soil of his congregation, with the dew and warmth of the Gospel; and for this reason he has the right to seek fruit on his fig tree.

But what do you have to answer? Look back upon the year that has passed. Early and again and again the Lord has called to you by his Word and said: “Soon I am coming to seek fruit.” “Not yet, not yet,” you have said, and allowed the world and its love to rule and dry out your heart.

Look upon him who hangs and bleeds on the cross for you; will you—do you dare—trample his blood under your feet?

Perhaps you have been frightened and awakened from your sleep of sin; perhaps you have wept over your condition and cried out: “Yes, yes, Lord, I will repent and bear fruit for you”—and then immediately forgotten it.

Has not the Lord also had his severe dealings with you—illness, sorrow, death, loss of what was dearest—and you have made promises to the Lord and not kept them?

Now the righteous Judge stands over you: “Two evils my people have committed: they have forsaken me, the fountain of living waters, and hewn out cisterns for themselves, broken cisterns that hold no water” (Jeremiah 2:13). “Therefore the kingdom of God will be taken away from you and given to a people who will bear its fruit.” And two evils you have committed against me: my Gospel you have despised, and your entrusted place in the congregation you have made useless.

Thus the verdict sounds: 

\textbf{Cut it down!}

\begin{quote}
Where shall I flee\\
From the thunder of the Law?\\
With my many sins,\\
Where shall I grasp consolation?\footnote{Hymn by Hans Adolf Brorson (1694--1764); the complete text is printed in the front matter.}
\end{quote}



But the eternal abyss yawns open beneath you; the anguished cry of the rich man sounds up from below: One drop of water! One drop of water! One minute, and you are lost forever.

Then a voice comes in between, full of unspeakable love and compassion: “Let it stand this year also! One more year, one more attempt—perhaps it may then bear fruit. Oh, how I will dig and fertilize it, if only it may yet stand one more year.”

And the Lord says: “Let it be—one more year!”

Who is it, then, who so narrowly escaped the dreadful torment of eternal judgment?

It is you, O friend, who are reading this—who have entered a new year and with shame perhaps must look back upon an entire life of sin and negligence; yes, who belong to the Lord’s congregation and up to this time have stood as a dry, unfruitful branch.

You have received yet a small season of grace in the new year. A mother perhaps has prayed for you and moistened the Lord’s feet with her tears for her lost child. A friend, a servant of the Lord, a congregation has pressed in upon God that he should spare you for Christ’s blood’s sake.

How will you regard this final call, the new year God in grace has granted you? The Lord will require a soul of you.

And you who have prevailed upon the Lord with your prayer, and who have promised to dig and fertilize—shall this be a new year for you? Will you in this year give yourself no rest or peace with the gifts and powers the Lord has given you, until the soul you prayed for is won for Jesus?

You to whom children are entrusted—immortal souls, bought with Jesus’ blood—shall this be a new year for you, so that you not only pray for your children, but also labor for them with reverence and inward love?

You servant of God, for whom the work of the congregation often becomes so manifold and overwhelming that you at times forget the individual souls, especially if they are healthy in body—shall this also be a new year for you, so that you sincerely strive to keep what you have promised, to dig and fertilize the unfruitful fig tree in your vineyard, not only by the general preaching of the Word, but by all the manifold means which true love is so inventive in using, to approach that soul gently and tenderly and to draw near with particular and personal care?

And you, the Lord’s planting, Christ’s congregation, where it must be confessed with shame that multitudes of fig trees are without fruit—you holy “remnant” of God in a ravaged garden—will you immediately and gladly offer yourself “to uproot and tear down,” cut down! cut down! or will you humbly and lovingly cry to your heavenly Father: Spare the tree one more year!—and then begin anew “to build and to plant,” with burning zeal and heavenly power, because the Lord has granted also you

“one more year!”
