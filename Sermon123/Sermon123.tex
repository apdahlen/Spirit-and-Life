\section{Third Sunday after Easter: Many Rooms in My Father’s House}


\begin{quote}

John 14:1–12: Let not your heart be troubled; believe in God, believe also in me. In my Father’s house there are many rooms. If it were not so, I would have told you. I go to prepare a place for you, and when I have gone and prepared a place for you, I come again and will take you to myself, that where I am, you also shall be. And where I go, you know, and the way you know." Thomas says to him: "Lord, we do not know where you go; how then can we know the way?" Jesus says to him: "I am the way and the truth and the life; no one comes to the Father except through me. If you had known me, you would also have known my Father; from now on you know him and have seen him." Philip says to him: "Lord, show us the Father, and it is enough for us." Jesus says to him: "So long a time have I been with you, and you have not known me, Philip? He who has seen me has seen the Father; how then do you say, ‘Show us the Father’? Do you not believe that I am in the Father and the Father is in me? The words that I speak to you I do not speak from myself; but the Father who abides in me, he does the works. Believe me that I am in the Father and the Father is in me; but if not, then believe for the sake of the works themselves. Truly, truly I say to you: he who believes in me shall also do the works that I do, and greater than these shall he do, because I go to my Father.


\end{quote}

\bigskip


Our text belongs to Jesus’ farewell discourses to the disciples; he prepares them for his departure through the death of the cross to the Father.

His departure struck the disciples like a dreadful blow. We cannot comprehend the sorrow it caused. Of all that bears the name of love on earth, there has never been anything truer or more intimate than the disciples’ love for Jesus. Drawn by him, they had let go of all else; him they had followed; by his word and works they had been lifted up into an entirely new life, of which they had never before had any foreboding; called by him, they had become partakers in a work and a ministry that filled them with the most blessed joy, with the most glorious expectations. They were sent out by him, and they returned to him; in his look they saw approval or rejection of their works, words, and thoughts; by his word their hearts were filled with joy and their souls with new life.

Because no life of love had ever been more blessed, no separation had ever been heavier or more solemn than the one now before them. The beloved Master, the faithful Friend, was to suffer the death of a criminal upon the cross; the little band of disciples was to be left utterly forsaken and alone in the world.

They needed comfort—strong and blessed consolation—the kind only the Lord could give. And richly and abundantly he gives the consolation in these farewell discourses which we read in the Gospel of John in chapters 14–17.

The first great consolation which Jesus gives the disciples, and which he gives us, is this:

"In my Father’s house there are many rooms."

The mighty Savior, the heavenly Friend, points upward to the eternal dwellings prepared for those who have left all for his sake and have no abiding place here on earth. He himself, by his wondrous call, has torn them loose from the earthly home and broken the bonds that bound them. Now in the hour of separation, he would keep the eyes of their hearts fixed on the home above, that they may not sink back into the bondage of corruption: "In my Father’s house there are many rooms."

Praised and blessed be the Lord for this mighty word of consolation. You lonely, forsaken soul, who have become a stranger in the world because you followed the calling voice of Jesus and have lost all in the world for his sake — there is room for you in the home above, room enough; room for all who love the Lord Jesus.

Let me then let go all in the world and love him who loved me; if I have no longer any home here, there above is the Father’s house with many rooms.

"I go to prepare a place for you."

This is the second great ground of consolation for the Christian soul. He who is Son in the house has himself gone home to prepare a place for his friends. The Son comes home, blood-sprinkled from the battle with sin and with the power of death and of hell. Then the doors are opened, then lock and bar spring open, then the reconciled Father embraces the Reconciler, the Son of Man, our Brother and our Friend. The wrath is gone, for sin is atoned, and our flesh and blood is set at the Father’s right hand.

He who has the key of David, and who opens and no one shuts, he himself opens the door to the Father’s house for the company of friends upon earth, for every poor and needy sinner who in faith flees to him.

Friend, a place is prepared for you in the Father’s house. The door stands open; will you not go in? Why do you wander homeless in the desert of this world, where yet the thorns pierce and the heart suffers? Look upward to the blessed home that is opened for you. Why do you sit solitary in a foreign land with tear-wet eye and stare toward the earth, while the soul hungers and thirsts and cannot be satisfied with the swine’s food? O come to yourself and say in your heart: "I will arise and go to my Father."

See, the home beckons; see, the Father is ready to receive you. O come, come and find your place prepared for you.

Why does your gaze sink so dark and heavy toward the earth? Why will you not lift it freely and boldly toward the light? O I know what you lack: sin, sin, sin. You dare not, you still shrink back; you know that you are unclean from the sole of the foot to the crown of the head; then you draw your rags together about you and sink down in bitter despair and shame: "I cannot, I dare not show myself there in the blessed home of light, a wretched sinner as I am." Then it is worse than before. Heaven stands open. Blessedness beckons. But I do not belong there; for me it is impossible to take a place in the kingdom of light; for me the darkness is a hiding place for my misery.

Yet there is still a word of consolation:

"I come again and will take you to myself, that where I am, you also shall be."

Jesus will not leave any sinful and sorrowing soul behind in despair. He who has prepared heaven for us will also prepare us for heaven. He has made our cause clear before the Father; he has opened heaven for us; then he comes again and speaks kindly to the soul about the reconciled Father’s love. Now he will help you home, you poor, wandering child. You are so unclean and sinful, you dare not come along. But see, he bends down over your grief-worn head, and from his blood-sprinkled brow there flows a drop of the sacrificial blood that was shed for you; it is the forgiveness of sins, it is cleansing from all your uncleanness. "The blood of Jesus Christ, God’s Son, cleanses from all sin." It runs like fire through your bones, and for the first time, you dare to lift your eye boldly toward heaven. Blessed sight; it is the Father who stretches out his arms toward you; it is the Son who stands at your side, ready to lead you home to blessed meeting with the Father: "I will take you to myself."

You feel so weary; you cannot walk to the home. Jesus has strong arms; he himself has said that he will carry the strayed lambs back to the sheepfold; fear not, he does not grow weary, he does not fail; if he has found you, he shall lay you upon his shoulders with joy, and you poor lamb shall come safely home. Safely home — yes — that is the point; safely home, for Jesus has prepared the home, Jesus has opened the home, Jesus himself comes and brings you home.

Blessed is the life of Christians when Jesus goes at their side and leads them so safely on the way; but their death is more blessed still, when they go with him from a foreign land to the Father’s house:

% https://hymnary.org/text/jeg_ved_mig_en_sovn_i_jesu_navn#Author

\begin{quote}
Then the door is opened to heaven’s city,
There the redeemed are called by name.
God grant that we all may meet in gladness,
And none of ours be missing!
May God grant this to us for Christ’s blood,
That we may find harbor in heaven.\footnote{Magnus Brostrup Landstad \textit{Jeg veed mig en Søvn i Jesu Navn}, 6th verses}
\end{quote}

\pagebreak

\textbf{I Know a Sleep in Jesus’ Name}

By Magnus Brostrup Landstad (1802—1880) 
Original Danish (Landstad, 1861)
\begin{comment}
\begin{quote}
1) Jeg veed mig en Søvn i Jesu Navn,
Den kvæger de trætte Lemmer,
Der redes en Seng i Jordens Favn,
Saa moderlig hun mig gjemmer,
Min Sjæl er hos Gud i Himmerig,
Og Sorgerne sine glemmer.

2) Jeg veed mig en Aften-Time god,
Og længes vel somme Tider,
Naar jeg er af Reisen træt og mod,
Og Dagen saa tungsom skrider:
Jeg vilde til Sengs saa gjerne gaa,
Og sovne ind sødt omsider.

3) Jeg veed mig en Morgen lys og skjøn,
Der synges i Livsens Lunde,
Da kommer han, Guds velsigned' Søn,
Med lystelig' Ord i Munde,
Da vækker han os af Søvne op
Alt udi saa sæle Stunde.

4) Jeg haver den Morgen mig saa kjær,
Og drager den tidt til Minde,
Da synge jeg maa, og se den nær,
Den Sol, som strør guld paa Tinde,
Som Smaafuglen ud mod Morgenstund
Op under de høie Linde.

5) Da træder Guds Søn til Gravens Hus,
Hans Røst i al Verden høres,
Da brydes alt Stængsel ned i Grus,
De dybe Havsgrunde røres,
Han raaber: Du Døde, kom herud!
Og frem vi forklaret føres.

6) Da aabnes den Dør til Himlens Stad,
Der nævnes de Kaarnes Navne.
Gud lade os alle mødes glad,
Og ingen af Vore savne!
Det unde os Gud for Kristi Blod,
Vi maatte i Himlen havne!

7) O Jesus, træd du min Dødsseng til,
Ræk Haanden med Miskund over,
Og sig: Denne Dreng, den Pigelil
Hun er ikke død men sover!
Og slip mig ei før, at op jeg staar,
I Levendes Land dig lover!
\end{quote}
\end{comment}
Translation (2026)

\begin{quote}
1) I know a sleep in Jesus’ name,
It soothes my weary frame;
A bed is made in earth’s embrace,
She keeps me in her gentle care;
My soul with God in heaven shall be,
And all its sorrows flee.

2) I know a blessed evening hour,
I long for it at times,
When I am weary from the road
And heavy moves the day;
I gladly would lie down to rest
And sweetly fall asleep.

3) I know a morning bright and fair,
Where songs of life resound;
Then comes the blessed Son of God
With joyful words upon his lips;
He wakes us from our sleep
In that most happy hour.

4) That morning is so dear to me,
I often call it to mind;
Then I shall sing and see draw near
The sun that gilds the heights,
Like a small bird at break of day
Beneath the towering lindens.

5) Then God’s own Son draws near the grave,
His voice is heard through all the world;
All bars are shattered into dust,
The ocean’s depths are stirred;
He cries, “You dead, come forth!”
And we arise in glory.

6) Then opens wide the gate of heaven,
The chosen are called by name;
God grant that we may meet in joy
And none of ours be missing;
May God grant this to us for Christ’s blood:
That we may find harbor in heaven.

7) O Jesus, come to my deathbed,
Stretch out your hand in mercy,
And say: “This boy, this little maid,
She is not dead, but sleeping.”
Do not let go until I rise
And praise you in the land of life.

\end{quote}


