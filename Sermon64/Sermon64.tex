

\section{Septuagesima Sunday: Talents and Interest}


\begin{quote}

Matthew 25:14–30. A man who was about to travel abroad called his servants and entrusted his property to them; and to one he gave five talents, to another two, and to another one, each according to his ability, and immediately departed. Then the one who had received the five talents went away and traded with them and gained five more talents. In the same way, the one who had received the two gained two more. But the one who had received the one went away and dug in the ground and hid his master’s money. After a long time the master of those servants came and settled accounts with them. And the one who had received the five talents came forward and brought five more talents, saying, “Master, you entrusted me with five talents; see, I have gained five more talents.” His master said to him, “Well done, good and faithful servant. You have been faithful over little; I will set you over much. Enter into the joy of your master.” And the one who had received the two talents also came forward and said, “Master, you entrusted me with two talents; see, I have gained two more talents.” His master said to him, “Well done, good and faithful servant. You have been faithful over little; I will set you over much. Enter into the joy of your master.” But the one who had received the one talent also came forward and said, “Master, I knew you to be a hard man, reaping where you did not sow and gathering where you did not scatter; and I was afraid, and I went away and hid your talent in the ground. See, here you have what is yours.” But his master answered him and said, “You wicked and lazy servant! You knew that I reap where I did not sow and gather where I did not scatter; then you ought to have entrusted my money to the bankers, and at my coming I would have received what was mine with interest. Therefore take the talent from him and give it to the one who has the ten talents. For to everyone who has, more shall be given, and he will have abundance; but from the one who has not, even what he has shall be taken away. And cast the worthless servant into the outer darkness. There will be weeping and gnashing of teeth.”
\end{quote}

\bigskip

In the Kingdom of Heaven, there is inquiry both after talents and after interest. The one who has no talent is outside the Kingdom of Heaven; for in the Kingdom of Heaven all have received talents, “each according to his ability.” But the one who does not produce return from his talents does not enter into his master’s joy; he is not saved. Still more, the worthless servant is cast out into the outer darkness, where there is weeping and gnashing of teeth.

Therefore each person should first ask himself whether he has received any talent at all. This is the first thing that matters; for the one who wishes to begin with the interest before he has received the talents goes entirely astray. You must receive your talent first; then you can begin with the interest.

This seems simple enough. Nevertheless, many go wrong in this simple matter. They think that they must begin by giving God something before they have yet received anything from God. But this is in vain; it is of no use. God must give you talents first; then comes the time to trade and to gain interest.

Therefore test yourself: Do you have any talent? Have you had an encounter with God in which he gave you something with which to trade? And what is your talent? How great is it? What grace-gift have you received, and in what direction does it lie?

For here we are not speaking of the gifts and abilities you received at your natural birth; here we are speaking of the grace-gift you received when you became a participant in the Kingdom of Heaven. Natural endowment is also God’s gift, and you shall answer for it; but you cannot answer God with interest from your natural endowment unless you have received it sanctified and transformed by grace. Therefore in our text it is the natural endowment that is meant when it is said that the servants received talents “each according to his ability”; this “ability” is the natural endowment with which every human being is equipped by God, and which is sanctified by grace-gifts and taken into God’s service by the Holy Spirit.

You therefore have ability by nature; but do you also have talent by grace? With your natural ability alone you are still outside the Kingdom of Heaven; but if you have also received talent by grace, then you are inside. You cannot, with the most brilliant abilities, produce interest for the Lord; he himself must give you the capital with which you are to trade.

Yes, you say, if it is so that all who enter the Kingdom of Heaven also receive their talent, then I too must have a talent; for I was received into the Kingdom of Heaven as a little child through holy baptism. And you speak truly. But, dear brother or sister, where is now this your talent? Have you begun to trade with it, or did you long, long ago bury it in the ground?

If you must confess that you have not yet thought about your talent—what it is or how you are to gain interest from it—then you are in a dreadful condition. You have received talents from the Lord, and you must answer for their return; but you do not even know where the talents are, let alone the interest. You careless and thoughtless servant, what will you do? The time has already advanced far, and you have not begun to trade; and the capital has been buried, and the interest is not even to be thought of as obtainable. What will you do if you are now immediately called to account?

If this is your condition, then hurry to the Lord, that you may be renewed in grace, that with him you may receive your talent renewed, so that you can begin to trade. Yes, hurry; for it takes time to gain interest, and you do not know how much time you have left.

I know well that one or another will answer and say: “But the thief on the cross was saved, and he had no time to gain interest.” But you are mistaken. He used the time—the short time, the hour of death—so well that his words have become a sermon for thousands and for millions. It will surely be shown that he gained interest. But you—are you in his situation? Have you, as was the case with him, not encountered Jesus until the hour of death? Or do you have, as he had, the opportunity to become a witness for all peoples and generations and tongues? Friend, do not kid yourself! You cannot, according to your will or your calculation, become like the thief on the cross.

No—ask yourself immediately, today: Do I have any talent, and how do I use it? And if you do not yet have any talent, then hurry to repent and turn to the Lord, and he who gives generously and without reproach shall give you the talent with which you can trade.

For the other great question in the matter of salvation is the interest. The talent comes from God. From it, you must bring forth a return.

If you have become a participant in the Kingdom of Heaven, then there are talents enough. You have received the Word and grace, the Son and the Spirit, reconciliation and life, forgiveness of sins and love. What do you now do with this? You have also received your particular grace-gift. What do you accomplish with it? What do you use it for?

There is an abundance of spiritual gifts in the congregation—but how few of them are used! How then shall it go with us in the Lord’s judgment? We content ourselves with going into the congregation, calling the pastor, listening to the sermon, receiving the sacraments, paying the pastor’s salary; but for many, that is also the end of it. Do you think that you are trading with your talent and gaining interest in this way? Up, up to the Lord’s work! Is there not enough of it?

Is there no unconverted soul in your circle of acquaintance? Have you spoken to the Lord about him or her, and have you spoken to him or her about the Lord? Are there no sick to visit, no distress to relieve, no tears to wipe away? Are there not little children who are to be brought to Jesus? Are there not millions of heathen—of the nations—to whom the gospel is to be brought? Have you done it? Does God’s living love drive you to pray and to work?

Oh, if only you would think about the interest! \textbf{The Lord will demand it, and if you do not have it, then you are a worthless servant who is cast out into the outer darkness.} If your Christianity became for you merely like the miser’s useless treasures—a matter you kept to yourself for your own enjoyment, your own advantage, your own honor—then you are without interest because you are without love.

Live in God’s love and love the brothers as God loved you, and there will be interest from your pound.

But, you say, I am not able; in many directions it is of no use for me to work. My position is not such that I can go out and preach to the heathen, for example; can I then not be saved? Yes—then the word about the bankers comes into its proper place. “Therefore,” it says in verse 27, “you ought to have entrusted my money to the bankers.” Our missionaries are our bankers. Mission is God’s bank, where your gift can be put to work. What do you do for the sending of missionaries? Do you work for heathen mission and Jewish mission? There is much to do for that cause, even if you cannot yourself go out and preach. 

Have you truly done all that God’s Spirit urged you to do?

If, then, we have received talents, let us trade. Let every member of the congregation take hold of the work for the Kingdom of God and not leave it to the pastors alone. Only in this way will there be powerful awakening; only in this way will there be vigorous congregational life. Only in this way will proper provision be made for the education of pastors, for the sending of missionaries. Only in this way will there be interest from God’s precious talents. And only in this way shall we hear that blessed word: “Well done, good and faithful servant. You have been faithful over little; I will set you over much. 

Enter into the joy of your master.”

