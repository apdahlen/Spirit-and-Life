

\section{Sunday Sexagesima: The Harvest and the Laborers}


\begin{quote}

Matthew 9:36–38; 10:1–7. And when he saw the crowds, he was moved with deep compassion for them; for they were fainting and scattered, like sheep that have no shepherd. Then he said to his disciples: The harvest indeed is great, but the laborers are few. Pray therefore the Lord of the harvest, that he would drive out laborers into his harvest! And he called his twelve disciples to himself and gave them authority over unclean spirits, to cast them out, and to heal every disease and every infirmity. And these are the names of the twelve apostles: first, Simon, who is called Peter, and Andrew his brother; James the son of Zebedee, and John his brother; Philip and Bartholomew; Thomas and Matthew the tax collector; James the son of Alphaeus, and Lebbaeus, surnamed Thaddaeus; Simon the Cananaean, and Judas Iscariot, who also betrayed him. These twelve Jesus sent out and commanded them, saying: Go not into the way of the Gentiles, and enter not into any city of the Samaritans; but go rather to the lost sheep of the house of Israel. And as you go, preach and say: The kingdom of heaven has drawn near.

\end{quote}

\bigskip



Is it not now as it was then? Are God’s people still fainting today, “stricken”—as it should literally be translated—stripped bare and abused? The congregation that is called by Christ’s name, the new Israel, is it not also among us in a lamentable condition, according to the well-known hymn verse:

\begin{center}
The land and shore with baptized teem,
But where is faith’s bright flame?
\end{center}

Thousands of enemies lie in wait for the poor souls in this land. Mammon stands in every street and at every crossroads and beckons like an angel of light and says: “Worship me, and I will give you enough and abundance of all the good things of the earth!” And in droves people stream into his seductive net and are robbed of the hope of eternal life. In vain the Lord cries out to the wandering crowds who chase after the goods of this world:

“Why do you weigh out money for that which is not bread, and your labor for that which does not satisfy? Listen diligently to me! Then you shall eat what is good, and your soul shall delight itself in fatness. Incline your ear and come to me! Listen, and your soul shall live” (Isaiah 55:2–3).

Many who in Norway, in earthly poverty, worshiped the living God and feared him, have here exchanged their fear of God for earthly pleasures; song and prayer have fallen silent in their families, and the book of the Word of Life is hidden away on a shelf or displayed in a fine binding upon a table—for never to be opened.

The saloon first, unbelief afterward, entice the young on every corner; light-minded and thoughtless, they cast away the faith of childhood, and with laughter and mockery they compete to outdo each other on the road to ruin. At home sits an old father or mother, white-haired and forsaken, crying and screaming to God for their children and weeping, because they themselves once failed, even with heartfelt love, to draw them to Christ. And when the years have passed, these same smiling young men and women are found again plundered, beaten, half-dead by the roadside! For “my people have committed two evils: they have forsaken me, the fountain of living waters, to hew out for themselves cisterns, broken cisterns that hold no water.” Oh, “as a woman is faithless to her companion, so you have been faithless to me, O house of Israel, says the Lord.”

“Is there then no balm in Gilead, is there no physician there?” Praise be to the Lord! If the condition among us in spiritual respect is as sorrowful as it was then, yet the blessed Savior still stands among us. If our people, as it is said of Israel, are truly “scattered,” torn out of their fathers’ land and cast about on these wide prairies, where the bread of life in many places is only sparingly distributed to the hungry and fainting, yet the Lord still stands in his congregation with his Word through his servants, and he is “moved with deep compassion” for his people.

Shall he call us in vain to his great harvest, where the laborers are still so few?

Friends, priests of the Lord, all children of God, open your ears and listen!

“A voice is heard on the bare heights, the weeping and pleading of the children of Israel; for they have perverted their way, forgotten the Lord their God.” Shall they cry for help in vain?

Both priests and congregations have many weighty matters to attend to in this land. All are for the building up of the kingdom—both inwardly and outwardly. But “one thing is necessary.” The salvation of a soul is more precious than all the treasures of the world. Oh, then let us once again unite in prayer that the power of God’s Spirit may be poured out upon us, that we may be filled with new, heartfelt love for souls and burning zeal for their salvation, so that we might lay aside all other things and go out in Jesus’ name, as he sent his apostles, to “seek the lost sheep of the house of Israel.” See the dogs, see the mutilation, see the evil workers! See the power and lust of the world that rules over the great crowds, strips them and makes them miserable—and begin to cry out with full voice: “Land, land, land! Hear the word of the Lord! Return, you apostate children, I will heal your apostasy, says the Lord.”

Oh, that God’s servants especially might more earnestly apply themselves not to be satisfied with the proclamation of the Word from the pulpit alone, but to go to the plain and unlearned, not only when they are sick and dying and send for them, often in fleshly fear and superstition, but to the healthy, strong, secure sinners, spiritually lame, crippled, and leprous, at crossroads and street corners, and compel them to come in. That there might come a new time, a time of awakening and of spring, when the turtledove once more lets its song be heard, and that from the prairies and forests of this land, among our people from one end to the other, there might come, like a mighty wind of Pentecost, this answer of the prophet: “See, we come to you; for you are the Lord our God!”

But if this is to happen, then it is also necessary “to pray the Lord of the harvest to send forth more laborers for his harvest,” men of good testimony and full of the Holy Spirit and wisdom (Acts 6:3), earnest, zealous, God-fearing pastors. Of these there is always need of more and more. Many do not seriously consider what spiritual distress there is in many places among us, so that souls faint for lack of the proclamation of God’s Word. We are inclined to settle down and thank God because we have it so well, and then let the others try to help themselves. Oh, if God had dealt with us in such a way!

When we consider that whole settlements for years are without the ministry of God’s Word by God-fearing pastors—and that many of God’s servants must nearly work themselves to death in order every three or four weeks to reach their many congregations with the Word and the Sacraments, then surely many of us should awaken and remember the poor souls who are neglected, and the Lord’s admonition: “Pray the Lord of the harvest, that he would send forth laborers for his harvest.”

Whoever truly prays will also work. That is what is needed. Therefore, brothers and sisters, let us not grow weary in doing good; let us awaken and with renewed strength take hold of the work and become zealous for the instruction of children in the fear of God, and for the establishment and preservation of schools for the training of living men, filled with and driven by God’s Spirit, to go out among the fainting crowds and cry:

“Return, you apostate children, I will heal your apostasy, says the Lord.”

See him, the blessed Savior, see him as he is “moved with deep compassion over the people,” see him hanging on the cross for you—compassion, not in words, but in deed; and know that as he has done for you, as he has prayed, labored, and suffered, so he wills that you should do for the “fainting and scattered sheep” whom he has redeemed with his blood. Will you?

