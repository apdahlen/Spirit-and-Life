
\section{Second Sunday after Epiphany: Zacchaeus}

\begin{quote}

Luke 19:1–10. And he entered and passed through Jericho. And behold, there was a man called Zacchaeus, and he was a chief tax collector and was rich. And he sought to see who Jesus was, but could not because of the crowd, for he was short. And he ran ahead and climbed up into a sycamore tree to see him, for he was about to pass that way. And when Jesus came to the place, he looked up and noticed him and said to him: Zacchaeus, hurry and come down; for today it is necessary for me to stay at your house. And he hurried and came down and received him with joy. And when they saw it, they all murmured, saying: He has gone in to lodge with a sinful man. But Zacchaeus stood up and said to the Lord: Behold, Lord, half of my goods I give to the poor; and if I have defrauded anyone by deceit, I restore it fourfold. And Jesus said to him: Today salvation has come to this house, since he also is a son of Abraham; for the Son of Man has come to seek and to save the lost.

\end{quote}

\bigskip


“It is easier for a camel to go through the eye of a needle than for a rich man to enter the kingdom of God,” it is written; and of Zacchaeus it is written: “And he was a chief tax collector and was rich.”

It was not easy for Zacchaeus. Everything seemed to stand in the way of his salvation. In every respect he appeared to be far from the kingdom of God, and all doors seemed shut.

He was a tax collector; and these despised men, who had taken service with the Romans—how could they hope to share in Israel’s promised inheritance? Had not the tax collectors done as Esau and sold their birthright for a dish of food? In Roman gold and bread they sucked and cheated their countrymen and grew rich by plundering the poor Israelites. Truly, they had forfeited all right to Israel’s glorious promise, to the kingdom of God which the Messiah brought. If Jesus was truly the Messiah, then there was little prospect of help for Zacchaeus with him.

Zacchaeus was a man of high standing. His high office was a new hindrance to him. “Not many mighty, not many noble are called,” it is written. People of high rank and position are often too great and too proud to inquire after Jesus of Nazareth. Often they are also too busy to set aside time to seek the Savior of the soul. And here in Jericho the coming of Jesus was accompanied by such a commotion among the “common people,” such a crowd gathered, that it was utterly beneath Zacchaeus’s dignity to have anything to do with the matter. Had Jesus been a refined and noble philosopher, it would have been another thing; then the crowd would have stayed at home, and Zacchaeus could have had him to himself as an honored guest. But out on the street, in the midst of the crowd, Zacchaeus had to seek him if he would find him.

Zacchaeus was rich; and his wealth had been acquired in a bad and ignoble way, as is often the case. This was almost the greatest hindrance. What had the rich man to gain from the poor one? What more was there for Zacchaeus to desire than money, the “almighty” money?

But Zacchaeus was not satisfied or content. There burned a fire within him that gave him no rest. He had to, he had to see Jesus. Let it cost what it would, and let it be the same whether it helped or did not help; he would see him in spite of everything.

And Zacchaeus did see Jesus; but it was a hard struggle for him. The little eager man cared nothing for mockery and scornful words; the distinguished tax collector cared nothing for the pressure and shouting and laughter of the crowd; the rich man found no satisfaction in his riches. He tried to force his way through the crowd to Jesus, but could not; he ran ahead, climbed up into a sycamore tree, and there he sat, anxiously watching for Jesus who was to pass by.

And Jesus passed by and saw Zacchaeus. Just as Zacchaeus had thought of Jesus, so Jesus had also thought of him. Or how could anyone seek to see Jesus, and Jesus not seek to meet the seeker? Can longing burn in a human heart for the Savior, and his longing of love not be kindled? Oh no; he seeks those who seek him. He will meet those who inquire after him.

\textbf{And therefore there is a blessed meeting. That Zacchaeus sees Jesus is not yet enough; but that Jesus also sees Zacchaeus and enters his house—that is the main thing. And who can describe or explain the joy that is in such a meeting with Jesus? It is impossible, because human language can scarcely express the human feelings that move a heart; but it is powerless before the divine movements that shake a person’s inmost being when a new heart is created and a new spirit within him. Small and poor images of it may be found in the joy that earthly love can produce in a human soul; but the joy of finding Jesus and tasting his love, of belonging to him and being born again, is not only infinitely greater, but also of an entirely different kind and nature. As heaven is higher than the earth, so Jesus’ love is higher than earthly love, and the joy in him higher than all the joy of the earth.}

Therefore it is not strange that Zacchaeus “hurried and came down and received him with joy.” Like a fresh living spring for the thirsty, so were Jesus’ words and love for Zacchaeus’s soul. He was born again; he was a son of Abraham.

But people did not know it; only Jesus’ eye saw what was taking place in the tax collector’s heart. Therefore the onlookers murmured and said: “He has gone in to lodge with a sinful man.”

Yes indeed, “a sinful man”; and blessed and praised be the Lord for it. He has come to “seek and to save the lost.” Here he has found a lost sinner. In this lies all our comfort; on this we build all our hope: the Son of Man lodges with “a sinful man.” And yet no longer “a sinful man.” Now that he has found Jesus, and Jesus has found him, now the great change has taken place; now he is converted, now he is born again and has become a son of Abraham, an heir of the promise, with free access to the kingdom of God.

For the meeting with Jesus has opened his heart, so that he says: “Behold, Lord, half of my goods I give to the poor; and if I have defrauded anyone by deceit, I restore it fourfold.”

This is the language of repentance and love. He does not deny that he is a sinful man; he confesses that there is unjust property in his hands; but now, since Jesus has entered his house, he will hasten to be rid of it. Unrighteous mammon and Jesus, the friend and Savior of sinners, do not belong together. And not only will he give up all unjust property, but he will restore it fourfold and thus cause joy where he previously caused sorrow and bitter pain. Truly he is a converted man, a son of Abraham through faith. The friend of the poor, the helper of the oppressed has he become—he, the tax collector, the deceiver, the bloodsucker.

“The sinful man” is no longer a sinful man.

What do you think now? If this is how a sinner is saved, how is it then with you and me? How is it with our rich men, who have gathered riches unjustly and who perhaps for the sake of their riches are highly esteemed in the congregations? Where is the poor man’s half, or has anyone heard of the one who was defrauded receiving fourfold back from the deceiver? Alas, there are surely not many like Zacchaeus. We so gladly comfort ourselves with the thought that Jesus is “the friend of tax collectors and sinners”; we so gladly rejoice that he “lodges with a sinful man.” But what does it help if Jesus’ friendship does not make us new people, with the love of God poured out in our hearts?

Let us then awaken in earnest and seek Jesus until we find him and he finds us, so that there may be new birth, a new heart, a new mind, and a new way of life, so that we too may become children of Abraham and heirs according to the promise. Behold, the Son of Man has come to seek and to save the lost!

Friend, has he also found and saved you?
