

\section{Fifth Sunday after Epiphany: The Seed that Grows}


\begin{quote}

Mark 4:26–29. And he said: This is how it is with the kingdom of God: when a man casts seed into the ground and sleeps and rises night and day, and the seed sprouts and grows, though he himself does not know how. For the earth bears fruit of itself, first the blade, then the ear, then the full grain in the ear. But when the fruit is ripe, he immediately swings the sickle, because the harvest is at hand.

\end{quote}

\bigskip


There are three things this parable presses upon the heart concerning the kingdom of God:

1. That the seed, that is, the life of God in us, grows and must grow;
2. That the power for this growth is not our own power; and
3. That this growth has a conclusion and a final goal.

These are truths that touch the very foundation of the life of God, and without whose inward and daily recognition and embrace life, both in the individual and in the congregation, will wither and die. For just as it is said: “Whoever is not with me is against me, and whoever does not gather scatters,” so it is also true that what does not grow withers, and the one who does not go forward in grace and faith goes backward and falls away.

The seed of which the Lord speaks is life through the Word of God. “You have been born again,” says Peter, “not of perishable seed but of imperishable, through the living and abiding Word of God.” Faith is therefore not mere intellectual assent that leaves the heart untouched, nor is it a moral striving to live outwardly respectable and devout; faith is a heavenly seed, a new life created by the Spirit of God in a heart hungry for salvation, as the Lord himself says: “Unless you are born again, you can neither see the kingdom of God nor enter the kingdom of God; for what is born of the flesh is flesh, and only what is born of the Spirit is spirit,” and: “As many as believe in him, to them he gave the right to become children of God, who were born not of blood, nor of the will of the flesh, nor of the will of man, but of God.”

If the grain of wheat that is laid in the ground does not after some time come up, it is because it was dead, without germinating power. But if it is a living seed, then it will indeed lie some days unseen in the ground, yet it is at work there, drawing nourishment from the soil, and soon pushes up out of the earth, despite all hindrances, as a fresh green shoot which, though not always quickly, gradually brings forth blade, ear, and full grain, and promises the sower fruit and joy.

So it is with the seed of faith in a human heart. It may indeed for a time seem hidden from the eyes of other people, but not forever; otherwise it is dead and useless. If it is of the right kind—if it is born of God—then in its time, whether you understand it or not, it goes to work within the hidden depths of a person, sanctifies them, permeates them with its divine life, and finally, and not too long thereafter, breaks forth, despite the resistance of the flesh and the world, as a fair and fresh planting of God, so that all may see and know that something new and divine has taken place here, a great miracle of God. The one who stole, the one who drank, the one who sought all his joy in the world and mocked God and his gospel does so no longer; the one who was proud, self-centered, hateful, slanderous, greedy, is so no more, but loves people and praises God in word and deed. Is it so with you, O soul?

And yet it is not enough. Sun and drought may come, and the shoot wither, and that is worse than if it had never been planted. If the plant does not set leaves and ears, it cannot bear fully ripe fruit.

Thus the life of God in human hearts must grow and go forward, if it is not to wither and die. Here the Word of God contains so many earnest exhortations, not only to take up the whole armor of God, to fight and strive against the enemies, to renounce and suffer with Christ, but above all to be filled with the love of Christ and to practice it more and more, so that people may see your good works and glorify your Father who is in heaven. It is here that the children of God are often lukewarm and indifferent, as they take comfort in blessed memories from the first time of awakening and childhood, and forget to apply all diligence to bring forth into the open, in the power of faith, but also in power, knowledge, self-control, perseverance, godliness, brotherly love, and, as the crown of it all, love toward all, and to have these things in abundant measure and not remain idle or unfruitful in the knowledge of our Lord Jesus Christ, lest they become nearsighted, forgetting the cleansing from former sins, and thereby make firm both their calling and election and be preserved from falling (2 Peter 1).

A faith active in love is what God requires of us, so that we show mercy to the poor, the suffering, the helpless—not in word, but in deed—just as Christ has shown mercy to us, not with lofty and glorious speeches, but by giving his life for us; so we also are obligated to give our lives for the brothers, if our faith is not to die because it has no works (James 2).

This is the growth that is required, friend, if you are not to become unfruitful and cast out as a dry branch. But alas, you say, how shall I, so frail and powerless, be able to do these things, and if God requires this of me, how then can I be saved?

Alas, I say, how often do we not seek to slip away from God’s holy demand by all manner of empty excuses about our weakness and great sinfulness and frailty!

If you meant it sincerely, you would say with Paul: “When I am weak, then I am strong.” Your powerlessness to grow comes from this, that you seek the power for it in yourself instead of in him who gave birth to it in you, who cannot sin, he who laid the seed in you so that you received your life hidden with Christ in God and were made a partaker of the divine powers of eternal life, which were given to you undeservedly, without money and without payment, for the sake of Jesus Christ alone, and which therefore are called grace.

This is the power that brings the life of God in a human heart to grow and bear fruit, and this alone is the power. If you go daily in childlike faith to the fountains of grace, to Siloah’s quietly flowing streams of the cross, then you shall become like “a tree planted by streams of water, that yields its fruit in its season, and whose leaf does not wither,” and in all that you do you shall prosper, because it is done in God. It is God who works in us both to will and to do according to his good pleasure, and if I forget all that lies behind me and, emptied of all my own, let this power of grace work through the Holy Spirit in my heart, then you also, however small and disregarded you may be among people, may confess with Paul: “I can do all things through Christ who strengthens me.”

If, then, there is no progress or growth in your life of God—no more inward fellowship with God in prayer and in the knowledge of the Word, no stronger will to renounce your own and to confess Jesus in word and deed, however slowly it may go—then you must seek the cause not in any lack of love or patience on God’s part, but in your own unwillingness to give room in your heart to the power of his grace. There is danger that you grow lukewarm, become weary in the struggle, or find it less necessary, become indifferent about watching, or fall into spiritual pride and fleshly self-confidence. For the Lord will not only begin but also complete; the Lord does not grow weary or tired in his work of grace, and when your sin has abounded, grace has abounded all the more.

Therefore, when this happens to you, that there is stagnation or sluggishness in your spiritual growth, then first look within yourself and test whether you stand in the faith, then look upward toward heaven and toward the goal that has been set for you: to bear fruit for the harvest in its time.

For God has also given each individual human being a certain span of time—according to his own reckoning of time (2 Peter 3:8)—a time of grace in which he imparts the grace he has determined, and after which he requires the fruit which his gracious power, according to the measure allotted to each, can and shall bring forth. Then comes the day of reckoning, when it shall be rendered to us according to what has been done through the body, whether good or evil: for those who remain steadfast in the faith, eternal blessedness; for those who have squandered grace, eternal perdition: “When the fruit is ripe, he immediately swings the sickle, because the harvest is at hand.”

Therefore Paul says to Timothy (2 Timothy 4:5): “Be watchful in all things, endure suffering—fulfill your ministry,” and adds concerning himself: “I am already being poured out, and the time of my departure is at hand; I have fought the good fight, I have finished the course—and the crown of righteousness is laid up for me, and not only for me, but for all who have loved his appearing.” And in another place he says that he counts everything as nothing and does not even hold his life dear, in order that he may finish his course with joy—the course he compares to the marked-out distance that must be run on a racecourse, where only the one who finishes receives the prize.

This is something for each of us, and for every day, to take to heart:

Begun is not finished—\\
Mark this well:\\
You who know your Jesus,\\
Press on.\\
(Hans Adolph Brorson)

God will require a soul from us and has set a boundary of time when the account must be settled, sometimes swiftly as with the rich farmer—“Tonight you must die!”—sometimes somewhat extended—“Spare the tree yet one year!”—but always, humanly speaking, short: today, and for the great multitude of people, unexpected. The Lord will require fruit—the naked clothed, the hungry fed, the wretched comforted—so that he may say to you, humble and sincere as the one who is saved, because he knows it is all of grace: “See, this you have done to me, for you have done it to the least of my little ones.”

O friends, let us never lose sight of this holy and blessed goal; it hastens toward the end, and there is no time to waste. Let the power of God’s grace through a childlike faith therefore be given great room in our hearts; let us daily examine ourselves thoroughly and see whether we have oil in our lamps—so that when the day comes, we may not, with shame, be left standing outside. For if the righteous, those justified by grace, are scarcely saved, how shall the ungodly, the lukewarm, the sluggish, the apostate come through?



