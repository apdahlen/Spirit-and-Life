
\section{Fifth Sunday after Easter: Our Father}

\begin{quote}
Matthew 6:5–13: When you pray, you shall not be as the hypocrites; for they love to stand and pray in the synagogues and at the corners of the streets, that they may be seen by men; truly I say to you, they have received their reward. But you, when you pray, go into your chamber, and shut your door, and pray to your Father who is in secret; and your Father who sees in secret shall reward you openly. And when you pray, do not heap up empty words as the heathen do; for they think that they shall be heard for their many words. Therefore you shall not be like them; for your Father knows what you have need of before you ask him. Therefore you shall pray thus: Our Father, in heaven! Hallowed be your Name! Your kingdom come! Your will be done, as in heaven, so also upon earth! Give us this day our daily bread! And forgive us our debt, as we also forgive our debtors! And lead us not into temptation! But deliver us from evil! For yours is the kingdom and the power and the glory for ever. Amen.
\end{quote}

\bigskip


Jesus warns against the prayer of the hypocrite, which is held in the synagogues and at the street corners for the sake of men. The heavenly Father does not hear such prayer. If prayer does not proceed from your heart, then it does not reach the Father’s heart. Prayer belongs to the hidden man of the heart, to a quiet and imperishable spirit, which is very precious before God. Where such prayer is found, so that a human heart is turned toward God daily, early, and without ceasing, there the Apostle’s exhortation is fulfilled: “Pray without ceasing.” In such a life there is also a hidden power which bears up under the tribulations and toil of earthly life and gives that heart-peace which is the open recompense that accompanies the hidden life in God.

The Savior also warns against the heathen’s empty words in prayer. The many words dishonor the Father; for he knows his children’s needs before they ask him. It is not he who needs to hear our long explanations of all that we lack; it is we who need the Father’s help. Like the disciples’ cry, ‘Lord, save us, we perish!’ such is the prayer of those who truly cry out of the depths to the Lord. In the hour of distress the cry of need is short and piercing; and he who is not in distress does not pray. Distress is the mother of prayer; therefore prayer is so brief in words and so certain to press forward to the Father’s merciful ear and compassionate heart. If you have no distress and no need, this is the greatest misery; cry to the Lord even over this, that he may show you your wretchedness, and you shall be heard.

But his disciples Jesus teaches to pray, “Our Father.” And with good reason both long and short explanations have been written over this prayer; and yet its content is not exhausted and its glory not dimmed. Still it is for all needy children of God equally beautiful and equally rich, equally weighty and humbling, equally blessed and uplifting. Gladly would we also, though with stammering tongue, speak a few words concerning Our Father.

And chiefly this is to be said: he who can truly say the first words, “Our Father, in heaven,” he has come far in the art of praying aright. For a poor sinner, who has every reason, like the tax collector, to strike his breast and turn his eyes toward the earth, it is an unspeakable mystery of grace to dare to turn to the Lord, who sits enthroned on high and is the Holy One, and say: “Abba, Father.” Had he not himself taught us this, the blessed Son of God, through whom we have access to the Father; had not the Spirit of the Lord himself taught us this in our hearts’ brokenness and shame, we would never have found our way to it. It is the Son and the Spirit alone who place the Abba-cry into the sinner’s heart and give boldness to draw near to the throne of grace. O that my heart might rightly cling to this blessed, lovely Name, so that with full confidence I might lay my cause into God’s hand; then all else would be light and bright; for if he who governs all is my Father, what else can meet me but love and mercy upon all my ways? All prayer will be easy when I am asking my Father. Were there more real, sincere, and simple faith in the Father, then the life of God’s children would be less sighing and complaining, more full of heavenly peace and joy. Then all the following petitions in the Lord’s Prayer would become more truth in our heart and in our mouth.

For if we have first rightly grasped the Father-name in prayer and faith, then it is easy for the Spirit—though hard for the flesh—to pray: “Hallowed be your Name, your kingdom come, your will be done!” For even though our flesh rises up in us and desires its own glory, its own power, and its own will—yet for the man of God in us, who is born of God, the only blessedness, peace, and rest lie herein: that God is glorified, that his kingdom comes, and his will is done. Bow yourself, heart, and learn to pray: Not my will, but yours be done! Though it must happen with tears and in fear and anguish as for the Savior in Gethsemane—if it is your heavenly Father’s will under which you bow, it is nevertheless your best, it is nevertheless your good. But therefore so many prayers are not fulfilled, because we pray: Our will be done! O no, Father, not my desire, but yours; not my foolish, unwise, sinful will, but your good and wise and perfect will—let that be done, let that be done now and always in me and with me; then I know it will go well with me; for you, Father, do not give your child anything evil.

But if you have truly wrestled thus far in prayer, and have received grace to believe this fully and wholly, then you shall also receive boldness to pray the four last petitions: “Give us this day our daily bread, and forgive us our debt, as we also forgive our debtors, and lead us not into temptation, but deliver us from evil!”

For truly, you have nothing but need and guilt and distress to lay before God in these petitions; yet if it is to a gracious Father you come, if it is with full faith in his grace, if it is with full surrender to his will, then it is nevertheless safe to come, though you come empty-handed, indebted, tempted, and suffering. See, it is the Father’s delight and joy to give you all that you need. As it is your childlike joy that your Father be exalted and glorified, so it is his fatherly joy to be able to shower you with good gifts, according as you need and can bear. Remember well that you are a very fragile vessel so long as you live in this earthly life; and thank the Lord in humility that he does not give you according to your folly, but according to his wisdom. In his time he who must now cautiously lay heavenly gifts into your soul shall bestow upon you an eternal and beyond all measure weighty glory, when all danger is over, when all temptation is past.

Then you may confidently set Amen to this prayer. If you prayed it in faith as a child to his Father, it is fulfilled and it shall be fulfilled. Ask all God’s children, and they will tell you so. With shame and blushing they will confess that it often happened to them that they did not make it through their Our Father, because it became too great for them and their heart became too full; but so long as they received grace to pray, so long they also found answer to prayer, and from day to day there came response from the heavenly Father to their stammering prayer.

Thus be the Lord praised for “Our Father” and for all the peace and blessedness which this prayer has given our hearts. Thanks be to the Holy Spirit, who loosed our bound tongue and taught us to cry: Abba, Father!

And as this prayer has gone with light and strength down through the centuries from the days of the Lord’s flesh until now, so shall it also remain all believers’ comfort and joy until the Lord comes again. And though the proud unbelief of the world may silence many tongues, the Lord shall nevertheless through the Lord’s Prayer raise up a power on earth from the mouths of babes and sucklings, whom a pious mother taught the Lord’s own prayer.

