

\section{Mid-Lent Sunday: Life from God’s Life}


\begin{quote}

John 6:52–65. Then the Jews disputed among themselves, saying, “How can this man give us his flesh to eat?” Jesus therefore said to them, “Truly, truly, I say to you: unless you eat the flesh of the Son of Man and drink his blood, you do not have life in yourselves. Whoever eats my flesh and drinks my blood has eternal life, and I will raise him up on the last day. For my flesh is truly food, and my blood is truly drink. Whoever eats my flesh and drinks my blood abides in me, and I in him. Just as the living Father has sent me, and I live because of the Father, so also the one who eats of me shall live through me. This is the bread that came down from heaven; not as your fathers ate the manna and died; whoever eats this bread shall live forever.” These things he said in a synagogue, where he was teaching in Capernaum. Many of his disciples, when they heard it, said, “This is a hard saying; who can bear him?” But Jesus, knowing in himself that his disciples were murmuring about this, said to them, “Does this offend you? What then if you were to see the Son of Man ascending to where he was before? It is the Spirit who gives life; the flesh is of no avail; the words that I speak to you are spirit and are life. But there are some of you who do not believe.” For Jesus knew from the beginning who they were who did not believe, and who it was who would betray him. And he said, “For this reason I told you that no one can come to me unless it is given him by my Father.”

\end{quote}

\bigskip


Jesus had said: “I am the bread of life,” and again: “I am the living bread that came down from heaven; if anyone eats of this bread, he shall live forever; and the bread that I will give is my flesh, which I will give for the life of the world.”

This last saying offended the Jews. Alarmed and offended, they ask: “How can this man give us his flesh to eat?” The Jews think: this is impossible. Thus Nicodemus had also thought concerning the new birth: this is impossible; an old man cannot be born again.

But Jesus had said to Nicodemus that it was just as necessary for salvation. And in the same way he answers the Jews here: it is absolutely necessary for salvation to eat my flesh and drink my blood; the one who does not do this has no life; the one who does it has eternal life, and I will raise him up on the last day.

Thus, what the Jews call impossible, Jesus calls necessary for eternal life, for redemption and salvation.

This is the teaching of Jesus. By nature we are flesh; but all flesh is grass, and all its glory as the flower of the grass. It is sinful; therefore it is perishable. All our life is subjected to death, and even our strongest efforts and our highest and noblest thoughts and deeds cannot lift us out of the power of sin and death that holds us down. Our life is torn away from God, who alone has imperishability; and cut off from the source of life, we are utterly powerless, either in ourselves or in the world, to find new sources of life from which we can draw eternal life.

But—God be eternally praised and glorified—what our sin has taken from us, Jesus brings us as a free gift of grace from God. Jesus will give us himself, so that we abide in him and he in us. And if this fellowship enters between us and him, then we receive life from God’s life; for then he says: “Just as the living Father has sent me, and I live because of the Father, so also the one who eats of me shall live through me.”

Thus there is indeed salvation from death; for Jesus is the mediator of life, who again gives poor sinners life from God’s life. In the midst of the perishable world there is a source of life and imperishability in Jesus, since his blood cleanses us from all our sins. Against sin and uncleanness there is an open fountain, a living well, in the death and resurrection of Jesus; therefore there is also eternal life in him. From the living Father he brings life—the eternal, imperishable, and incorruptible life—to the children of sin and death.

But how, how can this happen? How can we come into such fellowship with Jesus that his life becomes our life, and through him we receive life from God’s life? To “eat his flesh and drink his blood”—that is indeed “a hard saying; who can listen to it?”

Therefore Jesus adds a word that is mighty to open our dull hearts: “It is the Spirit who gives life; the flesh is of no avail; the words that I speak to you are spirit and are life.”

Eating and drinking with the mouth alone does nothing, just as hearing with the ears does nothing. Such outward connection with Jesus does not give a person eternal life. Spirit is required—Spirit from God, the Spirit of Jesus Christ—to establish the life-giving fellowship with Jesus Christ. Even among Jesus’ disciples, who followed him in the days of his flesh, who attached themselves to him in a fleshly manner, there were some who did not believe. Therefore their fellowship with Jesus availed them nothing. The Spirit had not united and bound them to him, so that they lived in him and he lived in them.

It is the Spirit who gives life; but the Spirit is in Christ’s word; therefore faith alone is the appropriation of Christ by which his life becomes our life, and we receive God’s life through him. Therefore there is no one who can come to the Son unless it is given him by the Father.

\textbf{There is no one who by his own strength or reason can believe in Christ or come to him.} It is no fleshly work to eat Christ’s flesh and drink his blood so that one thereby receives eternal life. It is a spiritual work, wrought in us by the Holy Spirit.

Thus the Spirit works faith; faith grasps Christ and unites with him in the fellowship of Spirit and life; and it is Christ’s word that is the means to give the Spirit and work faith. In this way our heart is united with Christ and our life with his life, and we receive eternal life in fellowship with him and with the Father.

Now then, friend, do you still live the life of the world, or do you live God’s life, the life in faith in the Son of God? This is eternal life, and there is absolutely no other. If you live in the world, then you die with the world; if you live in God, then you do not die, but you have eternal life abiding in you.

Let us hasten to the Son and grasp him in faith, and he who lives in the Father shall give us new and eternal life.

Life from God’s life.
