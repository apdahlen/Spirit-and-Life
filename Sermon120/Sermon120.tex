

\section{Second Sunday after Easter: The Shepherd and the Sheep}


\begin{quote}

John 10:1–10. Truly, truly, I say to you: He who does not enter by the door into the sheepfold, but climbs up some other way, he is a thief and a robber. But he who enters by the door is the shepherd of the sheep. To him the doorkeeper opens, and the sheep hear his voice; and he calls his own sheep by name and leads them out. And when he has driven out his own sheep, he goes before them; and the sheep follow him, because they know his voice. But a stranger they will not follow, but will flee from him, because they do not know the voice of strangers. This parable Jesus spoke to them; but they did not understand what it was that he spoke to them. Then Jesus said again to them: Truly, truly, I say to you: I am the door of the sheep. All who came before me are thieves and robbers; but the sheep did not hear them. I am the door; if anyone enters through me, he shall be saved, and shall go in and go out and find pasture. The thief comes only to steal and kill and destroy; I have come that they may have life and have abundance.

\end{quote}

\bigskip


Here we have from the Lord’s own mouth two parables which, with all their authoritative and arresting content and all their simplicity, were nevertheless incomprehensible to the Lord’s own disciples and have since caused the theological interpreters great difficulty. It was precisely on account of these things that Jesus had promised his disciples the Holy Spirit, who should teach them all things and remind them of all that he had spoken to them. And when the learned man suspiciously wrestles with the apparent lack of logic in the Word of God, the simple and immature leaps over all difficulties and finds also in these two parables green pastures and still flowing waters for refreshment and rest for his weary soul.

It appears irreconcilable with human reasoning that the Lord calls himself both “the door” and “the shepherd of the sheep” and “the doorkeeper.” But for the one who knows that there is no access to the Father except through the Son, that there is no entrance into heaven except through the crucified and risen body of Jesus, that there is no one who can give to him who hungers to eat or to him who thirsts to drink except Jesus, and that no one can defend and protect his sheep except he who has overcome the devil and taken his armor from him—for him it is in faith both rightly intelligible and exceedingly precious that Jesus calls himself at once the door and the shepherd of the sheep and the doorkeeper.

For first, there is only one door to the kingdom of heaven; human reason, self‑righteousness, and pride have indeed continually labored to fashion new and more convenient doors into the kingdom of God; yet there is but one through which one may enter and be saved: it is Jesus. Whoever comes into the house through him belongs there. He shall go in and go out. He is no longer under the yoke and bondage of the Law; he has the rights of a son and may claim his inheritance. Think what it means to be a child of God—a sheep of Jesus! To have God’s heaven as our house and home even while we are still here on earth, in this mortal life, so that whether we consider God’s Word or perform our earthly vocation, whether we pray or keep house, whether we go in or go out, sleep or wake, live or die—we belong to the Lord, have the full rights of children, and find our nourishment in the house. O why do so many seek for themselves other and deceptive doors of salvation, when this one stands open to them, if only they will go beneath the Cross and follow with Jesus from the tree of the curse to the opened grave! Jesus is not a door that is merely opened and shut. He gives the power to pass through. He is a living door, which gives new life to the dead, leads them through crucifixion and grave to resurrection and eternal life, not sparingly, but as grace upon grace, so that if sin has abounded, grace has abounded much more.

The Lord is therefore more than a door; he is a shepherd. Entering the Father’s house through Jesus isn’t like walking through an ordinary door. It’s not mechanical or impersonal. On the contrary, it is a living, personal relationship because he is a shepherd. He goes out into the wilderness to seek the lost sheep. When he has found them, he lays them upon his shoulders and bears them home with joy. There is a deep, personal bond between Jesus and those whom he saves, as between father and son, between brother and brother. Salvation doesn’t happen by belonging to a congregation or a society; it is nothing external, such as, for example, circumcision or belonging to an orthodox church or the like, that decides one’s matter of salvation. As the prophet Ezekiel taught his countrymen that he who eats sour grapes shall himself have sore teeth, so it is also in the new covenant that only he who believes in the Son has eternal life. We must each come to him, the precious Savior, and wait until he calls us by name and says: “Be of good cheer; your sins are forgiven.” Then you are his own; he calls you thereafter always by the same name; you may meet with him whenever you wish, more easily and intimately than with your closest friend, and he leads you out upon his beautiful pastures. Consider if a king or the President of the United States should seek you out, mention you by name, and speak familiarly and personally with you—how exalted you would feel, and how people round about would admire you and perhaps envy you! Now it is he who is King of kings, highly, highly exalted above them all, who comes to you, the most miserable and wretched of all his subjects, calls you by name, lifts you up to his throne, and says: “All this now belongs to you; you are my brother and shall inherit with me.”—The angels themselves marvel and sing for joy, and should it have taken place without your knowing it—you, whom it most nearly concerns? O friend, we do not come sleeping into Christ’s sheepfold; he calls us by name; have you heard it? Have you answered him? Or have you despised him, crucified him anew, and trampled his holy blood beneath your feet? O no, friend, there is nothing impersonal in salvation; it concerns you—to be called by name, and to cast yourself upon the Savior’s breast as upon a good shepherd and a dear Brother, and thereafter to follow him.

For he goes before and shows us the green pastures; as he has made himself known to us, so must we become familiar with him, know him and obey his blessed voice as the bride when she is called by her bridegroom. There is a deep bond of love and trust between Jesus and the pardoned sinner; they do not mistake his voice; when they follow him, they do not stray, as it is written that not even the fool shall err therein; and now that we know God, yea rather are known of him, how should we again turn back to the weak and beggarly elements to be once more in bondage under them? (Gal. 4:9.)

But he is also “the doorkeeper.” Wolves and dogs pursue the soul and would devour it; thieves and robbers seek after it; strangers and deceivers would lead it astray; and where shall the poor, miserable, helpless sheep go? “Fear not, little flock, for it is your Father’s good pleasure to give you the kingdom.” Only come, fearful soul, and flee close up to Jesus when the enemies press upon you—hide yourself by his bosom, in his wounds, when the enemies persecute you. He stands guard at the door, and no one can enter without his permission; once you are within his enclosure, you are secure. In temptation, in the struggle of the flesh, under the anxiety of losing the inheritance of life, he is as a rock and a horn for Israel. For who shall separate me from the love of Christ? Tribulation, or distress, or persecution, or famine, or nakedness, or peril, or sword? O no, I am persuaded that neither death nor life, neither angels nor principalities nor powers nor any other creature shall be able to separate us from the love of God in Christ Jesus our Lord. Truly, he is both the door and the shepherd and the doorkeeper, a strong and loving doorkeeper who can preserve my soul until he brings it home in incorruptibility. Friend, have you gone through the door—is Jesus your shepherd and your doorkeeper?

